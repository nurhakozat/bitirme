% Default to the notebook output style

    


% Inherit from the specified cell style.




    
\documentclass[11pt]{article}

    
    
    \usepackage[T1]{fontenc}
    % Nicer default font (+ math font) than Computer Modern for most use cases
    \usepackage{mathpazo}

    % Basic figure setup, for now with no caption control since it's done
    % automatically by Pandoc (which extracts ![](path) syntax from Markdown).
    \usepackage{graphicx}
    % We will generate all images so they have a width \maxwidth. This means
    % that they will get their normal width if they fit onto the page, but
    % are scaled down if they would overflow the margins.
    \makeatletter
    \def\maxwidth{\ifdim\Gin@nat@width>\linewidth\linewidth
    \else\Gin@nat@width\fi}
    \makeatother
    \let\Oldincludegraphics\includegraphics
    % Set max figure width to be 80% of text width, for now hardcoded.
    \renewcommand{\includegraphics}[1]{\Oldincludegraphics[width=.8\maxwidth]{#1}}
    % Ensure that by default, figures have no caption (until we provide a
    % proper Figure object with a Caption API and a way to capture that
    % in the conversion process - todo).
    \usepackage{caption}
    \DeclareCaptionLabelFormat{nolabel}{}
    \captionsetup{labelformat=nolabel}

    \usepackage{adjustbox} % Used to constrain images to a maximum size 
    \usepackage{xcolor} % Allow colors to be defined
    \usepackage{enumerate} % Needed for markdown enumerations to work
    \usepackage{geometry} % Used to adjust the document margins
    \usepackage{amsmath} % Equations
    \usepackage{amssymb} % Equations
    \usepackage{textcomp} % defines textquotesingle
    % Hack from http://tex.stackexchange.com/a/47451/13684:
    \AtBeginDocument{%
        \def\PYZsq{\textquotesingle}% Upright quotes in Pygmentized code
    }
    \usepackage{upquote} % Upright quotes for verbatim code
    \usepackage{eurosym} % defines \euro
    \usepackage[mathletters]{ucs} % Extended unicode (utf-8) support
    \usepackage[utf8x]{inputenc} % Allow utf-8 characters in the tex document
    \usepackage{fancyvrb} % verbatim replacement that allows latex
    \usepackage{grffile} % extends the file name processing of package graphics 
                         % to support a larger range 
    % The hyperref package gives us a pdf with properly built
    % internal navigation ('pdf bookmarks' for the table of contents,
    % internal cross-reference links, web links for URLs, etc.)
    \usepackage{hyperref}
    \usepackage{longtable} % longtable support required by pandoc >1.10
    \usepackage{booktabs}  % table support for pandoc > 1.12.2
    \usepackage[inline]{enumitem} % IRkernel/repr support (it uses the enumerate* environment)
    \usepackage[normalem]{ulem} % ulem is needed to support strikethroughs (\sout)
                                % normalem makes italics be italics, not underlines
    

    
    
    % Colors for the hyperref package
    \definecolor{urlcolor}{rgb}{0,.145,.698}
    \definecolor{linkcolor}{rgb}{.71,0.21,0.01}
    \definecolor{citecolor}{rgb}{.12,.54,.11}

    % ANSI colors
    \definecolor{ansi-black}{HTML}{3E424D}
    \definecolor{ansi-black-intense}{HTML}{282C36}
    \definecolor{ansi-red}{HTML}{E75C58}
    \definecolor{ansi-red-intense}{HTML}{B22B31}
    \definecolor{ansi-green}{HTML}{00A250}
    \definecolor{ansi-green-intense}{HTML}{007427}
    \definecolor{ansi-yellow}{HTML}{DDB62B}
    \definecolor{ansi-yellow-intense}{HTML}{B27D12}
    \definecolor{ansi-blue}{HTML}{208FFB}
    \definecolor{ansi-blue-intense}{HTML}{0065CA}
    \definecolor{ansi-magenta}{HTML}{D160C4}
    \definecolor{ansi-magenta-intense}{HTML}{A03196}
    \definecolor{ansi-cyan}{HTML}{60C6C8}
    \definecolor{ansi-cyan-intense}{HTML}{258F8F}
    \definecolor{ansi-white}{HTML}{C5C1B4}
    \definecolor{ansi-white-intense}{HTML}{A1A6B2}

    % commands and environments needed by pandoc snippets
    % extracted from the output of `pandoc -s`
    \providecommand{\tightlist}{%
      \setlength{\itemsep}{0pt}\setlength{\parskip}{0pt}}
    \DefineVerbatimEnvironment{Highlighting}{Verbatim}{commandchars=\\\{\}}
    % Add ',fontsize=\small' for more characters per line
    \newenvironment{Shaded}{}{}
    \newcommand{\KeywordTok}[1]{\textcolor[rgb]{0.00,0.44,0.13}{\textbf{{#1}}}}
    \newcommand{\DataTypeTok}[1]{\textcolor[rgb]{0.56,0.13,0.00}{{#1}}}
    \newcommand{\DecValTok}[1]{\textcolor[rgb]{0.25,0.63,0.44}{{#1}}}
    \newcommand{\BaseNTok}[1]{\textcolor[rgb]{0.25,0.63,0.44}{{#1}}}
    \newcommand{\FloatTok}[1]{\textcolor[rgb]{0.25,0.63,0.44}{{#1}}}
    \newcommand{\CharTok}[1]{\textcolor[rgb]{0.25,0.44,0.63}{{#1}}}
    \newcommand{\StringTok}[1]{\textcolor[rgb]{0.25,0.44,0.63}{{#1}}}
    \newcommand{\CommentTok}[1]{\textcolor[rgb]{0.38,0.63,0.69}{\textit{{#1}}}}
    \newcommand{\OtherTok}[1]{\textcolor[rgb]{0.00,0.44,0.13}{{#1}}}
    \newcommand{\AlertTok}[1]{\textcolor[rgb]{1.00,0.00,0.00}{\textbf{{#1}}}}
    \newcommand{\FunctionTok}[1]{\textcolor[rgb]{0.02,0.16,0.49}{{#1}}}
    \newcommand{\RegionMarkerTok}[1]{{#1}}
    \newcommand{\ErrorTok}[1]{\textcolor[rgb]{1.00,0.00,0.00}{\textbf{{#1}}}}
    \newcommand{\NormalTok}[1]{{#1}}
    
    % Additional commands for more recent versions of Pandoc
    \newcommand{\ConstantTok}[1]{\textcolor[rgb]{0.53,0.00,0.00}{{#1}}}
    \newcommand{\SpecialCharTok}[1]{\textcolor[rgb]{0.25,0.44,0.63}{{#1}}}
    \newcommand{\VerbatimStringTok}[1]{\textcolor[rgb]{0.25,0.44,0.63}{{#1}}}
    \newcommand{\SpecialStringTok}[1]{\textcolor[rgb]{0.73,0.40,0.53}{{#1}}}
    \newcommand{\ImportTok}[1]{{#1}}
    \newcommand{\DocumentationTok}[1]{\textcolor[rgb]{0.73,0.13,0.13}{\textit{{#1}}}}
    \newcommand{\AnnotationTok}[1]{\textcolor[rgb]{0.38,0.63,0.69}{\textbf{\textit{{#1}}}}}
    \newcommand{\CommentVarTok}[1]{\textcolor[rgb]{0.38,0.63,0.69}{\textbf{\textit{{#1}}}}}
    \newcommand{\VariableTok}[1]{\textcolor[rgb]{0.10,0.09,0.49}{{#1}}}
    \newcommand{\ControlFlowTok}[1]{\textcolor[rgb]{0.00,0.44,0.13}{\textbf{{#1}}}}
    \newcommand{\OperatorTok}[1]{\textcolor[rgb]{0.40,0.40,0.40}{{#1}}}
    \newcommand{\BuiltInTok}[1]{{#1}}
    \newcommand{\ExtensionTok}[1]{{#1}}
    \newcommand{\PreprocessorTok}[1]{\textcolor[rgb]{0.74,0.48,0.00}{{#1}}}
    \newcommand{\AttributeTok}[1]{\textcolor[rgb]{0.49,0.56,0.16}{{#1}}}
    \newcommand{\InformationTok}[1]{\textcolor[rgb]{0.38,0.63,0.69}{\textbf{\textit{{#1}}}}}
    \newcommand{\WarningTok}[1]{\textcolor[rgb]{0.38,0.63,0.69}{\textbf{\textit{{#1}}}}}
    
    
    % Define a nice break command that doesn't care if a line doesn't already
    % exist.
    \def\br{\hspace*{\fill} \\* }
    % Math Jax compatability definitions
    \def\gt{>}
    \def\lt{<}
    % Document parameters
    
    \title{Veri Biliminde Konular: Zaman Serilerinde Anomali ve Öngörü}
      

    % Pygments definitions
    
\makeatletter
\def\PY@reset{\let\PY@it=\relax \let\PY@bf=\relax%
    \let\PY@ul=\relax \let\PY@tc=\relax%
    \let\PY@bc=\relax \let\PY@ff=\relax}
\def\PY@tok#1{\csname PY@tok@#1\endcsname}
\def\PY@toks#1+{\ifx\relax#1\empty\else%
    \PY@tok{#1}\expandafter\PY@toks\fi}
\def\PY@do#1{\PY@bc{\PY@tc{\PY@ul{%
    \PY@it{\PY@bf{\PY@ff{#1}}}}}}}
\def\PY#1#2{\PY@reset\PY@toks#1+\relax+\PY@do{#2}}

\expandafter\def\csname PY@tok@w\endcsname{\def\PY@tc##1{\textcolor[rgb]{0.73,0.73,0.73}{##1}}}
\expandafter\def\csname PY@tok@c\endcsname{\let\PY@it=\textit\def\PY@tc##1{\textcolor[rgb]{0.25,0.50,0.50}{##1}}}
\expandafter\def\csname PY@tok@cp\endcsname{\def\PY@tc##1{\textcolor[rgb]{0.74,0.48,0.00}{##1}}}
\expandafter\def\csname PY@tok@k\endcsname{\let\PY@bf=\textbf\def\PY@tc##1{\textcolor[rgb]{0.00,0.50,0.00}{##1}}}
\expandafter\def\csname PY@tok@kp\endcsname{\def\PY@tc##1{\textcolor[rgb]{0.00,0.50,0.00}{##1}}}
\expandafter\def\csname PY@tok@kt\endcsname{\def\PY@tc##1{\textcolor[rgb]{0.69,0.00,0.25}{##1}}}
\expandafter\def\csname PY@tok@o\endcsname{\def\PY@tc##1{\textcolor[rgb]{0.40,0.40,0.40}{##1}}}
\expandafter\def\csname PY@tok@ow\endcsname{\let\PY@bf=\textbf\def\PY@tc##1{\textcolor[rgb]{0.67,0.13,1.00}{##1}}}
\expandafter\def\csname PY@tok@nb\endcsname{\def\PY@tc##1{\textcolor[rgb]{0.00,0.50,0.00}{##1}}}
\expandafter\def\csname PY@tok@nf\endcsname{\def\PY@tc##1{\textcolor[rgb]{0.00,0.00,1.00}{##1}}}
\expandafter\def\csname PY@tok@nc\endcsname{\let\PY@bf=\textbf\def\PY@tc##1{\textcolor[rgb]{0.00,0.00,1.00}{##1}}}
\expandafter\def\csname PY@tok@nn\endcsname{\let\PY@bf=\textbf\def\PY@tc##1{\textcolor[rgb]{0.00,0.00,1.00}{##1}}}
\expandafter\def\csname PY@tok@ne\endcsname{\let\PY@bf=\textbf\def\PY@tc##1{\textcolor[rgb]{0.82,0.25,0.23}{##1}}}
\expandafter\def\csname PY@tok@nv\endcsname{\def\PY@tc##1{\textcolor[rgb]{0.10,0.09,0.49}{##1}}}
\expandafter\def\csname PY@tok@no\endcsname{\def\PY@tc##1{\textcolor[rgb]{0.53,0.00,0.00}{##1}}}
\expandafter\def\csname PY@tok@nl\endcsname{\def\PY@tc##1{\textcolor[rgb]{0.63,0.63,0.00}{##1}}}
\expandafter\def\csname PY@tok@ni\endcsname{\let\PY@bf=\textbf\def\PY@tc##1{\textcolor[rgb]{0.60,0.60,0.60}{##1}}}
\expandafter\def\csname PY@tok@na\endcsname{\def\PY@tc##1{\textcolor[rgb]{0.49,0.56,0.16}{##1}}}
\expandafter\def\csname PY@tok@nt\endcsname{\let\PY@bf=\textbf\def\PY@tc##1{\textcolor[rgb]{0.00,0.50,0.00}{##1}}}
\expandafter\def\csname PY@tok@nd\endcsname{\def\PY@tc##1{\textcolor[rgb]{0.67,0.13,1.00}{##1}}}
\expandafter\def\csname PY@tok@s\endcsname{\def\PY@tc##1{\textcolor[rgb]{0.73,0.13,0.13}{##1}}}
\expandafter\def\csname PY@tok@sd\endcsname{\let\PY@it=\textit\def\PY@tc##1{\textcolor[rgb]{0.73,0.13,0.13}{##1}}}
\expandafter\def\csname PY@tok@si\endcsname{\let\PY@bf=\textbf\def\PY@tc##1{\textcolor[rgb]{0.73,0.40,0.53}{##1}}}
\expandafter\def\csname PY@tok@se\endcsname{\let\PY@bf=\textbf\def\PY@tc##1{\textcolor[rgb]{0.73,0.40,0.13}{##1}}}
\expandafter\def\csname PY@tok@sr\endcsname{\def\PY@tc##1{\textcolor[rgb]{0.73,0.40,0.53}{##1}}}
\expandafter\def\csname PY@tok@ss\endcsname{\def\PY@tc##1{\textcolor[rgb]{0.10,0.09,0.49}{##1}}}
\expandafter\def\csname PY@tok@sx\endcsname{\def\PY@tc##1{\textcolor[rgb]{0.00,0.50,0.00}{##1}}}
\expandafter\def\csname PY@tok@m\endcsname{\def\PY@tc##1{\textcolor[rgb]{0.40,0.40,0.40}{##1}}}
\expandafter\def\csname PY@tok@gh\endcsname{\let\PY@bf=\textbf\def\PY@tc##1{\textcolor[rgb]{0.00,0.00,0.50}{##1}}}
\expandafter\def\csname PY@tok@gu\endcsname{\let\PY@bf=\textbf\def\PY@tc##1{\textcolor[rgb]{0.50,0.00,0.50}{##1}}}
\expandafter\def\csname PY@tok@gd\endcsname{\def\PY@tc##1{\textcolor[rgb]{0.63,0.00,0.00}{##1}}}
\expandafter\def\csname PY@tok@gi\endcsname{\def\PY@tc##1{\textcolor[rgb]{0.00,0.63,0.00}{##1}}}
\expandafter\def\csname PY@tok@gr\endcsname{\def\PY@tc##1{\textcolor[rgb]{1.00,0.00,0.00}{##1}}}
\expandafter\def\csname PY@tok@ge\endcsname{\let\PY@it=\textit}
\expandafter\def\csname PY@tok@gs\endcsname{\let\PY@bf=\textbf}
\expandafter\def\csname PY@tok@gp\endcsname{\let\PY@bf=\textbf\def\PY@tc##1{\textcolor[rgb]{0.00,0.00,0.50}{##1}}}
\expandafter\def\csname PY@tok@go\endcsname{\def\PY@tc##1{\textcolor[rgb]{0.53,0.53,0.53}{##1}}}
\expandafter\def\csname PY@tok@gt\endcsname{\def\PY@tc##1{\textcolor[rgb]{0.00,0.27,0.87}{##1}}}
\expandafter\def\csname PY@tok@err\endcsname{\def\PY@bc##1{\setlength{\fboxsep}{0pt}\fcolorbox[rgb]{1.00,0.00,0.00}{1,1,1}{\strut ##1}}}
\expandafter\def\csname PY@tok@kc\endcsname{\let\PY@bf=\textbf\def\PY@tc##1{\textcolor[rgb]{0.00,0.50,0.00}{##1}}}
\expandafter\def\csname PY@tok@kd\endcsname{\let\PY@bf=\textbf\def\PY@tc##1{\textcolor[rgb]{0.00,0.50,0.00}{##1}}}
\expandafter\def\csname PY@tok@kn\endcsname{\let\PY@bf=\textbf\def\PY@tc##1{\textcolor[rgb]{0.00,0.50,0.00}{##1}}}
\expandafter\def\csname PY@tok@kr\endcsname{\let\PY@bf=\textbf\def\PY@tc##1{\textcolor[rgb]{0.00,0.50,0.00}{##1}}}
\expandafter\def\csname PY@tok@bp\endcsname{\def\PY@tc##1{\textcolor[rgb]{0.00,0.50,0.00}{##1}}}
\expandafter\def\csname PY@tok@fm\endcsname{\def\PY@tc##1{\textcolor[rgb]{0.00,0.00,1.00}{##1}}}
\expandafter\def\csname PY@tok@vc\endcsname{\def\PY@tc##1{\textcolor[rgb]{0.10,0.09,0.49}{##1}}}
\expandafter\def\csname PY@tok@vg\endcsname{\def\PY@tc##1{\textcolor[rgb]{0.10,0.09,0.49}{##1}}}
\expandafter\def\csname PY@tok@vi\endcsname{\def\PY@tc##1{\textcolor[rgb]{0.10,0.09,0.49}{##1}}}
\expandafter\def\csname PY@tok@vm\endcsname{\def\PY@tc##1{\textcolor[rgb]{0.10,0.09,0.49}{##1}}}
\expandafter\def\csname PY@tok@sa\endcsname{\def\PY@tc##1{\textcolor[rgb]{0.73,0.13,0.13}{##1}}}
\expandafter\def\csname PY@tok@sb\endcsname{\def\PY@tc##1{\textcolor[rgb]{0.73,0.13,0.13}{##1}}}
\expandafter\def\csname PY@tok@sc\endcsname{\def\PY@tc##1{\textcolor[rgb]{0.73,0.13,0.13}{##1}}}
\expandafter\def\csname PY@tok@dl\endcsname{\def\PY@tc##1{\textcolor[rgb]{0.73,0.13,0.13}{##1}}}
\expandafter\def\csname PY@tok@s2\endcsname{\def\PY@tc##1{\textcolor[rgb]{0.73,0.13,0.13}{##1}}}
\expandafter\def\csname PY@tok@sh\endcsname{\def\PY@tc##1{\textcolor[rgb]{0.73,0.13,0.13}{##1}}}
\expandafter\def\csname PY@tok@s1\endcsname{\def\PY@tc##1{\textcolor[rgb]{0.73,0.13,0.13}{##1}}}
\expandafter\def\csname PY@tok@mb\endcsname{\def\PY@tc##1{\textcolor[rgb]{0.40,0.40,0.40}{##1}}}
\expandafter\def\csname PY@tok@mf\endcsname{\def\PY@tc##1{\textcolor[rgb]{0.40,0.40,0.40}{##1}}}
\expandafter\def\csname PY@tok@mh\endcsname{\def\PY@tc##1{\textcolor[rgb]{0.40,0.40,0.40}{##1}}}
\expandafter\def\csname PY@tok@mi\endcsname{\def\PY@tc##1{\textcolor[rgb]{0.40,0.40,0.40}{##1}}}
\expandafter\def\csname PY@tok@il\endcsname{\def\PY@tc##1{\textcolor[rgb]{0.40,0.40,0.40}{##1}}}
\expandafter\def\csname PY@tok@mo\endcsname{\def\PY@tc##1{\textcolor[rgb]{0.40,0.40,0.40}{##1}}}
\expandafter\def\csname PY@tok@ch\endcsname{\let\PY@it=\textit\def\PY@tc##1{\textcolor[rgb]{0.25,0.50,0.50}{##1}}}
\expandafter\def\csname PY@tok@cm\endcsname{\let\PY@it=\textit\def\PY@tc##1{\textcolor[rgb]{0.25,0.50,0.50}{##1}}}
\expandafter\def\csname PY@tok@cpf\endcsname{\let\PY@it=\textit\def\PY@tc##1{\textcolor[rgb]{0.25,0.50,0.50}{##1}}}
\expandafter\def\csname PY@tok@c1\endcsname{\let\PY@it=\textit\def\PY@tc##1{\textcolor[rgb]{0.25,0.50,0.50}{##1}}}
\expandafter\def\csname PY@tok@cs\endcsname{\let\PY@it=\textit\def\PY@tc##1{\textcolor[rgb]{0.25,0.50,0.50}{##1}}}

\def\PYZbs{\char`\\}
\def\PYZus{\char`\_}
\def\PYZob{\char`\{}
\def\PYZcb{\char`\}}
\def\PYZca{\char`\^}
\def\PYZam{\char`\&}
\def\PYZlt{\char`\<}
\def\PYZgt{\char`\>}
\def\PYZsh{\char`\#}
\def\PYZpc{\char`\%}
\def\PYZdl{\char`\$}
\def\PYZhy{\char`\-}
\def\PYZsq{\char`\'}
\def\PYZdq{\char`\"}
\def\PYZti{\char`\~}
% for compatibility with earlier versions
\def\PYZat{@}
\def\PYZlb{[}
\def\PYZrb{]}
\makeatother


    % Exact colors from NB
    \definecolor{incolor}{rgb}{0.0, 0.0, 0.5}
    \definecolor{outcolor}{rgb}{0.545, 0.0, 0.0}



    
    % Prevent overflowing lines due to hard-to-break entities
    \sloppy 
    % Setup hyperref package
    \hypersetup{
      breaklinks=true,  % so long urls are correctly broken across lines
      colorlinks=true,
      urlcolor=urlcolor,
      linkcolor=linkcolor,
      citecolor=citecolor,
      }
    % Slightly bigger margins than the latex defaults
    
    \geometry{verbose,tmargin=1in,bmargin=1in,lmargin=1in,rmargin=1in}
    
    
\begin{document}

\author{Nurhak Özat}    
\maketitle
\pagebreak

\tableofcontents
\pagebreak
    
    
    \begin{Verbatim}[commandchars=\\\{\}]
{\color{incolor}In [{\color{incolor}1}]:} \PY{c+c1}{\PYZsh{} install.packages(\PYZdq{}Quandl\PYZdq{}, repos=\PYZsq{}http://cran.us.r\PYZhy{}project.org\PYZsq{})}
        \PY{c+c1}{\PYZsh{}install.packages(\PYZdq{}forecast\PYZdq{}, repos=\PYZsq{}http://cran.us.r\PYZhy{}project.org\PYZsq{})}
        \PY{k+kn}{library}\PY{p}{(}Quandl\PY{p}{)}
        \PY{k+kn}{library}\PY{p}{(}forecast\PY{p}{)}
        \PY{k+kn}{library}\PY{p}{(}tseries\PY{p}{)}
\end{Verbatim}


    \begin{Verbatim}[commandchars=\\\{\}]
Warning message:
"package 'Quandl' was built under R version 3.4.3"Loading required package: xts
Loading required package: zoo

Attaching package: 'zoo'

The following objects are masked from 'package:base':

    as.Date, as.Date.numeric

Warning message:
"package 'forecast' was built under R version 3.4.3"Warning message:
"package 'tseries' was built under R version 3.4.3"
    \end{Verbatim}

    \section{Zaman Serisi ve Temel
Kavramlar}\label{zaman-serisi-ve-temel-kavramlar}

Zaman serisi, bir değişkenin zaman içindeki hareketini gözlemleyen,
gözlem sonuçlarının zamana göre dağılım gösterdiği serilerdir. Bütün
değerleri belirli bir zman farkı ile art arda gelen nümerik verilerden
oluşur. Bu değerler günlük, haftalık, aylık, yıllık olabilirler.

Zaman serisi değişkeni, n zaman serisindeki veri sayısı olmak üzere;
x(t), (t= 1, 2, \ldots{}, n) - ilk gözlemlenen veri x(1), - ikinci
gözlemlenen veri x(2), - son gözlemlenen veri x(n) ile ifade edilir.

Sürekli Zaman Serileri; zaman içinde sürekli olarak kaydedilebilen
verilere sahip serilerdir elektrik sinyalleri, voltaj, ses titreşimleri
gibi mühendislik alanlarına ait serilerdir.

Kesikli Zaman Serileri; belli aralıklarda elde edilebilen verilere sahip
serilerdir. faiz oranı, satış hacmi, üretim miktarı gibi iktisadi
serilerdir.

Zaman serilerinde veriler nicel ya da nitel verilerden oluşur.

\begin{itemize}
\item
  Nitel ( Kalitatif ) Yöntemler : Öngörü işlemi ile bilgilerden
  yararlanılarak geleceğe ait tahmin yapılmaktadır. Dolayısıyla, öngörü
  yöntemleri tecrübeye, kararlara, bilirkişinin düşüncelerine
  dayanmaktadır. Bu tür öngörü yöntemlerine genel olarak nitel (
  kalitatif ) yöntemler denir.
\item
  Nicel ( Kantitatif ) Yöntemler : Öngörü yöntemleri subjectif
  kararlardan ziyade elde edilen verilerin yapısını açıklayabilen
  modellere dayanırsa bu tür modellere dayanan öngörü yöntemlerine genel
  olarak nicel ( Kantitatif ) yöntemler denir.
\end{itemize}

Zaman Serisi analizinde,iki çeşit değişken sayısı bulunur.
\begin{itemize}
\tightlist
\item
  Tek Değişkenli ( Unvariate Time Series ): Eğer zaman serisi tek değişkenin zaman içerisindeki hareketi incelenir. ,

\item
  Çok Değişkenli ( Multivariate Time Series ): Eğer birden fazla değişkenin zaman içindeki değişimlerini incelendiğinde kullanılır.
\end{itemize}

\section{Zaman Serisi Bileşenleri :}\label{zaman-serisi-bileux15fenleri}

Zaman serilerinin gözlemlenen değerleri arasında artma, azalma ya da
belli dönemlerde sabit kalma gibi değişmeler gerçekleşebilir. Bu
değişmeler çeşitli nedenleri olabilir. Bu değişmelerin incelenmesi , bir
öngürü yapılmak istendiğinde yardımcı olur. Nedeni ise zaman serileri
gelecekte de benzer özellikler gösterebilir. Bu değişmeler zaman
serilerinin bileşenlerini oluşturur. Bu değişmeler dört grupta
incelenir: 

- Trend bileşen 

- Mevsimsel bileşen

- Çevrimsel bileşen (Konjoktürel ) 

- Düzensiz Bileşen ( Rassal Bileşen )

\subsection{Trend :}\label{trend}

Zamana göre gözlemlenen bir değişkenin uzun dönemde gösterdiği artış
veya azalışa denir. İki farklı şekilde ifade edilebilir.Trend iki
şekilde olabilir: 

- Doğrusal Trend 

- Doğrusal Olmayan Trend

Gözlemlerin toplanış sırası ile aldığı değerler arasındaki korelasyonun
önemini belirlemek için trend analizleri yapılır. Trend analizi
matematiksel bir tekniktir ve bu analiz ile geçmişe ait sonuçları
kullanarak geleceğe yönelik sonuçlar için tahmin yapılır. Serinin gözlem
sayısı arttıkça trende sahip olmadığı daha kolay anlaşılır. Zaman
içerisinde artış veya azalış göstermeyen, aynı düzeyde kararlılık
gösteren serilerin ise trendi yoktur.

Trend doğrusal olabileceği gibi üstel de olabilir. Trend tamamen
öngörülebiliyorsa deterministik, tamamen öngörülemiyorsa rassaldır.

\subsection{Mevsimsel Bileşen:}\label{mevsimsel-bileux15fen}

Sistematik ( gözlemlenebilir ve ölçülebilir ) etkileri olan
faktörlerdir. Mevsim etkisinde olan zaman serileri yılın bazı
dönemlerinde diğer dönemlere göre daha yüksek ve ya daha düşük değerler
ulaşır. Örnek olarak; bazı dönemlerde doğalgaz kullanımının artması veya
azalması verilebilir.Mevsimsellik farklı biçimlerde ( Anneler Günü,
Babalar Günü, Öğretmenler günü , Dini bayramlar da yapılan alışverişteki
artışlar )gözlemlenebilir. Zaman serilerinde mevsimsellik etkisi bir yıl
ve daha az zaman süresinde tekrarlanan periyodik dalgalanmalar olarak
tanımlanır.

Zaman serisi analizlerinde mevsimsel etkinin olup olmadığını belirlemek
için bir çok yöntem kullanılır. Zaman serisinde mevsimsel bileşen varsa
, veriden mevsimsel dalgalanmaları arındırılması için en çok kullanılan
yöntemler mutlak ve nispi hareketli ortalamalar ve ortalama yüzde
yöntemleridir.

\subsection{Çevrimsel Bileşen:}\label{uxe7evrimsel-bileux15fen}

Mevsimsel değişimle ilişkisi olmayan daha uzun zaman aralığındaki
dönemsel değişimdir.Trend doğrusu veya eğrisinin etrafındaki uzun süreli
dalgalanmalardır. Örnek olarak; Ekonominin maksimum olduğu noktada
oluşan bir kriz ile birlkte ekonominin düşüşe başlamasını , devamında
belli bir süre hareketsizlik gözlenmesi ,ardından yeniden kıvılcımlarla
canlanmaya başlamasını ve bu sürecin tekrarlanıp devam etmesini
verebiliriz.

\subsection{Düzensiz Bileşen ( Rassal Bileşen
)}\label{duxfczensiz-bileux15fen-rassal-bileux15fen}

Rassal veya geçici olarak ortaya çıkan , trend , mevsimsel ve çevrimsel
bileşenlerin dışında kalan bileşenelere veya belirli bir modele uymayan
dalgalanmalara düzensiz bileşen denir. Değişimin zamanı ve şiddeti
tahmin edilemez. Serideki yanıltıcı hareketlerdir.

\section{Geleneksel Zaman Serisi Ayrışım
Yöntemleri}\label{geleneksel-zaman-serisi-ayrux131ux15fux131m-yuxf6ntemleri}

Serinin trend konjonktürel ve mevsimsel etkisi altında kaldığını
varsayılmaktadır. Toplamsal Ayrıştırma ( Additive )ve Çarpımsal
Araştırma ( Multiplicative ) yöntemleri olmak üzere iki yöntemi vardır.

Toplamsal Ayrıştırma ( Additive ) :

\[ \hat Y_{d_t} = T_{t}+ S_{t}+ C_{t}+ I_{t} \]

Çarpımsal Araştırma ( Multiplicative ):

\[ \hat Y_{d_t} = T_{t}\cdot S_{t}\cdot C_{t}\cdot I_{t} \]

    \section{Doğrusal Zaman Serisi
Analizi:}\label{doux11frusal-zaman-serisi-analizi}

\subsection{1- Otoregresif Süreç ( AutoRegressive Process
)}\label{otoregresif-suxfcreuxe7-autoregressive-process}

Otoregresif süreç AR ( p ) ile ifade edilir ve serinin mevcut değeri
önceki p adet değere bağlıdır.

p'ninci dereceden otoregresif süreç: \(Y_{t}\) değeri sadece bir önceki
değerine değil önceki bir çok değere bağlı olabilir.
\[ \hat Y_{t} = ϵ_{t}+a_{0}+ \sum_{i=1}^\ell a_{i}\cdot y_{t-i}  \]

Başka bir şekilde ifade edilirse, bağımlı değişken geçmişteki değerinin
bir fonksiyonudur.( p otoregresif sürecin derecesini vermektedir. )

a : sabit terim

\(a_{1} ….\) : Gecikmeli değerlerin şimdiki değerle olan ilişkisi

ε : hata terimi -- rassal şoklar

Otoregresif süreç durağan olma koşulu

\[ |a_n|\leq 1 \text{ ve } \sum_{i=1}^{n-1} a_i \leq 1 \]

olmasına bağlıdır.

Otoregresif modeli özetlemek gerekirse:

Tüm gecikmeli değerler \(Y_t\) üzerinde birikimli bir etkiye sahiptir.
Dolayısıyla otoregresif modellerin uzun süreli bir etkiye sahiptir.
Rassal değişken ile \(y_{t}\) arasındaki korelasyon sıfır olduğundan
otoregresif modelleri En küçük kareler yöntemini kullanarak tahmin
edilebilir. Otoregresif süreçlerde ACF gecikmeden sonra üstel olarak
azalmakta ve PACF değeri de sıfır değerini almaktadır.

\subsection{2- Hareketli Ortalama Süreci ( Moving Average
)}\label{hareketli-ortalama-suxfcreci-moving-average}

Serinin gecikmeli hata terimi, şu andaki hata terimini etkilemesine
hareketli ortalama süreci denir. Değişkenin tahmin değeri hata teriminin
tahmin değeri ile ilişkilidir.Modelin derecesi geçmiş dönemdeki hata
terim sayısına gore belirlenir. Örneğin geçmiş dönemde bir hata varsa
modelin derecesi 1, n tane hata varsa modelin derecesi n olmaktadır.
Hareketli ortalama sürece MA (q) olarak ifade edilir ve
\(Y_{t}\sim N(0,s^2_{e})\) olacak şekilde
\[ \hat Y_{d_t} = b_{0} + \sum_{i=1}^q b_i e_{t-i}  \]\\
kurulur.

\(b_{i}\): bilinmeyen sabit parametrelerdir.

\subsection{3- Karma Otoregresif Hareketli Ortalama Süreci
(ARMA)}\label{karma-otoregresif-hareketli-ortalama-suxfcreci-arma}

Zaman serilerinde veriler bazen hem otokorelasyon hem de kısmi
otokorelasyon fonksiyonlarını belirli bir gecikmede kesilmeden sıfıra
doğru çok yavaş hareket edebilir. Bu durumda seri hem otoregresif hem de
hareketli ortalama bileşenlerini aynı anda içerebilir. Özetle seri
modelini hem AR, hem de MA bileşenleriyle p ve q dereceden olmak üzere
ARMA(p,q) olarak tanımlanır

Arma sürecinin özetlemek gerekirse:

Karma otoregresif hareketli ortalama sürecinin ARMA (p,q ) otokorelasyon
fonksiyonu ve otoregresif sürecin AR(p) sürecinin özelliklerini taşır.

ARMA(p,q) modelinde büyük gecikme uzunluklarında ACF ve PACF değerleri
sıfıra yaklaşır. Model de durağanlığı AR(p) koşulu ile , tersine
çevirebilirliği ise MA(q) koşuluyla aynıdır.

\subsection{4- Homojen Durağan Olmayan Süreç
(ARIMA)}\label{homojen-duraux11fan-olmayan-suxfcreuxe7-arima}

Fark alma işlemiyle, durağan olmayan modellerin durağan hale
dönüştürülmüş serilere uygulanmasıyla elde edilen modellere ``durağan
olmayan doğrusal stokastik modeller'' denir. Bu tür serilere uygulanan
model ARIMA modelidir. Bu modeller d dereceden farkı alınmış serilere
uygulanan, değişkenin t-dönemindeki değerinin belirli sayıdaki geri
dönem değerleri ile aynı dönemdeki hata teriminin doğrusal bir
fonksiyonu olarak ifade edildiği AR ve değişkenin t-dönemindeki
değerinin aynı dönemdeki hata terimi ve belirli sayıda geri dönem hata
terimlerinin doğrusal fonksiyonu olarak ifade edildiği MA modellerinin
birer birleşimidir. Farklı bir şekilde ifade edilirse ARIMA,
otoregresresif (AR), entegre (I) ve hareketli ortalama (MA)
süreçlerinden oluşur. Modelin gösterimi ARIMA (p, d, q) şeklinde ifade
edilir. Burada p değişkeni otoregresif(AR) model derecesini, q değişkeni
hareketli ortalama (MA) derecesini ve d değişkeni de fark alma
derecesini gösterir.

Eğer d=1 ise orijinal seri birinci derecen farkı alınarak durağan hale
getirilmiştir, d=0 olduğunda ise orijinal seri zaten durağandır ve
farkının alınmasına da gerek yoktur. Bu durumda ARIMA(p,d=0,q)=ARMA(p,q)
olur.

Örnek olarak ARIMA(3,2,2) alınırsa; zaman serisini durağan olması için
farkının iki kez (d= 2) alınması gerekir. İlk farkı alınan zaman
serisini ARMA(2,2) süreci olarak yazılabilir. Bu süreçte otoregresif
modeli (AR) iki terimiyle ve hareketli ortalama modeli (MA) iki
terimiyle modellenebilir.
\pagebreak

Bilgi Kriterleri :

\begin{enumerate}
\def\labelenumi{\arabic{enumi}.}
\tightlist
\item
  Akaiki bilgi kriteri (Akaike information criterion-AIC)
\end{enumerate}

Akaiki, veri kümeleri için istatistiksel göreceli model ölçüsüdür. AIC
modeli verilerin mmodellerin her birini göreceli olarak tahmin ediyor.
Doğru modelin seçilmesini sağlar. Verilen bilgileri kullanara; model
verileri oluşturur, işlem temsil etmek için kullanılır ve bunlara
dayanarak göreceli bir tahmin sunmaktadır. Bu sayede ,modelin uyum
ölçüsü ve model karmaşıklığı anlaşılır.

\[ AIC = 2k - 2ln(L) \]

k: tahmin edilen parametre sayısı

L: maksimum likelihood

Model seçiminde en düşük AIC değeri seçilir.

\begin{enumerate}
\def\labelenumi{\arabic{enumi}.}
\setcounter{enumi}{1}
\tightlist
\item
  Schwarz Bayesian kriteri (BIC)
\end{enumerate}

Schwarz Bayesian bilgi kriteri ( BIC ) değeri, otoregresif sürecin
gecikme derecesinin belirlenmesinde kullanılır.Akaiki bilgi kriteri ile
yakından ilişkilidir. \[ BIC = ln(n)\cdot k - 2ln(L) \]

L = modelin olabilirlik fonksiyonunun maksimum değeri

x = gözlenen veriler

n = veri noktalarının sayısı

k = model tarafından tahmin edilen parametre sayısı

Model seçiminde en düşük BIC değeri seçilir.

\begin{enumerate}
\def\labelenumi{\arabic{enumi}.}
\setcounter{enumi}{2}
\tightlist
\item
  Hannan Quinn bilgi kriteri (HOİC )
\end{enumerate}

\[ HOİC= -2L_{max} + 2k\cdot ln(ln(n)) \]

Bilgi kriterleri iki bileşeni vardır:

\begin{itemize}
\tightlist
\item
  Kalanların karelerinin toplamı
\item
  Eklenen her parametrenin neden olduğu serbestik derecesi kaybı için
  modelin tabi tutulduğu ceza
\end{itemize}

bileşenleridir. Modele eklenen her yeni değişken, yani her yeni gecikme,
ceza teriminin değerini arttırır. Aynı zamanda kalanların karelerinin
değerini düşürmektedir. Akaiki bilgi kriteri (AIC) değeri, otoregresif
sürecin gecikme derecesinin belirlenmesinde kullanılır.Schwarz Bayesian
bilgi kriteri ( BIC ) değeri, otoregresif sürecin gecikme derecesinin
belirlenmesinde kullanılır.

BIC , AIC ile karşılaştırıldığında edildiğinde örnek büyüklüğü orta veya
geniş ise gecikme sayısı az olan AR modelin seçimine eğilim söz
konusudur.

    \section{Box-Jenkins Yöntemi
(ARIMA)}\label{box-jenkins-yuxf6ntemi-arima}

George E.P. BOX ve Gwilym M. JENKINS tarafından durağan tek değişkenli
zaman serilerinin analizi için geliştirilen (Yaffee and McGee 2000) ve
öngörü uygulamalarında kullanılan Box-Jenkins yöntemi, ARIMA modelleri
olarak da adlandırılmaktadır. Box-Jenkis yöntemi ele alınan zaman
serisinin özelliklerine göre belirlenen çeşitli modeller arasında uygun
olanını seçerek tahmin etme sürecini kapsamaktadır.Zaman serisi
modellerinde serinin kendi iç dinamiği önemli olup oluşturulan modeldeki
değişken, kendi gecikmeli değerleri ve hata teriminin gecikmeli
değerleriyle açıklanmaktadır. Box ve Jenkins uygulama aşamaları:

\begin{itemize}
\tightlist
\item
  Yöntemi Belirleme :
\end{itemize}

Durağan zaman serisine ARMA modellerinden herhangi birinin aday olarak
belirlenmesi aşamasıdır. Yani ARMA (p,q) modelinde p ve q nun
derecesinin bulunması aşamasıdır.Değerleri otoregresif ve hareketli
ortalamalar sürecinin özelliklerinden faydalanılarak bulunur. ARMA
modelleri otokorelasyon ve kısmi otokorelasyon fonksiyonları sayesinde
karakterize edilebilir.

\begin{itemize}
\tightlist
\item
  Parametre Tahmini :
\end{itemize}

ARMA modellerinde parametrelerini tahmin edilebilmesi için bir çok
yöntem bulunmaktadır.Yöntemler olabilirlik fonksiyonuna dayalı yöntemler
ya da eğrisel en küçük kareler yöntemleridir. ARMA modelinin
parametreleri olabilirlik fonksiyonunun maksimize edilebilir ya da hata
kareler fonksiyonunun minimize edilmesiyle bulunabilir.

\begin{itemize}
\tightlist
\item
  Teşhis Etme :
\end{itemize}

Modelin yeterliliğini kontrol edilmesidir.Zaman serisi modeli belirlenip
parametreleri tahmin edildikten sonra ayırt edici testler yardımıyla
belirlenen modelin ne kadar doğru olduğu ortaya konulabilir. Modelin
artık terimleri, test süreci için önemlidir. Belirlenen model yeterli
ise artıklar yaklaşık olarak bir beyaz gürültü sürecidir (white noise).
Beyaz gürültü sürecinde olması, ARIMA (p,d,q) modeli öngörü için yeterli
düzeydedir.

LjungBox testi, Modelin artık terimlerinin standart normal dağılımlı
rasgele değişkenlerin dizisinden oluşan bir beyaz gürültü sürecinde olup
olmadığının test etmek için kullanılmaktadır.

    \section{Durağanlık}\label{duraux11fanlux131k}

Serinin durağan olması, zaman serisinde ortalaması, varyansı ve
kovaryansı zaman içinde sabit kalmasıyla söylenir. Durağan zaman
serisinde ard arda gelen iki değer arasındaki fark zamanın kendisinden
kaynaklanmamakta, sadece zaman aralığından kaynaklanmaktadır. Bundan
dolayı serinin ortalaması zamanla değişmemektedir. \(Y_{t}\) serisi
tanımlayalım.

\[ E(Y_{t}) = µ \] \[ var(Y_{t}) = γ_{0} \]
\[ cov(Y_{t},Y_{t+k} ) = γ_{k} \]

Başlangıç noktasını t'den t + k'ye kaydırdıgımızı düşünelim. Y durağan
ise \(Y_{t}\) ve \(Y_{t+k}\) serilerinin ortalama, varyans ve
kovaryansları aynı olmalıdır. Eğer k = 0 ise
\(cov(Y_{t}, Y_{t+0}) = var(Y_{t}) = σ^2\) 'dir

Durağan olmayan seriler birim kök içerirler. Bir serideki birim kök
sayısı serinin durağan olana dek alınması gereken fark sayısına eşittir.
\(Y_{t}\) serisi d= 1 farkı alınınca durağan hale geliyorsa seri 1.
dereceden durağandır denir ve I(1) olarak gösterilir. Genel olarak seri
d kez farkı alınınca durağan oluyorsa seri d. dereceden durağandır denir
ve I(d) ile gösterilir.

\subsection{Trend Durağanlık ve Fark
Durağanlık}\label{trend-duraux11fanlux131k-ve-fark-duraux11fanlux131k}

Trend durağan olmayan süreçler için durağanlaştırmak için iki temel
yöntem vardır:

\begin{itemize}
\tightlist
\item
  Durağan olmayan zaman serisi için kurulacak regresyon denkleminde,
  seri trend (zaman) üzerine regrese edilir. Daha sonra bu regresyondan
  elde edilen kalıntılar üzerinde analizler yapılır.
\item
  Zaman serisine trend bir regresör olarak ilave edilerek gerekli
  analizler yapılır.
\end{itemize}

\[Y_{t} = \mu + B_{t} + e_{t}\]

(\(Y_{t}\): deterministik trend,\(e_{t} ~ IID(0,)\): durağan stokastik
bileşen)

Rassal Yürüyüş:

Rastsal yürüyüşü en basit şekilde şöyle gösterilir:

\[ Y_{t} = Y_{t−1} + u_{t} \] ( \(u_{t}\): beyaz gürültü)

Markov 1.derece özbağlanımsal tasarımla rastsal yürüyüş bağlantısı :

\[ Y_{t} = ρ\cdot Y_{t−1} + u_{t}  ,  −1 < ρ < 1 \]

Rastsal yürüyüşte ρ = 1 oldugu için, bu sürece ˘ ``birim kök'' (unit
root) süreci de denilmektedir.Rastsal yürüyüş sürecinde \(u_{t}\)
sarsıntıları kalıcıdır:

\[ Y_{1} = Y_{0} + u_{1} \]
\[ Y_{2} = Y_{1} + u_{2} = Y_{0} + u_{1} + u_{2} \]
\[ Y_{3} = Y_{2} + u_{3} = Y_{0} + u_{1} + u_{2} + u_{3} \] T
dönemindeki deger :

\[ Y_{t} = Y_{0} + \sum_t u_{t}  \]

Herhangi bir dönemdeki değerin daha önceki tüm rastsal sarsıntıların
toplamı olmasına, rastsal yürüyüşün ``sonsuz bellek'' (infinite memory)
özelligi de denir. \(E(u_{t}) = 0\) oldugundan, \(E(Y_{t}) = Y_{t}\)
olduguna dikkat edelir. Başka bir ifadeyle \(Y_{t}\) 'nin ortalaması
sabittir.Rastsal hatalar toplandıgı için, \(var(Y_{t})\) sürekli
artmakta ve böylece duraganlık varsayımı engellenmemiş olmaktadır.
\(Y_{t}\) 'nin varyansının \(var(Y_{t}) = tσ^2\) oldugu gösterilebilir.
Buna göre, t sonsuza giderken varyans da sonsuza gitmektedir
\pagebreak
\subsection{Durağanlık Analizi}\label{duraux11fanlux131k-analizi}

Durağanlık analizini grafiksel analiz, kolegram analiz ve birim kök
analizi kullanılarak yapılmaktadır.

\subsubsection{Birim Kök Analizi}\label{birim-kuxf6k-analizi}

Serinin durağanlığını ve durağanlık derecesini belirlemede
kullanılır.Birim Kök Analizi'nde 5 farklı yöntem bulunmaktadır.

\begin{itemize}
\tightlist
\item
  Dickey Fuller Testi (DF)
\item
  Genişletilmiş Dickey-Fuller Testi (ADF)
\item
  Phillips- Perron Testi (PP)
\item
  Kwiatkowski-Phillips-Schmidt-Shin Testi (Kpss)
\item
  Ng Perron Testi
\end{itemize}

En çok kullanılan DF,ADF ve PP testleridir.

\subsubsection{Dickey Fuller Testi (DF)}\label{dickey-fuller-testi-df}

Serinin durağan olmadığı (birim köke sahip olduğu) boş hipotezinin ,
durağan olduğu (birim kök olmadığı ) alternatif hipotezine göre
sınanmasıdır. \(Y_{t}\) değişkeninin bu dönemde aldığı değerin geçen
dönemdeki değeri olan ile ilişkisi, \[ Y_{t}=p\cdot Y_{t-1} + u_{t}\]
şeklinde ifade edilir. (\(u_{t}\): kalıntı terim )

Yukarıdaki model 1. dereceden otoregresif AR(1) modelidir.ρ katsayısı 1
eşit olursa birim kök sorunu (durağan olmama durumu) ortaya çıkmaktadır
ve model \(p=1\) ise \[ Y_{t}=Y_{t-1} + u_{t} \] şeklini almaktadır.
Sonuç bir önceki dönemde iktisadi değişkenin değerinin ve dolayısıyla o
dönemde maruz kaldığı şokun olduğu gibi sistemde kalması anlamına gelir.
Bu şokların kalıcı nitelikte olması serinin durağan olmaması ve zaman
içinde gösterdiği trendin stokastik olması anlamına gelir.

\(p<1\) ise geçmiş dönemlerdeki şoklar belli bir süreetkilerini
sürdürseler de, bu etki giderek azalacak ve kısa bir dönem sonra tamamen
ortadan kalkacaktır.

\subsubsection{Genişletilmiş Dickey-Fuller Testi
(ADF)}\label{geniux15fletilmiux15f-dickey-fuller-testi-adf}

Dickey-Fuller testi, hata terimlerinin otokorelasyon içermesi halinde
kullanılamamaktadır. Zaman serisinin gecikmeli değerleri kullanılarak
hata terimindeki otokorelasyon ortadan kaldırılabilmektedir.
Dickey-Fuller bağımlı değişkenin gecikmeli değerlerini, bağımsız
değişken olarak modele dahil eden yeni bir test geliştirmiştir. Bu test
Genişletilmiş Dickey-Fuller testidir. Burada gecikmeli değişkene ait
uygun gecikme mertebesi belirlenirken Akaike ve Schwarz kriterlerinden
yararlanılmaktadır.

ADF denklemi :
\[\delta Y_{t}= a+ b_{t}+\gamma Y_{t-1}+ u_{t-1} +c\sum_{i=1}^n  \delta Y_{t} \]\\
\(\delta\) : delta

\subsubsection{Phillips- Perron Testi
(PP)}\label{phillips--perron-testi-pp}

Phillips-Perron birim kök testi ise hata teriminin zayıf derecede
bağımlı olmasına ve heterojen olarak dağılmasına izin vermektedir. Bu
sayede otokorelasyon sorunu ortaya çıkmamaktadır.

\subsubsection{Kwiatkowski-Phillips-Schmidt-Shin Testi
(Kpss)}\label{kwiatkowski-phillips-schmidt-shin-testi-kpss}

KPSS testinde amaç gözlenen serideki deterministik trendi arındırarak
serinin durağan olmasını sağlamaktır. Bu testte kurulan birim kök
hipotezi ADF testinde kurulan hipotezlerden farklıdır. Sıfır hipotezi
serinin durağan olduğunu ve birim kök içermediğini, buna karşın
alternatif hipotez ise seride birim kök olduğunu ve durağan olmadığını
ifade eder. Boş hipotezdeki durağanlık trend durağanlıktır. Çünkü
seriler trendden arındırılmışlardır.

Trendden arındırılan seride birim kök olmaması, serinin trend
durağanlığını gösterir. KPSS testinin en önemli özelliği bir veya daha
büyük bir MA yapısı içeren serilerde ADF' nin aksine gücünün
azalmamasıdır.

KPSS testi LM testi ile benzer biçimde belirlenmektedir. Dolayısıyla LM
istatistiğinin oluşumu önemlidir. LM testinde boş hipotez, rassal
yürüyüşün sıfır varyansa sahip olduğunu ve serinin deterministik trend,
rassal yürüyüş ve durağan kalıntılar toplamından oluştuğunu ima eder.

\subsubsection{Ng Perron Testi}\label{ng-perron-testi}

Ng-Perron birim kök testi, Phillips-Perron birim kök testinde ortaya
çıkan hata teriminin hacmindeki çarpıklığın düzeltilmesi için
geliştirilmiştir.PP testinde serilerde negatif hareketli ortalama yapısı
olduğunda, büyük oranda hata teriminde nörneklem çarpıklığı olmaktadır.
DF testlerinde bu durum çok büyük bir sorun yaratmamaktadır.

NG - Perron testi bu nedenle PP testlerini ve bilgi kriterlerini
modifiye etmektedir.MSB ve MPT lerde sıfır hipotezi birim kökün
olmadığını , alternatif hipotez ise birim kökün varlığını ifade eder.

adf.test ve pp.test için: - H0: birim kök içerir. - HA: birim kök
içermez, durağandır.

kpss.test için: - H0: level veya trend durağandır. - HA: durağan
değildir.

Test sonuçlarına göre adf ve pp testleri için p-value (test
istatistiğinin olasılık değeri) \%5'den küçük ise (\%95 güven
seviyesinde) testin boş hipotezi ret edilir. Kpss test için ise durumun
tam tersi doğrudur.

    \section{Normallik Testi- Shapiro Wilk
Test}\label{normallik-testi--shapiro-wilk-test}

Veri seti üzerinde yapılan bir çok veri analizi verinin normal dağıldığı
varsayımına dayanır. Veri analizine başlamadan önce verinin normal
dağılıma uygun olup olmadığına bakılmalıdır. Normallik kontrolü için
yaygın olarak kullanılan testlerden birisi de Shapiro-Wilks testidir.

Shapiro-Wilk (1965) testi, verinin sıra istatistiklerinin uygun bir
lineer bileşeninin karesinin, kareler toplamına bölümü uygulanarak elde
edilir.

\[  W = \frac{(\sum_{i=1}^n a_{i} x_{i})^2}{\sum_{i=1}^n (x_{i} - x^-_{i})^2 }  \text{ ve } a_{i}=(a_{1},a_{2},...,a_{n})   \]

\(a_{i}=(a_{1},a_{2},...,a_{n})\): standart normal dağılımda N(0,1)' da
n adet sıra istatistiğinin beklenen değerlerinin vektörünü ifade eder.

Sadece numeric (numeric factor) veriyi, argüman olarak
shapiro.test(veri) fonksiyonuna eklenir.

    \section{Otokorelasyon}\label{otokorelasyon}

Otokorelasyon durumunda parametrelerin en küçük kareler tahmincileri
sapmasız ve tutarlıdır,ancak etkin değildir. Hata teriminin varyansının
tahmincisi sapmalıdır ve bu yüzden parametrelerin varyansları da sapmalı
olur. Pozitif otokorelasyon varsa sapma negatif olur. Varyanslar
olduğundan küçük bulunur. Bunun sonucunda t test istatistiği değeri
büyük çıkar. Böylece anlamsız bir katsayının anlamlı olma olasılığı
artar. \(R_{2}\) de yükselir. Dolayısıyla F değeri olduğundan büyük
bulunur. Sonuç olarak t ve F testleri güvenilirliğini yitirip yanıltıcı
sonuç verirler. Otokorelasyonu durumunu göstermek için 2 yöntem
kullanılır.

\begin{itemize}
\tightlist
\item
  Grafik Yorumu
\item
  Otokorelasyon Testleri ( Durbin-Watson d , Wallis , King ,Breush ve
  Godfrey)
\end{itemize}

NOT: Zaman serilerinin durağan olması demek, zaman içinde varyansın ve
ortalamanın sabit olması ve gecikmeli iki zaman periyodundaki
değişkenlerin kovaryansının değişkenler arasındaki gecikmeye bağlı olup,
zamana bağlı olmamasıdır.

\subsection{Grafik Yorumu}\label{grafik-yorumu}

\subsubsection{Otokorelasyon Katsayısı
(ACF)}\label{otokorelasyon-katsayux131sux131-acf}

Otokorelasyon katsayısı, zaman serisiyle bu serinin gecikmeli serileri
arasındaki ilişkiyi gösterir.Serideki bütün gecikmeler otokorelasyon
fonksiyonunu oluşturmaktadır. Otokorelasyona; ihmal edilen değişkenler,
modelin fonksiyonel biçiminin doğru belirlenmemesi, verilerle ilgili
ölçme hataları, tesadüfi olarak ortaya çıkan savaş, kuraklık gibi
olayların, etkilerini ortaya çıktıkları dönemlerden sonra da sürdürmesi
neden olabilirler.

\[ E(u_{i},u_{j})=0 i≠j \]

şeklinde ifade edilir.
\[  ACF(k) = \frac{\sum_{t=1+k}^T (Y_{t}- Y^-)(Y_{t-k}- Y^-)}{\sum_{t=1}^T (Y_{t}- Y^-)^2 }  \]

Eğer bir seri tesadüfi ise, her hangi bir gecikmede yani \(Y_{t}\) ve
\(Y_{t-k}\) arasındaki otokorelasyonlar sıfıra yakın olmaktadır. Bu
durumda zaman serisinin ardışık değerlerinin birbirleriyle ilişkisi
yoktur.

Eğer bir zaman serisi trende sahipse ise, \(Y_{t}\) ve \(Y_{t-1}\)
arasında yüksek korelasyon bulunacaktır. Birkaç gecikmeden sonra
otokorelasyon katsayıları hızlıca sıfıra yaklaşacaktır. İlk gecikmede
otokorelasyon kat sayısı 1'e yakındır. İkinci gecikmede de oldukça
yüksektir. Ve daha sonra hızlıca azalır.

Hata terimleri arasında bir ilişki varsa yani otokorelasyon durumu söz
konusu ise \[ E(u_{i},u_{j})≠0, i≠j \]

şeklinde ifade edilir.

\subsubsection{Kısmi Otokorelasyon Fonksiyonu (PACF
)}\label{kux131smi-otokorelasyon-fonksiyonu-pacf}

Kısmi korelesyon; ilk başta bilinemeyen zaman serisinin değerleri olan p
ve q değerlerinin kaçıncı dereceden olduğunu bulmak için uygulanır.Kısmi
korelasyon, gecikmeli değişkenler arasındaki ilişkiyi ifade eder.Yani
kısmi korelasyon diğer bütün gecikmeli gözlemlerin etkisinden
arındırıldıktan sonra \(x_{t}\) değişkeni ile bu değişkenden herhangi
bir k-gecikmesiyle elde edilen \(x_{t+k}\) değişkeni arasındaki ilişkiyi
inceler. Kısmi korelasyon PACF ile gösterilir.

\subsection{Otokorelasyon Testleri}\label{otokorelasyon-testleri}

\subsubsection{Durbin- Watson d Testi}\label{durbin--watson-d-testi}

Sabit parametresi olan modellerde 1.dereceden otokorelasyonun varligini
test etmek icin kullanilir.

Testin 4 aşamaması vardır. Sırasıyla incelersek ; - Hipotezin kurulması

\(H_{0}\): p=0 ise otokorelasyon yoktur. \(H_{1}\): p≠0 ise
otokorelasyon vardır.

\begin{itemize}
\tightlist
\item
  Tablo değerlerinin incelenmesi
\end{itemize}

Durbin Watson tablo değerlerinden d istatisliğinin ;seçilen bir
anlamlılık düzeyi ile gözlem sayısı ve açıklayıcı değişken sayısına göre
alt (\(d_{l}\)) ve üst (\(d_{u}\)) sınır değerleri bulunur.

\begin{itemize}
\tightlist
\item
  Durbin Watson d istatistiği hesaplanması
\end{itemize}

\[  d = \frac{\sum_{t=1}^n (e_{t} -e_{t-1})^2}{\sum_{e=1}^n (e_{t})^2 } \]

\begin{itemize}
\tightlist
\item
  Karşılaştırma yapılması ve karar verilmesi
\end{itemize}

Eğer  0\textless{}d \textless{} \(d_{L}\)  ise pozitif otokorelasyon
vardır.

Eğer \(d_{L}\) ≤ d ≤ \(d_{u}\)  ise karar verilmemektedir.

Eğer \(d_{u}\) \textless{} d \textless{}4-\(d_{u}\) ise otokorelasyon
yoktur.

Eğer  4- \(d_{u}\) ≤ d ≤4- \(d_{L}\)  ise karar verilmemektedir.

Eğer  4- \(d_{L}\)  \textless{}d \textless{}4 ise negatif
otokorelasyon vardır.

Durbin-Watson testini : - Model sabit terimsiz ise, - Bağımsız X
değişkenleri stokastikse, - Otokorelasyonun derecesi 1'den büyük ise, -
Zaman serisinde ara yıllar noksan ise, - Modelde bağımsız değişken
olarak gecikmeli bağımlı değişken varsa uygulanamaz.

Durbin-Watson d istatistiği tablosu \(n<15\) için \(d_{L}\) ve \(d_{u}\)
değerlerini vermemektedir. Bu durumda, Von-Neumann testi
kullanılmaktadır.

\subsubsection{Wellis Testi:}\label{wellis-testi}

3'er aylik verilerde 4.dereceden otokorelasyon varligini test etmek icin
kullanilir. Durbin Watson testinin, 4.dereceden otokorelasyon varligini
test etmek icin duzenlenmis halidir.

\subsubsection{Breush ve Godfrey Testi:}\label{breush-ve-godfrey-testi}

Yuksek dereceden otokorelasyonun varliginin test edilmesi icin
kullanilir.Orjinal denklemde bagimli degiskenin gecikmeli halinin
bagimsiz degiskenler arasinda yer almasi durumunda da kullanilir.

Bagimli degisken: Orjinal model kalintilari

Bagimsiz degiskenler: Orjinal modelin aciklayici degiskenleri ve
incelenen otokorelasyonun derecesi sayisinda gecikme katsayisi modelde
yer alir.

Örneğin 2.dereceden otokorelasyonun varligi arastiriliyorsa, orjinal
modelin bagimsiz degiskenleri ve \[ u\cdot (t-1), u\cdot (t-2) \]
katsayilari yeni modelde bagimsiz degiskenler arasinda yer alir. Test
istatistigi: \[ n \cdot R{_2} \] ile yeni modelin karesi hesaplanir, n
ile carpilarak test istatistigi elde edilir.
\pagebreak
    \section{Giriş}\label{giriux15f}

Bitirme çalışmasında toplamsal ve çarpımsal ayrışma modellerini analiz
etmek ve modeller için öngörüde bulunabilmek için iki farklı veri
inceleyeceğiz. İlk verimiz ABD de kullanılan doğal gaz tüketimi hakkında
olup çarpımsal ayrışma modele örnektir.İkinci verimiz Florida'daki ham
petrolün üretimi hakkında olup toplamsal ayrışma modeline örnektir.
Verilerin hangi modele ait olduğu ve öngörü analizleri aşağıda detaylı
olarak inceleyeceğiz.

\section{Natural Gas Futures}\label{natural-gas-futures}

Verinin Yayıncısı: U.S. Energy Information Administration Data

ABD ulusal ve devlet verilerinin;elektrik, kömür, doğal gaz ve petrol
gibi tüm ana enerji ürünleri üzerindeki üretim, tüketim ve diğer
göstergelere ilişkisini göstermektedir.

Verinin Adı: Natural Gas Futures Contract 3

ABD ulusal ve devlet verilerinin;doğal gaz üretim, tüketim ve diğer
göstergelere ilişkisini göstermektedir.Veri zaman serisi olarak
ilerlemektedir.

Verinin Sıklığı: Aylık

Verinin Zaman Aralığı: 31.01.1994- 31.03.2018

Verinin Alındığı Yer: Quandl

Quandl yatırım profesyonellerine hizmet veren finansal, ekonomik ve
alternatif veriler için bir platformdur.Verileri 500'den fazla
yayıncıdan sağlar. Quandl'ın verilerine bir API aracılığıyla
erişilebilir. Bir çok programlama dili için API erişimi (R, Python,
Matlab, Maple ve Stata) vardır.

Excel eklentisi, stok fiyat bilgisi dahil olmak üzere verilere erişim
sağlar.

    \begin{Verbatim}[commandchars=\\\{\}]
{\color{incolor}In [{\color{incolor}2}]:} data\PY{o}{\PYZlt{}\PYZhy{}}Quandl\PY{p}{(}\PY{l+s}{\PYZdq{}}\PY{l+s}{EIA/NG\PYZus{}RNGC3\PYZus{}M\PYZdq{}}\PY{p}{,} api\PYZus{}key\PY{o}{=}\PY{l+s}{\PYZdq{}}\PY{l+s}{BtbwkANCR4aHKBscujRb\PYZdq{}}\PY{p}{,} collapse\PY{o}{=}\PY{l+s}{\PYZdq{}}\PY{l+s}{monthly\PYZdq{}}\PY{p}{)}
\end{Verbatim}


    Veriyi quandl veri sitesinden alıyoruz. Veri iki sütundan oluşmaktadır.
İlk sütun tarih ikinci sütun ise kullanım miktarıdır.

    \begin{Verbatim}[commandchars=\\\{\}]
{\color{incolor}In [{\color{incolor}3}]:} hist\PY{p}{(}data\PY{p}{[}\PY{p}{,}\PY{l+m}{2}\PY{p}{]}\PY{p}{,}  main\PY{o}{=}\PY{l+s}{\PYZdq{}}\PY{l+s}{Natural Gas Futures Contract \PYZdq{}}\PY{p}{)}
\end{Verbatim}


    \begin{center}
    \adjustimage{max size={0.9\linewidth}{0.9\paperheight}}{output_10_0.png}
    \end{center}
    { \hspace*{\fill} \\}
    
    Aşağıda verimizi çizgizel grafik gösterimini görüyoruz.

    \begin{Verbatim}[commandchars=\\\{\}]
{\color{incolor}In [{\color{incolor}4}]:} plot\PY{p}{(}data\PY{p}{[}\PY{p}{,}\PY{l+m}{2}\PY{p}{]}\PY{p}{,}type\PY{o}{=}\PY{l+s}{\PYZdq{}}\PY{l+s}{l\PYZdq{}}\PY{p}{,} main\PY{o}{=}\PY{l+s}{\PYZdq{}}\PY{l+s}{Natural Gas Futures Contract \PYZdq{}}\PY{p}{)}
\end{Verbatim}


    \begin{center}
    \adjustimage{max size={0.9\linewidth}{0.9\paperheight}}{output_12_0.png}
    \end{center}
    { \hspace*{\fill} \\}
    
    \subsection{Decompose - Geleneksel Zaman Serisi Ayrışım
Yöntemleri}\label{decompose---geleneksel-zaman-serisi-ayrux131ux15fux131m-yuxf6ntemleri}

    Veriyi sezonsal veriden ayırmak için R paketinin decompose fonksiyonunu
kullanıyoruz. Decompose fonksiyonu ile veri sezonsal etkiden ve trend
etkisinden arınmaktadır.Veri aylık olduğundan dolayı sıklığı 12 olarak
alıyoruz.

Ek olarak verinin ayrışma yönteminin, toplamsal ya da çarpımsal olduğunu
anlamak için type= c("additive") ve type= c("multiplicative") fonksiyonu
kullanıyoruz.

Dönemsel mevsimselliği incelemek için ise R 'ın figure fonksiyonu
kullanılmaktadır.

    Aşağıda zaman serisi (ts) verimize toplamsal ayrıştırma tipine decompose
fonsiyonu uyguluyoruz. Böylelikle verimizdeki sezonsal ve trendsel
etkiden çıkartmış oluyoruz. Plot fonksiyonu ile yaptığımız modeli
grafiksel olarak gösteriyoruz. Figure fonksiyonu ile de zaman seriminizn
12 aylık sezonsal etkisini gösteriyoruz.

    \begin{Verbatim}[commandchars=\\\{\}]
{\color{incolor}In [{\color{incolor}5}]:} additive\PYZus{}model\PY{o}{\PYZlt{}\PYZhy{}}decompose\PY{p}{(}ts\PY{p}{(}data\PY{p}{[}\PY{p}{,}\PY{l+m}{2}\PY{p}{]}\PY{p}{,}start \PY{o}{=} \PY{k+kt}{c}\PY{p}{(}\PY{l+m}{1994}\PY{p}{,} \PY{l+m}{1}\PY{p}{)}\PY{p}{,} end \PY{o}{=} \PY{k+kt}{c}\PY{p}{(}\PY{l+m}{2018}\PY{p}{,} \PY{l+m}{3}\PY{p}{)}\PY{p}{,} freq\PY{o}{=}\PY{l+m}{12}\PY{p}{)}\PY{p}{,}type\PY{o}{=} \PY{k+kt}{c}\PY{p}{(}\PY{l+s}{\PYZdq{}}\PY{l+s}{additive\PYZdq{}}\PY{p}{)}\PY{p}{)}
        plot\PY{p}{(}additive\PYZus{}model\PY{p}{)}
        plot\PY{p}{(}additive\PYZus{}model\PY{o}{\PYZdl{}}figure\PY{p}{,}type\PY{o}{=}\PY{l+s}{\PYZdq{}}\PY{l+s}{b\PYZdq{}}\PY{p}{)}
\end{Verbatim}


    \begin{center}
    \adjustimage{max size={0.9\linewidth}{0.9\paperheight}}{output_16_0.png}
    \end{center}
    { \hspace*{\fill} \\}
    
    \begin{center}
    \adjustimage{max size={0.9\linewidth}{0.9\paperheight}}{output_16_1.png}
    \end{center}
    { \hspace*{\fill} \\}
    
    Toplamsal ayrışma modelini incelediğimizde trend bileşeninin 1994
yıllarında düşük seviyelerde olduğunu görüyoruz.Ancak 2000 yılından
itibaren trend değerleri artmıştır. Trendsel etkiyi incelediğimizde
2003-2008 arası trend bileşeninin değerleri en yüksel noktalara
ulaşmıştır.2013'ten itibaren ise trendsel etki giderek azaldığını
görüyoruz. Sezonsal etkiyi incelersek veri bütün yıllarda aynı sezonsal
değerleri aldığını görüyoruz.

Figure fonksiyonu ile gelen mevsimsellik etki inclendiğinde veri şubat
ayında en düşük seviyede, haziran ayında ise en yüksek mevsimsellikten
etkilendiğini görebiliyoruz.

\pagebreak
    Aşağıda zaman serisi (ts) verimize çarpımsal ayrıştırma tipine decompose
fonsiyonu uyguluyoruz. Böylelikle verimizdeki sezonsal ve trendsel
etkiden çıkartmış oluyoruz. Plot fonksiyonu ile yaptığımız modeli
grafiksel olarak gösteriyoruz. Figure fonksiyonu ile de zaman seriminizn
12 aylık sezonsal etkisini gösteriyoruz.

    \begin{Verbatim}[commandchars=\\\{\}]
{\color{incolor}In [{\color{incolor}6}]:} multiplicative\PYZus{}model\PY{o}{\PYZlt{}\PYZhy{}}decompose\PY{p}{(}ts\PY{p}{(}data\PY{p}{[}\PY{p}{,}\PY{l+m}{2}\PY{p}{]}\PY{p}{,} freq\PY{o}{=}\PY{l+m}{12}\PY{p}{)}\PY{p}{,} type\PY{o}{=} \PY{k+kt}{c}\PY{p}{(}\PY{l+s}{\PYZdq{}}\PY{l+s}{multiplicative\PYZdq{}}\PY{p}{)}\PY{p}{)}
        plot\PY{p}{(}multiplicative\PYZus{}model\PY{o}{\PYZdl{}}figure\PY{p}{,} type\PY{o}{=}\PY{l+s}{\PYZdq{}}\PY{l+s}{b\PYZdq{}}\PY{p}{)}
        plot\PY{p}{(}multiplicative\PYZus{}model\PY{p}{)}
\end{Verbatim}


    \begin{center}
    \adjustimage{max size={0.9\linewidth}{0.9\paperheight}}{output_19_0.png}
    \end{center}
    { \hspace*{\fill} \\}
    
    \begin{center}
    \adjustimage{max size={0.9\linewidth}{0.9\paperheight}}{output_19_1.png}
    \end{center}
    { \hspace*{\fill} \\}
    
    Çarpımsal ayrışma modelini incelediğimizde trend bileşeninin 1994
yıllarında düşük seviyelerde olduğunu görüyoruz.Ancak 2000 yılından
itibaren trend değerleri artmıştır. Trendsel etkiyi incelediğimizde
2003-2008 arası trend bileşeninin değerleri en yüksel noktalara
ulaşmıştır.2013'ten itibaren ise trendsel etki giderek azaldığını
görüyoruz. Sezonsal etkiyi incelersek veri bütün yıllarda aynı sezonsal
değerleri aldığını görüyoruz. Random değerlerinin sürekli değişiklik
gösterdiğini görüyoruz.

Figure fonksiyonu ile gelen mevsimsellik etki inclendiğinde veri şubat
ayında en düşük seviyede, haziran ayında ise en yüksek mevsimsellikten
etkilendiğini görebiliyoruz.

    Toplumsal ve çarpımsal ayrışma modellerindeen büyük farklılık random
değerlerinde gözlenmektedir. Toplamsal modelde random değerler -2 ile 4
arasında seyrederken , çarpımsal modelde ise 0.8 ile 1.4 arasında
seyretmektedir.

    Toplamsal ve çarpımsal ayrışma model incelendiğinde büyük bir farklılık
görünmediğinden bu aşamada karar verilememektedir. Bu yüzden ilerleyen
aşamalarda veriye arima modeli uygulayacağız.

Summary fonksiyonu ile verinin özetini göstermiş olunur.

    Aşağıda asıl verimize, toplamsal ayrıştırma sezonsal ve trendsel etkiden
ayrılmış verimize, çarpımsal ayrıştırma ile sezonsal ve trendsel etkiden
ayrılmış verimizin özetlerini görüyoruz.

    \begin{Verbatim}[commandchars=\\\{\}]
{\color{incolor}In [{\color{incolor}7}]:} \PY{k+kp}{summary}\PY{p}{(}data\PY{p}{[}\PY{p}{,}\PY{l+m}{2}\PY{p}{]}\PY{p}{)}
        \PY{k+kp}{summary}\PY{p}{(}na.omit\PY{p}{(}additive\PYZus{}model\PY{o}{\PYZdl{}}random\PY{p}{)}\PY{p}{)}
        \PY{k+kp}{summary}\PY{p}{(}na.omit\PY{p}{(}multiplicative\PYZus{}model\PY{o}{\PYZdl{}}random\PY{p}{)}\PY{p}{)}
\end{Verbatim}


    
    \begin{verbatim}
   Min. 1st Qu.  Median    Mean 3rd Qu.    Max. 
  1.490   2.572   3.669   4.308   5.352  14.178 
    \end{verbatim}

    
    
    \begin{verbatim}
     Min.   1st Qu.    Median      Mean   3rd Qu.      Max. 
-1.925256 -0.323617 -0.062930 -0.001936  0.196409  4.021991 
    \end{verbatim}

    
    
    \begin{verbatim}
   Min. 1st Qu.  Median    Mean 3rd Qu.    Max. 
 0.7722  0.9172  0.9859  0.9938  1.0545  1.3901 
    \end{verbatim}

    
    \subsection{Normallik Testi- Shapiro Wilk
Test}\label{normallik-testi--shapiro-wilk-test}

Verinin normalliğini kontrol etmek için shapiro.test fonksiyonu
kullanılmaktadır.P değeri testen ne kadar emin olduğunu göstermektedir.P
değerinde güven aralığını genelde \%95 olarak alınır, yani p değeri
0.05'tan küçük ise yaptığımız testen emin olabiliriz. W değeri 1
değerine ne kadar yakınsa veri normalliğe o derece yakındır.

    Shapiro testini toplumsal ve çarpımsal ayrışma modeline uygulanır.

    \begin{Verbatim}[commandchars=\\\{\}]
{\color{incolor}In [{\color{incolor}8}]:} shapiro.test\PY{p}{(}na.omit\PY{p}{(}additive\PYZus{}model\PY{o}{\PYZdl{}}random\PY{p}{)}\PY{p}{)}
        shapiro.test\PY{p}{(}na.omit\PY{p}{(}multiplicative\PYZus{}model\PY{o}{\PYZdl{}}random\PY{p}{)}\PY{p}{)}
\end{Verbatim}


    
    \begin{verbatim}

	Shapiro-Wilk normality test

data:  na.omit(additive_model$random)
W = 0.79155, p-value < 2.2e-16

    \end{verbatim}

    
    
    \begin{verbatim}

	Shapiro-Wilk normality test

data:  na.omit(multiplicative_model$random)
W = 0.96499, p-value = 2.579e-06

    \end{verbatim}

    
    Toplamsal ve çarpımsal ayrışma modeline uygulanan normallik testi
incellendiğinde p değerlerinin oldukça küçük, neredeyse 0 yakın olduğunu
görüyoruz. Yaptığımız iki testte de oldukça eminiz. W değeri
incelendiğin de ise çarpımsal modelin 1 daha yakın olduğu görülmektedir.
Verimiz normaldir, diyebiliriz.

    \subsection{Otokorelasyon}\label{otokorelasyon}

Zaman serisiyle bu serinin gecikmeli serileri arasındaki ilişkiyi
göstermek için R'ın acf ve pacf fonksiyonunu toplamsal ve çarpımsal
ayrışma modeline uygulanır.

Otokorelasyon fonksiyonu ile durağanlıkta ölçülebilir.Eğer
ACF(otokorelasyon fonksiyonu) çok yüksek bir değerden başlayıp çok yavaş
küçülüyorsa, bu seri için durağan değildir diyebiliriz.

    \begin{Verbatim}[commandchars=\\\{\}]
{\color{incolor}In [{\color{incolor}9}]:} acf\PY{p}{(}na.omit\PY{p}{(}additive\PYZus{}model\PY{o}{\PYZdl{}}random\PY{p}{)}\PY{p}{)}
        acf\PY{p}{(}na.omit\PY{p}{(}multiplicative\PYZus{}model\PY{o}{\PYZdl{}}random\PY{p}{)}\PY{p}{)}
\end{Verbatim}


    \begin{center}
    \adjustimage{max size={0.9\linewidth}{0.9\paperheight}}{output_30_0.png}
    \end{center}
    { \hspace*{\fill} \\}
    
    \begin{center}
    \adjustimage{max size={0.9\linewidth}{0.9\paperheight}}{output_30_1.png}
    \end{center}
    { \hspace*{\fill} \\}
    
    Toplamsal ve çarpımsal ayrıştırma modelleri için acf değerlerini
incelediğimizde iki durumunda durağan halde olduğunu söyleyebiliriz.
\pagebreak
    \begin{Verbatim}[commandchars=\\\{\}]
{\color{incolor}In [{\color{incolor}10}]:} pacf\PY{p}{(}na.omit\PY{p}{(}additive\PYZus{}model\PY{o}{\PYZdl{}}random\PY{p}{)}\PY{p}{)}
         pacf\PY{p}{(}na.omit\PY{p}{(}multiplicative\PYZus{}model\PY{o}{\PYZdl{}}random\PY{p}{)}\PY{p}{)}
\end{Verbatim}


    \begin{center}
    \adjustimage{max size={0.9\linewidth}{0.9\paperheight}}{output_32_0.png}
    \end{center}
    { \hspace*{\fill} \\}
    
    \begin{center}
    \adjustimage{max size={0.9\linewidth}{0.9\paperheight}}{output_32_1.png}
    \end{center}
    { \hspace*{\fill} \\}
\pagebreak    
    \begin{Verbatim}[commandchars=\\\{\}]
{\color{incolor}In [{\color{incolor}11}]:} hist\PY{p}{(}na.omit\PY{p}{(}multiplicative\PYZus{}model\PY{o}{\PYZdl{}}random\PY{p}{)}\PY{p}{)}
\end{Verbatim}


    \begin{center}
    \adjustimage{max size={0.9\linewidth}{0.9\paperheight}}{output_33_0.png}
    \end{center}
    { \hspace*{\fill} \\}
\pagebreak    
    \subsection{Durağanlık Analizi}\label{duraux11fanlux131k}

Veride durağanlık durumunu belirlemek için sırasıyla ;Augmented
Dickey-Fuller Test- Kwiatkowski-Phillips-Schmidt-Shin (KPSS) Test -
Phillips--Perron Birim Kök Testi uygulanır.

    \subparagraph{Augmented Dickey Fuller
Test}\label{augmented-dickey-fuller-test}

İlk olarak Augmented Dickey Fuller testini verinin bileşenlerden
ayrılmamış haline, toplamsal ve çarpımsal ayrışma modellerine sırasıyla
uyguluyarak verinin durağanlığını kontrol ediyoruz. ADF testini R içinde
bulunan adf.test fonksiyonu ile elde edebiliriz.

    \begin{Verbatim}[commandchars=\\\{\}]
{\color{incolor}In [{\color{incolor}12}]:} adf.test\PY{p}{(}data\PY{p}{[}\PY{p}{,}\PY{l+m}{2}\PY{p}{]}\PY{p}{)}
         adf.test\PY{p}{(}na.omit\PY{p}{(}additive\PYZus{}model\PY{o}{\PYZdl{}}random\PY{p}{)}\PY{p}{)}
         adf.test\PY{p}{(}na.omit\PY{p}{(}multiplicative\PYZus{}model\PY{o}{\PYZdl{}}random\PY{p}{)}\PY{p}{)}
\end{Verbatim}


    
    \begin{verbatim}

	Augmented Dickey-Fuller Test

data:  data[, 2]
Dickey-Fuller = -2.3992, Lag order = 6, p-value = 0.408
alternative hypothesis: stationary

    \end{verbatim}

    
    \begin{Verbatim}[commandchars=\\\{\}]
Warning message in adf.test(na.omit(additive\_model\$random)):
"p-value smaller than printed p-value"
    \end{Verbatim}

    
    \begin{verbatim}

	Augmented Dickey-Fuller Test

data:  na.omit(additive_model$random)
Dickey-Fuller = -8.8024, Lag order = 6, p-value = 0.01
alternative hypothesis: stationary

    \end{verbatim}

    
    \begin{Verbatim}[commandchars=\\\{\}]
Warning message in adf.test(na.omit(multiplicative\_model\$random)):
"p-value smaller than printed p-value"
    \end{Verbatim}

    
    \begin{verbatim}

	Augmented Dickey-Fuller Test

data:  na.omit(multiplicative_model$random)
Dickey-Fuller = -8.2717, Lag order = 6, p-value = 0.01
alternative hypothesis: stationary

    \end{verbatim}

\pagebreak    
    \subparagraph{Kwiatkowski-Phillips-Schmidt-Shin (KPSS)
Test}\label{kwiatkowski-phillips-schmidt-shin-kpss-test}

İlk olarak Kwiatkowski-Phillips-Schmidt-Shin (KPSS) Test testini verinin
bileşenlerden ayrılmamış haline, toplamsal ve çarpımsal ayrışma
modellerine sırasıyla uyguluyarak verinin durağanlığını kontrol
ediyoruz. KPSS testini R içinde bulunan kpss.test fonksiyonu ile elde
edebiliriz.

    \begin{Verbatim}[commandchars=\\\{\}]
{\color{incolor}In [{\color{incolor}13}]:} kpss.test\PY{p}{(}data\PY{p}{[}\PY{p}{,}\PY{l+m}{2}\PY{p}{]}\PY{p}{,} null\PY{o}{=}\PY{l+s}{\PYZdq{}}\PY{l+s}{Trend\PYZdq{}}\PY{p}{)}
         kpss.test\PY{p}{(}na.omit\PY{p}{(}additive\PYZus{}model\PY{o}{\PYZdl{}}random\PY{p}{)}\PY{p}{,} null\PY{o}{=}\PY{l+s}{\PYZdq{}}\PY{l+s}{Trend\PYZdq{}}\PY{p}{)}
         kpss.test\PY{p}{(}na.omit\PY{p}{(}multiplicative\PYZus{}model\PY{o}{\PYZdl{}}random\PY{p}{)}\PY{p}{,} null\PY{o}{=}\PY{l+s}{\PYZdq{}}\PY{l+s}{Trend\PYZdq{}}\PY{p}{)}
\end{Verbatim}


    \begin{Verbatim}[commandchars=\\\{\}]
Warning message in kpss.test(data[, 2], null = "Trend"):
"p-value smaller than printed p-value"
    \end{Verbatim}

    
    \begin{verbatim}

	KPSS Test for Trend Stationarity

data:  data[, 2]
KPSS Trend = 1.2645, Truncation lag parameter = 3, p-value = 0.01

    \end{verbatim}

    
    \begin{Verbatim}[commandchars=\\\{\}]
Warning message in kpss.test(na.omit(additive\_model\$random), null = "Trend"):
"p-value greater than printed p-value"
    \end{Verbatim}

    
    \begin{verbatim}

	KPSS Test for Trend Stationarity

data:  na.omit(additive_model$random)
KPSS Trend = 0.0083309, Truncation lag parameter = 3, p-value = 0.1

    \end{verbatim}

    
    \begin{Verbatim}[commandchars=\\\{\}]
Warning message in kpss.test(na.omit(multiplicative\_model\$random), null = "Trend"):
"p-value greater than printed p-value"
    \end{Verbatim}

    
    \begin{verbatim}

	KPSS Test for Trend Stationarity

data:  na.omit(multiplicative_model$random)
KPSS Trend = 0.010033, Truncation lag parameter = 3, p-value = 0.1

    \end{verbatim}

\pagebreak    
    \subparagraph{Phillips--Perron Birim Kök
Testi}\label{phillipsperron-birim-kuxf6k-testi}

Phillips--Perron Birim Kök Testi testini verinin bileşenlerden
ayrılmamış haline, toplamsal ve çarpımsal ayrışma modellerine sırasıyla
uyguluyarak verinin durağanlığını kontrol ediyoruz. Phillips--Perron
Birim Kök testini R içinde bulunan pp.test fonksiyonu ile elde
edebiliriz.

    \begin{Verbatim}[commandchars=\\\{\}]
{\color{incolor}In [{\color{incolor}14}]:} pp.test\PY{p}{(}data\PY{p}{[}\PY{p}{,}\PY{l+m}{2}\PY{p}{]}\PY{p}{)}
         pp.test\PY{p}{(}na.omit\PY{p}{(}additive\PYZus{}model\PY{o}{\PYZdl{}}random\PY{p}{)}\PY{p}{)}
         pp.test\PY{p}{(}na.omit\PY{p}{(}multiplicative\PYZus{}model\PY{o}{\PYZdl{}}random\PY{p}{)}\PY{p}{)}
\end{Verbatim}


    
    \begin{verbatim}

	Phillips-Perron Unit Root Test

data:  data[, 2]
Dickey-Fuller Z(alpha) = -12.85, Truncation lag parameter = 5, p-value
= 0.3898
alternative hypothesis: stationary

    \end{verbatim}

    
    \begin{Verbatim}[commandchars=\\\{\}]
Warning message in pp.test(na.omit(additive\_model\$random)):
"p-value smaller than printed p-value"
    \end{Verbatim}

    
    \begin{verbatim}

	Phillips-Perron Unit Root Test

data:  na.omit(additive_model$random)
Dickey-Fuller Z(alpha) = -89.666, Truncation lag parameter = 5, p-value
= 0.01
alternative hypothesis: stationary

    \end{verbatim}

    
    \begin{Verbatim}[commandchars=\\\{\}]
Warning message in pp.test(na.omit(multiplicative\_model\$random)):
"p-value smaller than printed p-value"
    \end{Verbatim}

    
    \begin{verbatim}

	Phillips-Perron Unit Root Test

data:  na.omit(multiplicative_model$random)
Dickey-Fuller Z(alpha) = -93.661, Truncation lag parameter = 5, p-value
= 0.01
alternative hypothesis: stationary

    \end{verbatim}

    
    Verinin durağanlığını ölçmek için üç farklı test uyguladık. Bunlar
adf,kpss ve pp testleridir. ADF ve PP testlerinin sonuçlarına göre
\(p =0.01 <0.05\) çıkmıştır. Yani yapılan testten oldukça eminiz. Dickey
Fuller değerleri de oldukça düşüktür. Verimiz için toplamsal ve
çarpımsal model için durağan diyebiliriz. KPSS testini incelediğimizde
ise \(p=0.1\) çıkmıştır.

    \subsection{Arima}\label{arima}

    R'da olan auto.arima fonksiyonu ile verinin kalan sezonsal, trend ve
diğer etkenleri kontrol eder ve onlardan arındırır. Veriye auto.arima
uygulandığında AIC , BIC ve AICc değerlerinin en düşük olanı analiz için
daha kabul edilebilir bir değer almaktadır. ARIMA için p,d,q
fonksiyonları ile verinin karmaşıklığı fark edilir ve en sade , en kolay
model seçilir.

    \begin{Verbatim}[commandchars=\\\{\}]
{\color{incolor}In [{\color{incolor}15}]:} auto.arima\PY{p}{(}ts\PY{p}{(}data\PY{p}{[}\PY{p}{,}\PY{l+m}{2}\PY{p}{]}\PY{p}{)}\PY{p}{)}
         auto.arima\PY{p}{(}ts\PY{p}{(}na.omit\PY{p}{(}additive\PYZus{}model\PY{o}{\PYZdl{}}random\PY{p}{)}\PY{p}{)}\PY{p}{)}
         auto.arima\PY{p}{(}ts\PY{p}{(}na.omit\PY{p}{(}multiplicative\PYZus{}model\PY{o}{\PYZdl{}}random\PY{p}{)}\PY{p}{)}\PY{p}{)}
\end{Verbatim}


    
    \begin{verbatim}
Series: ts(data[, 2]) 
ARIMA(2,1,3) 

Coefficients:
          ar1     ar2     ma1      ma2      ma3
      -0.1193  0.8104  0.2938  -0.8454  -0.2725
s.e.   0.0761  0.0812  0.0869   0.0789   0.0626

sigma^2 estimated as 0.3433:  log likelihood=-255.04
AIC=522.08   AICc=522.37   BIC=544.12
    \end{verbatim}

    
    
    \begin{verbatim}
Series: ts(na.omit(additive_model$random)) 
ARIMA(5,0,1) with zero mean 

Coefficients:
         ar1     ar2      ar3      ar4      ar5     ma1
      0.0428  0.4151  -0.1535  -0.1207  -0.2632  0.7428
s.e.  0.1316  0.1130   0.0631   0.0614   0.0572  0.1276

sigma^2 estimated as 0.2156:  log likelihood=-179.43
AIC=372.86   AICc=373.27   BIC=398.28
    \end{verbatim}

    
    
    \begin{verbatim}
Series: ts(na.omit(multiplicative_model$random)) 
ARIMA(2,0,0) with non-zero mean 

Coefficients:
         ar1      ar2    mean
      0.8284  -0.2604  0.9939
s.e.  0.0576   0.0575  0.0106

sigma^2 estimated as 0.005939:  log likelihood=321.52
AIC=-635.04   AICc=-634.9   BIC=-620.5
    \end{verbatim}

    
    Verinin hiçbir bileşenden ayrılmadığı durumda auto.arima uygulandığında
p =2 , d= 1 ve q=3 çıkmaktadır. d=1 olması verinin durağan olmadığını
gösterir.

Verinin toplamsal ayrışma modeline auto.arima uygulandığında p= 5 , d=0
ve q=1 çıkmaktadır. d=0 veri durağandır.

Verilerde çarpımsal modele auto.arima uygulandığında p= 1 , d=0 ve q=1
çıkmaktadır. d=0 veri durağandır.

Hangi ayrışma modelini seçmemiz gerektiğine karar vermek için ilk olarak
p,d,q değerlerini karşılaştırmak gerekmektedir. Verinin durağan modeli
seçilmesi gerektiğinden hiçbir bileşenden ayrılmadığı model elenir.
Toplamsal ve çarpımsal model karşılaştırıldığında çarpımsal modelin daha
basit olduğu görülür. İkinci olarak ( Seçtiğimiz modelin doğruluğundan
emin olmak için ) auto.arima nın bize vermiş olduğu AIC,AICc ve BIC
değerlerinin en küçük olanı seçilmesi gerekmektedir. Ayrışma modelleri
için incelendiğinde çarpımsal modelin daha uygun olduğu görülmektedir. .
Formülü :

\[ \hat Y_{d_t} =0.5006 Y_{t-1}  E\]

d=0 olduğunda orijinal seri zaten durağandır ve farkının alınmasına da
gerek yoktur.

    Veride sezonsal etkinin olup olmadığını kontrol etmek için veriden
hiçbir bileşenin çıkarılmadığı haline auto.arima uygulanır.Bunun için
arimanın argümanlarından sseasonal.test'ten yararlanılır.

    \begin{Verbatim}[commandchars=\\\{\}]
{\color{incolor}In [{\color{incolor}16}]:} auto.arima\PY{p}{(}ts\PY{p}{(}data\PY{p}{[}\PY{p}{,}\PY{l+m}{2}\PY{p}{]}\PY{p}{)}\PY{p}{)}
         auto.arima\PY{p}{(}ts\PY{p}{(}data\PY{p}{[}\PY{p}{,}\PY{l+m}{2}\PY{p}{]}\PY{p}{)} \PY{p}{,}seasonal.test \PY{o}{=} \PY{k+kt}{c}\PY{p}{(}\PY{l+s}{\PYZdq{}}\PY{l+s}{ocsb\PYZdq{}}\PY{p}{,} \PY{l+s}{\PYZdq{}}\PY{l+s}{ch\PYZdq{}}\PY{p}{)}\PY{p}{)}
\end{Verbatim}


    
    \begin{verbatim}
Series: ts(data[, 2]) 
ARIMA(2,1,3) 

Coefficients:
          ar1     ar2     ma1      ma2      ma3
      -0.1193  0.8104  0.2938  -0.8454  -0.2725
s.e.   0.0761  0.0812  0.0869   0.0789   0.0626

sigma^2 estimated as 0.3433:  log likelihood=-255.04
AIC=522.08   AICc=522.37   BIC=544.12
    \end{verbatim}

    
    
    \begin{verbatim}
Series: ts(data[, 2]) 
ARIMA(2,1,3) 

Coefficients:
          ar1     ar2     ma1      ma2      ma3
      -0.1193  0.8104  0.2938  -0.8454  -0.2725
s.e.   0.0761  0.0812  0.0869   0.0789   0.0626

sigma^2 estimated as 0.3433:  log likelihood=-255.04
AIC=522.08   AICc=522.37   BIC=544.12
    \end{verbatim}

    
    Herhangi bir işlem yapılmayan veriye auto.arima ile sezonsal test
uygulandığında ARIMA modelinde; p,d,q değerlerinde ve AIC , AICc ,BIC
değerlerinde herhangi bir farklılık görünmemektedir. Bir farklı ifade
ile veride sezonsal etki bulunmamaktadır.

    Normallik testini seçtiğimiz çarpımsal modele tekrar uygulanır.

    \begin{Verbatim}[commandchars=\\\{\}]
{\color{incolor}In [{\color{incolor}17}]:} white\PYZus{}noise\PY{o}{\PYZlt{}\PYZhy{}}arima\PY{p}{(}ts\PY{p}{(}na.omit\PY{p}{(}multiplicative\PYZus{}model\PY{o}{\PYZdl{}}random\PY{p}{)}\PY{p}{)}\PY{p}{,} order\PY{o}{=}\PY{k+kt}{c}\PY{p}{(}\PY{l+m}{1}\PY{p}{,}\PY{l+m}{0}\PY{p}{,}\PY{l+m}{1}\PY{p}{)}\PY{p}{)}
         shapiro.test\PY{p}{(}white\PYZus{}noise\PY{o}{\PYZdl{}}residuals\PY{p}{)}
\end{Verbatim}


    
    \begin{verbatim}

	Shapiro-Wilk normality test

data:  white_noise$residuals
W = 0.97796, p-value = 0.0002546

    \end{verbatim}

    
    Normallik testine göre p değeri oldukça düşük ve w değeri yüksektir.
Artık verinin beyaz gürültü olduğunu ve veride herhangi bir bileşen
kalmadığını söylenebilir.

    \subsection{Zaman Serilerinin Trend Bileşeninden  Arındırılması
(de-trend)}\label{zaman-serilerinin-trendden-arux131ndux131rux131lmasux131-de-trend}

Lineer trend barındıran zaman serilerini trendden arındırmak için
genellikle doğrusal regresyonu kullanılmaktadır.Zaman serisine doğrusal
regresyon uygulanırsa, trend etkisi içinde olmayan zaman serisi elde
edilir.

Doğrusal regresyon dışında da farklı yöntemler kullanılarak zaman serisi
verilerini trendlerden arındırılabilmektedir.Örneğin; AR(1), fourier ve
wavelet dönüşümleri, kalman filtreleri gibi yöntemler kullanılabilir, ya
da forecast kütüphanesi içerisindeki seasonaldummy ve fourier
fonksiyonları kullanılarak de-trend yapılabilir.

Model sonucu kalıntılar de-trend edilmiş zaman serisidir.

    Doğrusal regresyon analizini gerçekleştiren lm() fonksiyonu ilk olarak
bağımlı değişkeni, ardından bağımsız değişkenleri parametre olarak
almaktadır. Bağımlı değişkenden sonra tilda (\textasciitilde{}) işareti,
birden fazla bağımsız değişkenimiz varsa değişkenlerin arasına artı (+)
işareti konulması unutulmamalıdır.

    Verimizi trendsel etkiden arındırmak için hem doğrusal regresyon
yöntemini hem de 1. derece otoregresif süreç yöntemi uygulanır.

Trendsel etkiden arındırıldıktan sonra normallik testi uygulanarak
yaptığımız modelin doğruluğundan ve normal olup olmadığına karar
verebiliriz.

    \begin{Verbatim}[commandchars=\\\{\}]
{\color{incolor}In [{\color{incolor}18}]:} trModel \PY{o}{\PYZlt{}\PYZhy{}} lm\PY{p}{(}data\PY{p}{[}\PY{p}{,}\PY{l+m}{2}\PY{p}{]}\PY{o}{\PYZti{}} \PY{k+kt}{c}\PY{p}{(}\PY{l+m}{1}\PY{o}{:}\PY{k+kp}{length}\PY{p}{(}data\PY{p}{[}\PY{p}{,}\PY{l+m}{2}\PY{p}{]}\PY{p}{)}\PY{p}{)}\PY{p}{)}
         trResid \PY{o}{\PYZlt{}\PYZhy{}} resid\PY{p}{(}trModel\PY{p}{)}
         
         trModel\PYZus{}AR1 \PY{o}{\PYZlt{}\PYZhy{}} arima\PY{p}{(}x\PY{o}{=}data\PY{p}{[}\PY{p}{,}\PY{l+m}{2}\PY{p}{]}\PY{p}{,} order\PY{o}{=}\PY{k+kt}{c}\PY{p}{(}\PY{l+m}{1}\PY{p}{,}\PY{l+m}{0}\PY{p}{,}\PY{l+m}{0}\PY{p}{)}\PY{p}{)}
         trResid\PYZus{}AR1 \PY{o}{\PYZlt{}\PYZhy{}} resid\PY{p}{(}trModel\PYZus{}AR1\PY{p}{)}
\end{Verbatim}


    \begin{Verbatim}[commandchars=\\\{\}]
{\color{incolor}In [{\color{incolor}19}]:} par\PY{p}{(}mfrow\PY{o}{=}\PY{k+kt}{c}\PY{p}{(}\PY{l+m}{1}\PY{p}{,}\PY{l+m}{2}\PY{p}{)}\PY{p}{)}
         plot\PY{p}{(}trResid\PY{p}{,} type\PY{o}{=}\PY{l+s}{\PYZdq{}}\PY{l+s}{l\PYZdq{}}\PY{p}{,} col\PY{o}{=}\PY{l+s}{\PYZdq{}}\PY{l+s}{blue\PYZdq{}}\PY{p}{,} main\PY{o}{=}\PY{l+s}{\PYZdq{}}\PY{l+s}{de\PYZhy{}trended (regresyon)\PYZdq{}}\PY{p}{)}
         plot\PY{p}{(}trResid\PYZus{}AR1\PY{p}{,}  type\PY{o}{=}\PY{l+s}{\PYZdq{}}\PY{l+s}{l\PYZdq{}}\PY{p}{,} col\PY{o}{=}\PY{l+s}{\PYZdq{}}\PY{l+s}{red\PYZdq{}}\PY{p}{,} main\PY{o}{=}\PY{l+s}{\PYZdq{}}\PY{l+s}{de\PYZhy{}trended (ar(1))\PYZdq{}}\PY{p}{)}
         shapiro.test\PY{p}{(}trResid\PY{p}{)}
         shapiro.test\PY{p}{(}trResid\PYZus{}AR1\PY{p}{)}
\end{Verbatim}


    
    \begin{verbatim}

	Shapiro-Wilk normality test

data:  trResid
W = 0.83751, p-value < 2.2e-16

    \end{verbatim}

    
    
    \begin{verbatim}

	Shapiro-Wilk normality test

data:  trResid_AR1
W = 0.82996, p-value < 2.2e-16

    \end{verbatim}

    
    \begin{center}
    \adjustimage{max size={0.9\linewidth}{0.9\paperheight}}{output_56_2.png}
    \end{center}
    { \hspace*{\fill} \\}
    
    Doğrusal regresyon ve otoregresif süreç karşılaştırıldığında otoregresif
sürecin daha tutarlı olduğu , doğrusal regresyonun ise sürekli inişli ve
çıkışlı bir durumda olduğu gözlenmektedir.

Normallik testleri açısından karşılaştırırsak iki sonuçta da p değerinin
güven aralığını sağladığı yani \(p<0.05\) olduğunu ve w değerinin 1
yakın olduğunu görmekteyiz. İki trendsel etkiden arındırılmış veri için
normal denilebilir.

    \subsection{Zaman Serilerinin Mevsimsel Bileşeninden Arındırılması
(de-seasonalize)}\label{zaman-serilerinin-mevsimsellikten-arux131ndux131rux131lmasux131-de-seasonalize}

Zaman serisinin mevsimsellik deseni hakkında fikir verir ve mevsimsel
etkiler olmadan verilerin modellenmesine yardımcı olur. Zaman serisini
mevsimsellikten arındırmak için aşağıdaki iki adım kullanılabilir. 1.
Adım: stl() fonksiyonu ile zaman serisi ayrılır. 2. Adım: forecast
kütüphanesi içerisindeki seasadj() fonksiyonu kullanılır.

    \begin{Verbatim}[commandchars=\\\{\}]
{\color{incolor}In [{\color{incolor}20}]:} datastl\PY{o}{\PYZlt{}\PYZhy{}} stl\PY{p}{(}ts\PY{p}{(}data\PY{p}{[}\PY{p}{,}\PY{l+m}{2}\PY{p}{]}\PY{p}{,} freq\PY{o}{=}\PY{l+m}{12}\PY{p}{)}\PY{p}{,} s.window\PY{o}{=}\PY{l+s}{\PYZdq{}}\PY{l+s}{periodic\PYZdq{}}\PY{p}{)}
         plot\PY{p}{(}datastl\PY{p}{)}
\end{Verbatim}


    \begin{center}
    \adjustimage{max size={0.9\linewidth}{0.9\paperheight}}{output_59_0.png}
    \end{center}
    { \hspace*{\fill} \\}
    
    \begin{Verbatim}[commandchars=\\\{\}]
{\color{incolor}In [{\color{incolor}21}]:} ts\PYZus{}sa \PY{o}{\PYZlt{}\PYZhy{}} seasadj\PY{p}{(}datastl\PY{p}{)}
\end{Verbatim}


    \begin{Verbatim}[commandchars=\\\{\}]
{\color{incolor}In [{\color{incolor}22}]:} par\PY{p}{(}mfrow\PY{o}{=}\PY{k+kt}{c}\PY{p}{(}\PY{l+m}{3}\PY{p}{,}\PY{l+m}{2}\PY{p}{)}\PY{p}{)}
         plot\PY{p}{(}data\PY{p}{[}\PY{p}{,}\PY{l+m}{2}\PY{p}{]}\PY{p}{,} type\PY{o}{=}\PY{l+s}{\PYZdq{}}\PY{l+s}{l\PYZdq{}}\PY{p}{,} main\PY{o}{=}\PY{l+s}{\PYZdq{}}\PY{l+s}{Real\PYZdq{}}\PY{p}{)}  
         plot\PY{p}{(}ts\PYZus{}sa\PY{p}{,} type\PY{o}{=}\PY{l+s}{\PYZdq{}}\PY{l+s}{l\PYZdq{}}\PY{p}{,} main\PY{o}{=}\PY{l+s}{\PYZdq{}}\PY{l+s}{Seasonal Adjusted\PYZdq{}}\PY{p}{)}
\end{Verbatim}

\pagebreak  
    \begin{center}
    \adjustimage{max size={0.9\linewidth}{0.9\paperheight}}{output_61_0.png}
    \end{center}
    { \hspace*{\fill} \\}
 
    Normallik testini ölçmek için sezonsal etkisi çıkarılmış veriye
shapiro.test uygulanır.

    \begin{Verbatim}[commandchars=\\\{\}]
{\color{incolor}In [{\color{incolor}23}]:} shapiro.test\PY{p}{(}ts\PYZus{}sa\PY{p}{)}
\end{Verbatim}


    
    \begin{verbatim}

	Shapiro-Wilk normality test

data:  ts_sa
W = 0.86198, p-value = 1.769e-15

    \end{verbatim}

    
    Durağanlık testine göre \(p<0.05\) ve w değeri 1'e yakındı.Bu durum için
yapılan sezonsal testin sonucunda doğruluktan eminiz diyebiliriz.

    \begin{Verbatim}[commandchars=\\\{\}]
{\color{incolor}In [{\color{incolor}24}]:} acf\PY{p}{(}ts\PYZus{}sa\PY{p}{)}
\end{Verbatim}


    \begin{center}
    \adjustimage{max size={0.9\linewidth}{0.9\paperheight}}{output_65_0.png}
    \end{center}
    { \hspace*{\fill} \\}
    
    Otokorelasyon katsayısı incelendiğinde verinin yavaş yavaş azaldığı
gözlenmektedir. Bu durumda ts\_sa mevsimsel etkiden arındırılmış zaman
serimizin durağan olmadığını göstermektedir.
\pagebreak  
    \begin{Verbatim}[commandchars=\\\{\}]
{\color{incolor}In [{\color{incolor}25}]:} pacf\PY{p}{(}ts\PYZus{}sa\PY{p}{)}
\end{Verbatim}


    \begin{center}
    \adjustimage{max size={0.9\linewidth}{0.9\paperheight}}{output_67_0.png}
    \end{center}
    { \hspace*{\fill} \\}
    
    Sezonsal etkiden arındırılan verinin (ts\_sa) , hala sezonsal etki
içerip içermediğini anlamak (mevsimsellik farkını bulmak) için nsdiffs
fonksiyonu uygulanır.

    \begin{Verbatim}[commandchars=\\\{\}]
{\color{incolor}In [{\color{incolor}26}]:} nsdiffs\PY{p}{(}ts\PYZus{}sa\PY{p}{)}
\end{Verbatim}


    0

    
    Mevsimsellik farkı sıfırdır. Mevsimsel etki yoktur.
\pagebreak
    \subsection{Natural Gas Futures Zaman Serisi Analizi
Özet}\label{natural-gas-futures-zaman-serisi-analizi-uxf6zet}

Verinin Adı: Natural Gas Futures Contract 3

Verinin Alındığı Yer: Quandl

Verinin Sıklığı: Aylık

Verinin Zaman Aralığı: 31.01.1994- 31.03.2018

Veride iki sütun bulunmaktadır. Sütunlardan birisi zaman, diğeri ise
kullanılan değerdir. Amaç zaman serisindeki gözlenen değerlerinin trend,
mevsimsel, çevresel ve rastgele etkenlerden ayırmak ve doğru modeli
seçmektir. Bu doğrultuda üç yöntem kullandık.

\begin{enumerate}
\def\labelenumi{\arabic{enumi}.}
\item
  Yöntem : R'da bulunan decompose fonksiyonu ile seriyi mevsimsel ve
  trendsel etkenlerden ayırmaktır. Decompose fonksiyonunu zaman
  serilerin hep toplamına hem de çarpımına ayrı ayrı uyguladık. Daha net
  sonuç veren veri seçilir.Toplamsal ayrışma modelini incelediğimizde
  veride trensel etkinin sürekli değiştiğini ve mevsimsel etkinin aynı
  olduğunu görüyoruz. Çarpımsal model de aynı sonuca varıyoruz. İki
  ayrıştırma modeli arasındaki fark random bileşende ortaya
  çıkmaktadır.Ancak kullanılan veride toplanan ve çarpılan veride net
  bir karar veremiyoruz. Verimizin zaman bileşenlerinden ayırldıktan
  sonraki normallik durumunu kontrol etmek için veriye normallik
  testi(shapiro.test )uyguladık, iki modelin de p değerleri güven
  aralığında (\(p<0.05\) ) çıktı. Sonuç olarak iki model içinde
  verimizin bileşenlerden ayrılmış haline normal diyebiliriz. Zaman
  bileşenlerinden ayrılmış olan verimizin durağanlığını kontrol etmek
  için adf,kpss ve pp testlerini uyguladık. Üç testin sonucu da
  verimizin toplamsal ve çarpımsal model için durağan olduğunu gösterdi.
  Sonraki aşamalarda decompose fonksiyonu ile veremediğimiz hangi
  modelin uygun olduğu kararını anlamak için arima testini uyguladık.
  Arima ile gelen p,q,d değerlerinden en basit olanını ve AIC, AICc ve
  BIC değerlerinden en düşük olanını yani çarpımsal modelin daha uygun
  olduğuna karar verdik. Ek olarak herhangi bir bileşenden ayrılmamış
  veriye arima ile sezonsal etkinin olup olmadığını ölçtük. Sonucunda
  ise herhangi bir farklılık olmadığını yani verimizde sezonsal bir
  etkinin olmadığını söyleyebiliriz. Decompose ve sonrasında uygulanan
  auto.arima ile kalan verimize normallik testini tekrar uygularsak
  güven aralığını sağladığını (\( p=0.0002966 \textless{}0.05 \)) ve w=
 0.97827 yani 1 oldukça yakın olduğunu görüyoruz. Bu da verimizin
  normal olduğunu söyler. Ek olarak artık verimize beyaz gürültü verisi
  diyebiliriz.
\item
  Yöntem : Zaman serisini trend bileşenlerinden ayırmak için doğrısal
  regresyon ve AR(1) modelini kullandık. Verilerin trendsel etkisinden
  ayrıldığı durumda normallik değerlerini ölçmek için her iki (Doğrusal
  regresyon,AR(1)) süreç için güven aralığını ( \(p=2.2e-16
  \textless{}0.05 \) ve w=0.83..) sağladığını gördük.
\item
  Yöntem :Zaman serisini mevsimsel bileşenlerinden ayırmak için R'da
  bulunan stl fonksiyonunu kullandık. Verilerin mevsimsel etkisinden
  ayrıldığı durumda normallik değerlerini ölçmek için normallik
  (shapiro.test) testini uyguladık. Güven aralığını ( \(p=2.098e-15
  \textless{}0.05 \) ve w=0.86274) sağladığını gördük.Ancak mevsimsellik
  etkisinden arındırılmış veri için acf testi uygulandığında verinin
  yavaş yavaş azaldığı gözlenmektedir. Bu durum için verinin normal
  olmadığı söylenmektedir.
\end{enumerate}

Ek olarak uyguladığımız nsdiffs fonksiyonu ile birlikte verinin
mevsimsel etki içerip içermediğine, ya da kalıp kalmadığına karar
verilmektedir. Fonksiyonu uyguladığımızda görülüyor ki veri mevsimsel
bileşen içermemektedir.

Yukarıdaki üç yöntemi karşılatırdığımızda (normalliklerini) kullanmamız
gereken modelin çarpımsal ve kullanmamız gereken yöntemin sırasıyla
decompose ve arima olmasına karar verdik.

    \subsection{Forecast}\label{forecast}

    \begin{Verbatim}[commandchars=\\\{\}]
{\color{incolor}In [{\color{incolor}27}]:} \PY{k+kn}{require}\PY{p}{(}forecast\PY{p}{)}
\end{Verbatim}


    \subsubsection{Üstel Düzleştirme -Exponential
Smooting}\label{uxfcssel-duxfczguxfcnleux15ftirme--exponential-smooting}

Verideki son değişim ve sıçramaları incelenerek tahminlerin ya da
öngörülerin devamlı olarak güncelleştirildiği bir yöntemdir.
Deterministik ve stokastik trende sahip olan tüm serilere
uygulanabilmektedir. Üstel düzleştirme yönteminde gelecek dönemin (t+1
döneminin) tahminin hesabında son döneme ait tahmin ile bu tahminden
elde edilen hatanın bir kısmı kullanılmaktadır.
\[ Y_{t+1}= Y_{t}+ a(e_{t}) \] ile ifade edilir.

\[ S_{t}= S_{t-1}+ a(Y_{t}-S_{t-1} ) \] \(S_{t}\): Bir sonraki (gelecek)
dönemin, yani (t+1)'inci dönemin tahmini

\(S_{t-1}\): Son döneme ait tahmin ( bu tahmin (t-1)'inci dönemde
yapılır.

\(Y_{t}\): Son döneme ait gerçek değer

a: Düzleştirme katsayısı yada ağırlık \(0<a<1\)

\(Y_{t}-S_{t-1}\): Son döneme ait tahmin hatası

\[ Y_{t+1}= S_{t}= a(Y_{t+1})+ (1-a)S_{t-1}\]

Gelecek dönemin tahmini=ağırlık.(son dönemin gözlem
değeri)+(1-ağırlık).(son döneme ait tahmin) Genel terimini yazmak
istenirse:\\
\[ \hat Y_{t+1} = S_{t} = a \sum_{i=1}^t (1-a)^i \cdot Y_{t-i}  \]

Üstel düzleştirme yöntemi uyguladığımızda trend ve mevsimsel
bileşenler için toplamda 15 metot vardır. - Trendsel Bileşen : N(None)
- A(Additive)- \(A_{d}\) (Additive Damped) - M(Multiplicative) -
\(M_{d}\) (Multiplicative Damped) - Sezonsal Bileşen: N(None) -
A(Additive)- M(Multiplicative)

ETS ile gelen metodta Ets(error, trend , seasonal ) sonuçları elde
edilir.Örnek olarak birkaç model incelenirse:

ETS(A,A,A) : Holt-Winters' metodunun toplamsal ayrışması ve toplamsal
ayrışmadaki hatalar yöntemi

ETS(M,A,M) : Holt-Winters' metodunun çarpımsal ayrışması ve çarpımsal
ayrışmadaki hatalar yöntemi

ETS(A,\(A_{d}\), N) : Toplamsal hata ve sönümlü trendsel etki içerip ,
sezonsal etki içermeyen yöntem

Üssel düzgünleştirme de 2 farklı yöntem vardır. Bu iki yöntem için de R
da bulunan fonksiyonlar kullanılır. - ets() fonksiyonu - HoltWinters()
fonkisyonu

    İlk olarak verimizin zaman serisi olduğunu gösteriyoruz. Zaman serisi
sıklığını aylık olarak tanımlıyoruz. Zaman aralığımızı 01.1994 - 12.2017
kadar olan süreyi alıyoruz. 01.2018-03.2018 süresini de test verisi
olarak alıyoruz.

    \begin{Verbatim}[commandchars=\\\{\}]
{\color{incolor}In [{\color{incolor}28}]:} time\PYZus{}series \PY{o}{\PYZlt{}\PYZhy{}}ts\PY{p}{(}data\PY{p}{[}\PY{p}{,}\PY{l+m}{2}\PY{p}{]}\PY{p}{,} start\PY{o}{=}\PY{k+kt}{c}\PY{p}{(}\PY{l+m}{1994}\PY{p}{,} \PY{l+m}{1}\PY{p}{)}\PY{p}{,} end\PY{o}{=}\PY{k+kt}{c}\PY{p}{(}\PY{l+m}{2017}\PY{p}{,} \PY{l+m}{12}\PY{p}{)}\PY{p}{,} freq\PY{o}{=}\PY{l+m}{12}\PY{p}{)}
         plot\PY{p}{(}time\PYZus{}series\PY{p}{)}
\end{Verbatim}


    \begin{center}
    \adjustimage{max size={0.9\linewidth}{0.9\paperheight}}{output_76_0.png}
    \end{center}
    { \hspace*{\fill} \\}
    
    Aşağıda üstel düzleştirme yönteminin ets() fonksiyonu kullanılmış
halini uygulayacağız.
\subsubsection{Öngörü Doğruluğunu Hesaplama - Accuracy Fonksiyonu}

Öngörü doğruluğunu hesaplamak için R'ın accuracy
fonksiyonunu kullanılır. Accuracy fonksiyonu ile gelen verilerde:

ME:Ortalama Hata \[  ME = 1/n \sum_{t=1}^n e_{t}  \]\\
Bu sonucun dezavantajı,Pozitif ve negatif hata değerlerinin birbirini
iptal edebileceğidir, bu yüzden bu ölçü genel uyumun çok iyi bir
göstergesi değildir.

MAE : Ortalama Mutlak Hata \[  MAE = 1/n \sum_{t=1}^n |e_{t}|  \]\\
Bu değer 0 ise, uyum (tahmin) mükemmeldir. Ortalama karesel hata
değerine kıyasla, bu uyum ölçüsü aykırı değerleri "vurgulamak", yani
benzersiz veya nadir büyük hata değerleri MAE'yi MSE değerinden daha az
etkileyecektir.

MSE: Ortalama Hata Kare \[  MSE = 1/n \sum_{t=1}^n e_{t}^2  \]\\
MPE: Ortalama Yüzde Hata \[  MPE = 1/n \sum_{t=1}^n PE_{t}  \] Bu değer
PE değerlerinin ortalaması olarak hesaplanır.

MAPE: Ortalama Mutlak Yüzde Hata \[  MAPE = 1/n \sum_{t=1}^n |PE_{t}| \]

\(e_{t}\) : t dönemindeki hata

\(PE_{t}\) : t dönemindeki yüzde hata

\(PE_{t}= [(Y_{t}-F_{t})/Y_{t}]\cdot 100\)

n: dönem sayısı

    Öngörü uygulandığında çıkan değerlerde 80 ve 95 güven aralığını belli
etmektedir. Gri olan bölgeler öngörüden \%95 emin olduğumuzu , mavi
bölge ise öngörüden \%80 emin olduğumuzu göstermektedir.

    Öngörü modellerinden ilk olarak ets modelini deniyoruz. Test modülümüzü
2018.01 ve 2018.03 arasına uyguluyoruz.

    \begin{Verbatim}[commandchars=\\\{\}]
{\color{incolor}In [{\color{incolor}29}]:} model11 \PY{o}{\PYZlt{}\PYZhy{}}ets\PY{p}{(}time\PYZus{}series\PY{p}{)}
         data\PYZus{}forecast\PYZus{}ets \PY{o}{\PYZlt{}\PYZhy{}} forecast\PY{p}{(}model11\PY{p}{,} h\PY{o}{=}\PY{l+m}{3}\PY{p}{)}
         accuracy\PY{p}{(}data\PYZus{}forecast\PYZus{}ets\PY{p}{)}
         plot\PY{p}{(}data\PYZus{}forecast\PYZus{}ets\PY{p}{,} \PY{p}{,}xlim\PY{o}{=}\PY{k+kt}{c}\PY{p}{(}\PY{l+m}{2018.1}\PY{p}{,}\PY{l+m}{2018.3}\PY{p}{)}\PY{p}{)}
         data\PYZus{}forecast\PYZus{}ets
\end{Verbatim}


    \begin{tabular}{r|lllllll}
  & ME & RMSE & MAE & MPE & MAPE & MASE & ACF1\\
\hline
	Training set & -0.001786263 & 0.5867771    & 0.3525912    & -0.490289    & 7.254289     & 0.2780497    & 0.1595532   \\
\end{tabular}


    
    
    \begin{verbatim}
         Point Forecast    Lo 80    Hi 80    Lo 95    Hi 95
Jan 2018       1.974613 1.739441 2.209785 1.614948 2.334278
Feb 2018       1.846689 1.522929 2.170450 1.351541 2.341838
Mar 2018       1.787294 1.397092 2.177495 1.190532 2.384055
    \end{verbatim}

    
    \begin{center}
    \adjustimage{max size={0.9\linewidth}{0.9\paperheight}}{output_80_2.png}
    \end{center}
    { \hspace*{\fill} \\}
    
    Verimize ets() fonksiyonu ile öngörü hesapladığımızda bize
ETS(M,\(A_{d}\),M) argümanlarını vermiştir. Bu da verimizide 
çarpımsal hatanın, sönümlü trendin ve çarpımsal mevsimsel etkinin sahip
olduğunu gösteriyor. Çıkan sonuçları orjinal değerler ile
karşılaştırıldığında :

Orjinal değerler:

\begin{itemize}
\tightlist
\item
  2018.03 2.784
\item
  2018.02 2.695
\item
  2018.01 2.800
\end{itemize}

Öngörü hesabındaki güven aralığı \%95 olan noktalardaki sonuçlar ise:

\begin{itemize}
\tightlist
\item
  2018.03 2.660738
\item
  2018.02 2.514919
\item
  2018.01 2.431952
\end{itemize}

olarak hesaplanmıştır. İki değer arasındaki farklar incelenirse : -
2018.03 0.089 - 2018.02 0.18001 - 2018.01 o.368048

olarak hesaplanmıştır.

    Düşük bir aplha değeri, geçmiş ortalamaya daha fazla ağırlık verir ve
rassal dalgalanmaların etkisini azaltır. Yüksek alpha değeri ise
talepteki değişikliklere daha fazla tepki verilmesini sağlar.

    \begin{Verbatim}[commandchars=\\\{\}]
{\color{incolor}In [{\color{incolor}30}]:} \PY{k+kp}{summary}\PY{p}{(}data\PYZus{}forecast\PYZus{}ets\PY{p}{)}
\end{Verbatim}


    \begin{Verbatim}[commandchars=\\\{\}]

Forecast method: ETS(M,Ad,A)

Model Information:
ETS(M,Ad,A) 

Call:
 ets(y = time\_series) 

  Smoothing parameters:
    alpha = 0.9999 
    beta  = 0.0066 
    gamma = 1e-04 
    phi   = 0.9255 

  Initial states:
    l = 2.6888 
    b = -0.0089 
    s=0.0223 0.045 -0.051 0.0241 0.1746 0.2976
           0.1571 -0.0229 -0.2054 -0.2297 -0.1699 -0.0417

  sigma:  0.0929

     AIC     AICc      BIC 
1071.518 1074.061 1137.452 

Error measures:
                       ME      RMSE       MAE       MPE     MAPE      MASE
Training set -0.001786263 0.5867771 0.3525912 -0.490289 7.254289 0.2780497
                  ACF1
Training set 0.1595532

Forecasts:
         Point Forecast    Lo 80    Hi 80    Lo 95    Hi 95
Jan 2018       1.974613 1.739441 2.209785 1.614948 2.334278
Feb 2018       1.846689 1.522929 2.170450 1.351541 2.341838
Mar 2018       1.787294 1.397092 2.177495 1.190532 2.384055

    \end{Verbatim}

    ETS(M,\(A_{d}\), M) için alpha değeri 0.988 çıkmıştır. Bu da 
talepteki değişikliklere daha fazla tepki vereceğini söylemektedir.

    \subsubsection{Basit Üstel Düzleştirme Yöntemi (Durağan Seriler için):}\label{basit-uxfcssel-duxfczguxfcnleux15ftirme-yuxf6ntemi-duraux11fan-seriler-iuxe7in}

Bu yöntem trende ve mevsimsel dalgalanmaya sahip olmayan sadece bir
ortalama düzey etrafında hareket eden serilerin analizinde
uygulanmaktadır. Bu yöntemde serinin tahminin elde edilebilmesi için
aşağıdaki formülden yararlanılmaktadır.

Düzleştirme katsayısı(aplha- a) tahmin hatasını en küçük yapan sabit bir
değerdir. Düzleştirme katsayısı (alpha )1 değerine yakın ise son
gözlemlerin değerleri doğrudan serinin tahminini etkiler ve bu durum
tahmin serisinde aşırı sıçramalara neden olabilir. Düzleştirme
katsayısının 0 değerine yakın olması durumunda orijinal serideki
değişimler tahmin serisini pek etkileyemez. Tahmin serisindeki ilk
verilerle son veriler arasında önemli bir fark olmaz. Bu iki durum da
tahminlerin güvenilir olması bakımından istenilen bir özellik değildir.

    Aşağıda üstel düzleştirme yönteminin HoltWinters() fonksiyonu
kullanılmış halini uygulayacağız.Bu model trend, sezonsal bileşenleri de
içermemediği durumdur.

    \begin{Verbatim}[commandchars=\\\{\}]
{\color{incolor}In [{\color{incolor}31}]:} model12 \PY{o}{\PYZlt{}\PYZhy{}}HoltWinters\PY{p}{(}time\PYZus{}series\PY{p}{,} beta\PY{o}{=}\PY{k+kc}{FALSE}\PY{p}{,} gamma\PY{o}{=}\PY{k+kc}{FALSE}\PY{p}{)}
         model12
         data\PYZus{}forecast\PYZus{}simple \PY{o}{\PYZlt{}\PYZhy{}} forecast\PY{p}{(}model12\PY{p}{,} h\PY{o}{=}\PY{l+m}{3}\PY{p}{)}
         accuracy\PY{p}{(}data\PYZus{}forecast\PYZus{}simple\PY{p}{)}
         plot\PY{p}{(}data\PYZus{}forecast\PYZus{}simple\PY{p}{,} \PY{p}{,}xlim\PY{o}{=}\PY{k+kt}{c}\PY{p}{(}\PY{l+m}{2018.1}\PY{p}{,}\PY{l+m}{2018.3}\PY{p}{)}\PY{p}{)}
         data\PYZus{}forecast\PYZus{}simple
\end{Verbatim}


    
    \begin{verbatim}
Holt-Winters exponential smoothing without trend and without seasonal component.

Call:
HoltWinters(x = time_series, beta = FALSE, gamma = FALSE)

Smoothing parameters:
 alpha: 0.9999286
 beta : FALSE
 gamma: FALSE

Coefficients:
      [,1]
a 2.038008
    \end{verbatim}

    
    \begin{tabular}{r|lllllll}
  & ME & RMSE & MAE & MPE & MAPE & MASE & ACF1\\
\hline
	Training set & -0.002665668 & 0.6110392    & 0.3739591    & -0.6182691   & 7.84757      & 0.2949002    & 0.1759845   \\
\end{tabular}


    
    
    \begin{verbatim}
         Point Forecast     Lo 80    Hi 80      Lo 95    Hi 95
Jan 2018       2.038008 1.2535693 2.822447  0.8383126 3.237703
Feb 2018       2.038008 0.9286838 3.147332  0.3414431 3.734573
Mar 2018       2.038008 0.6793851 3.396631 -0.0398264 4.115842
    \end{verbatim}

    
    \begin{center}
    \adjustimage{max size={0.9\linewidth}{0.9\paperheight}}{output_87_3.png}
    \end{center}
    { \hspace*{\fill} \\}
    
    \begin{Verbatim}[commandchars=\\\{\}]
{\color{incolor}In [{\color{incolor}32}]:} \PY{k+kp}{summary}\PY{p}{(}data\PYZus{}forecast\PYZus{}simple\PY{p}{)}
\end{Verbatim}


    \begin{Verbatim}[commandchars=\\\{\}]

Forecast method: HoltWinters

Model Information:
Holt-Winters exponential smoothing without trend and without seasonal component.

Call:
HoltWinters(x = time\_series, beta = FALSE, gamma = FALSE)

Smoothing parameters:
 alpha: 0.9999286
 beta : FALSE
 gamma: FALSE

Coefficients:
      [,1]
a 2.038008

Error measures:
                       ME      RMSE       MAE        MPE    MAPE      MASE
Training set -0.002665668 0.6110392 0.3739591 -0.6182691 7.84757 0.2949002
                  ACF1
Training set 0.1759845

Forecasts:
         Point Forecast     Lo 80    Hi 80      Lo 95    Hi 95
Jan 2018       2.038008 1.2535693 2.822447  0.8383126 3.237703
Feb 2018       2.038008 0.9286838 3.147332  0.3414431 3.734573
Mar 2018       2.038008 0.6793851 3.396631 -0.0398264 4.115842

    \end{Verbatim}

    Düzleştirme katsayı değeri arttıkça daha büyük tahmin farklılıkları
oluşmaktadır. Basit üstel düzleştirme yöntemi uyguladığımızda gelen
alpha değeri 0.9999435'dir. Bu sonuçta 1'e çok yakın ve değerin çok
büyük olduğunu, bileşenlere yüksek tepki vereceğini gösterir. Hata oranı
fazladır. O yüzden öngörü yaparken bu model seçilmemelidir.

    \subsubsection{Holt ÜsteL Düzleştirme Yöntemi: (Doğrusal Trendi Olan Seriler için)}\label{holt-uxfcstel-duxfczleux15ftirme-yuxf6ntemi-doux11frusal-trendi-olan-seriler-iuxe7in}

Trende sahip mevsimsel dalgalanması olmayan serilerin tahmin işleminde
Holt üstel düzleştirme yöntemi kullanılmaktadır.

    \begin{Verbatim}[commandchars=\\\{\}]
{\color{incolor}In [{\color{incolor}33}]:} model13 \PY{o}{\PYZlt{}\PYZhy{}}HoltWinters\PY{p}{(}time\PYZus{}series\PY{p}{,} gamma\PY{o}{=}\PY{k+kc}{FALSE}\PY{p}{)}
         model13
         data\PYZus{}forecast\PYZus{}holt \PY{o}{\PYZlt{}\PYZhy{}} forecast\PY{p}{(}model13\PY{p}{,} h\PY{o}{=}\PY{l+m}{3}\PY{p}{)}
         accuracy\PY{p}{(}data\PYZus{}forecast\PYZus{}holt\PY{p}{)}
         plot\PY{p}{(}data\PYZus{}forecast\PYZus{}holt\PY{p}{,} \PY{p}{,}xlim\PY{o}{=}\PY{k+kt}{c}\PY{p}{(}\PY{l+m}{2018.1}\PY{p}{,}\PY{l+m}{2018.3}\PY{p}{)}\PY{p}{)}
         data\PYZus{}forecast\PYZus{}holt
\end{Verbatim}


    
    \begin{verbatim}
Holt-Winters exponential smoothing with trend and without seasonal component.

Call:
HoltWinters(x = time_series, gamma = FALSE)

Smoothing parameters:
 alpha: 1
 beta : 0
 gamma: FALSE

Coefficients:
    [,1]
a  2.038
b -0.019
    \end{verbatim}

    
    \begin{tabular}{r|lllllll}
  & ME & RMSE & MAE & MPE & MAPE & MASE & ACF1\\
\hline
	Training set & 0.01639161  & 0.6123117   & 0.3755175   & -0.06358456 & 7.867708    & 0.2961291   & 0.1759017  \\
\end{tabular}


    
    
    \begin{verbatim}
         Point Forecast     Lo 80    Hi 80      Lo 95    Hi 95
Jan 2018          2.019 1.2331972 2.804803  0.8172183 3.220782
Feb 2018          2.000 0.8887070 3.111293  0.3004241 3.699576
Mar 2018          1.981 0.6199496 3.342050 -0.1005469 4.062547
    \end{verbatim}

    
    \begin{center}
    \adjustimage{max size={0.9\linewidth}{0.9\paperheight}}{output_91_3.png}
    \end{center}
    { \hspace*{\fill} \\}
    
    \begin{Verbatim}[commandchars=\\\{\}]
{\color{incolor}In [{\color{incolor}34}]:} \PY{k+kp}{summary}\PY{p}{(}data\PYZus{}forecast\PYZus{}holt\PY{p}{)}
\end{Verbatim}


    \begin{Verbatim}[commandchars=\\\{\}]

Forecast method: HoltWinters

Model Information:
Holt-Winters exponential smoothing with trend and without seasonal component.

Call:
HoltWinters(x = time\_series, gamma = FALSE)

Smoothing parameters:
 alpha: 1
 beta : 0
 gamma: FALSE

Coefficients:
    [,1]
a  2.038
b -0.019

Error measures:
                     ME      RMSE       MAE         MPE     MAPE      MASE
Training set 0.01639161 0.6123117 0.3755175 -0.06358456 7.867708 0.2961291
                  ACF1
Training set 0.1759017

Forecasts:
         Point Forecast     Lo 80    Hi 80      Lo 95    Hi 95
Jan 2018          2.019 1.2331972 2.804803  0.8172183 3.220782
Feb 2018          2.000 0.8887070 3.111293  0.3004241 3.699576
Mar 2018          1.981 0.6199496 3.342050 -0.1005469 4.062547

    \end{Verbatim}

    Düzleştirme katsayı değeri arttıkça daha büyük tahmin farklılıkları
oluşmaktadır. Basit üstel düzleştirme yöntemi uyguladığımızda gelen
alpha değeri 1'dir. Bu sonuçta değerin çok büyük olduğunu gösterir. Hata
oranı fazladır. O yüzden öngörü yaparken bu model seçilmemelidir.

    \subsubsection{Holt-Winters Yöntemi: (Doğrusal Trendi ve Mevsimlik Etkileri
Olan Seriler
İçin)}\label{holt-winters-yuxf6ntemi-doux11frusal-trendi-ve-mevsimlik-etkileri-olan-seriler-iuxe7in}

Holt-Winters Yöntemi üç temel düzeltme esitligine sahiptir. Temel, trend
ve mevsimsellik için olan bu üç düzeltme sabiti ile yapılan düzeltmeler,
Holt'un Dogrusal Yöntemi'ne benzer olarak tek toplam esitlikte
toplanarak mevsimsellik esitligi elde etmektedir.

    \begin{Verbatim}[commandchars=\\\{\}]
{\color{incolor}In [{\color{incolor}35}]:} model14 \PY{o}{\PYZlt{}\PYZhy{}}HoltWinters\PY{p}{(}time\PYZus{}series\PY{p}{)}
         model14
         data\PYZus{}forecast\PYZus{}holtWinter \PY{o}{\PYZlt{}\PYZhy{}} forecast\PY{p}{(}model14\PY{p}{,} h\PY{o}{=}\PY{l+m}{3}\PY{p}{)}
         accuracy\PY{p}{(}data\PYZus{}forecast\PYZus{}holtWinter\PY{p}{)}
         plot\PY{p}{(}data\PYZus{}forecast\PYZus{}holtWinter\PY{p}{,} \PY{p}{,}xlim\PY{o}{=}\PY{k+kt}{c}\PY{p}{(}\PY{l+m}{2018.1}\PY{p}{,}\PY{l+m}{2018.3}\PY{p}{)}\PY{p}{)}
         data\PYZus{}forecast\PYZus{}holtWinter
\end{Verbatim}


    
    \begin{verbatim}
Holt-Winters exponential smoothing with trend and additive seasonal component.

Call:
HoltWinters(x = time_series)

Smoothing parameters:
 alpha: 1
 beta : 0
 gamma: 0

Coefficients:
           [,1]
a    1.93268403
b    0.01312719
s1   0.09744097
s2  -0.11518403
s3  -0.18280903
s4   0.05598264
s5   0.32977431
s6  -0.11422569
s7   0.14994097
s8   0.13856597
s9  -0.09776736
s10 -0.16885069
s11 -0.19818403
s12  0.10531597
    \end{verbatim}

    
    \begin{tabular}{r|lllllll}
  & ME & RMSE & MAE & MPE & MAPE & MASE & ACF1\\
\hline
	Training set & -0.01718173 & 0.6298921   & 0.3989148   & -1.049968   & 8.759911    & 0.31458     & 0.1432632  \\
\end{tabular}


    
    
    \begin{verbatim}
         Point Forecast     Lo 80    Hi 80       Lo 95    Hi 95
Jan 2018       2.043252 1.2348476 2.851657  0.80690399 3.279600
Feb 2018       1.843754 0.7004976 2.987011  0.09529398 3.592215
Mar 2018       1.789257 0.3890587 3.189454 -0.35216134 3.930674
    \end{verbatim}

    
    \begin{center}
    \adjustimage{max size={0.9\linewidth}{0.9\paperheight}}{output_95_3.png}
    \end{center}
    { \hspace*{\fill} \\}
    
    Düzleştirme katsayı değeri arttıkça daha büyük tahmin farklılıkları
oluşmaktadır. Basit üstel düzleştirme yöntemi uyguladığımızda gelen
alpha değeri 1'dir. Bu sonuçta değerin çok büyük olduğunu gösterir. Hata
oranı fazladır. O yüzden öngörü yaparken bu model seçilmemelidir.

    \begin{Verbatim}[commandchars=\\\{\}]
{\color{incolor}In [{\color{incolor}36}]:} \PY{k+kp}{summary}\PY{p}{(}data\PYZus{}forecast\PYZus{}holtWinter\PY{p}{)}
\end{Verbatim}


    \begin{Verbatim}[commandchars=\\\{\}]

Forecast method: HoltWinters

Model Information:
Holt-Winters exponential smoothing with trend and additive seasonal component.

Call:
HoltWinters(x = time\_series)

Smoothing parameters:
 alpha: 1
 beta : 0
 gamma: 0

Coefficients:
           [,1]
a    1.93268403
b    0.01312719
s1   0.09744097
s2  -0.11518403
s3  -0.18280903
s4   0.05598264
s5   0.32977431
s6  -0.11422569
s7   0.14994097
s8   0.13856597
s9  -0.09776736
s10 -0.16885069
s11 -0.19818403
s12  0.10531597

Error measures:
                      ME      RMSE       MAE       MPE     MAPE    MASE
Training set -0.01718173 0.6298921 0.3989148 -1.049968 8.759911 0.31458
                  ACF1
Training set 0.1432632

Forecasts:
         Point Forecast     Lo 80    Hi 80       Lo 95    Hi 95
Jan 2018       2.043252 1.2348476 2.851657  0.80690399 3.279600
Feb 2018       1.843754 0.7004976 2.987011  0.09529398 3.592215
Mar 2018       1.789257 0.3890587 3.189454 -0.35216134 3.930674

    \end{Verbatim}

    \subsubsection{Arima Modeli}\label{arima-modeli}

Zaman serilerinde öngörü analizi yapmanın bir diğer yolu da arima modeli
uygulamaktır.

    \begin{Verbatim}[commandchars=\\\{\}]
{\color{incolor}In [{\color{incolor}37}]:} time\PYZus{}series2\PY{o}{\PYZlt{}\PYZhy{}}ts\PY{p}{(}data\PY{p}{[}\PY{p}{,}\PY{l+m}{2}\PY{p}{]}\PY{p}{,} start\PY{o}{=}\PY{k+kt}{c}\PY{p}{(}\PY{l+m}{1994}\PY{p}{,} \PY{l+m}{1}\PY{p}{)}\PY{p}{,} end\PY{o}{=}\PY{k+kt}{c}\PY{p}{(}\PY{l+m}{2017}\PY{p}{,} \PY{l+m}{12}\PY{p}{)}\PY{p}{,} freq\PY{o}{=}\PY{l+m}{12}\PY{p}{)}
\end{Verbatim}


    \begin{Verbatim}[commandchars=\\\{\}]
{\color{incolor}In [{\color{incolor}38}]:} model21 \PY{o}{\PYZlt{}\PYZhy{}} arima\PY{p}{(}time\PYZus{}series2\PY{p}{,} order\PY{o}{=}\PY{k+kt}{c}\PY{p}{(}\PY{l+m}{1}\PY{p}{,}\PY{l+m}{0}\PY{p}{,}\PY{l+m}{1}\PY{p}{)}\PY{p}{)}
         model21
         data\PYZus{}forecast\PYZus{}arima \PY{o}{\PYZlt{}\PYZhy{}} forecast\PY{p}{(}model21\PY{p}{,} h\PY{o}{=}\PY{l+m}{3}\PY{p}{)}
         accuracy\PY{p}{(}data\PYZus{}forecast\PYZus{}arima\PY{p}{)}
         plot\PY{p}{(}data\PYZus{}forecast\PYZus{}arima\PY{p}{,} \PY{p}{,}xlim\PY{o}{=}\PY{k+kt}{c}\PY{p}{(}\PY{l+m}{2018.1}\PY{p}{,}\PY{l+m}{2018.3}\PY{p}{)}\PY{p}{)}
         data\PYZus{}forecast\PYZus{}arima
\end{Verbatim}


    
    \begin{verbatim}

Call:
arima(x = time_series2, order = c(1, 0, 1))

Coefficients:
         ar1     ma1  intercept
      0.9542  0.1786     4.0934
s.e.  0.0173  0.0550     0.8477

sigma^2 estimated as 0.3541:  log likelihood = -260.53,  aic = 529.07
    \end{verbatim}

    
    \begin{tabular}{r|lllllll}
  & ME & RMSE & MAE & MPE & MAPE & MASE & ACF1\\
\hline
	Training set & 0.006323252 & 0.5950555   & 0.3656777   & -1.272849   & 7.823498    & 0.2883696   & 0.01722624 \\
\end{tabular}


    
    
    \begin{verbatim}
         Point Forecast     Lo 80    Hi 80     Lo 95    Hi 95
Jan 2018       2.096852 1.3342576 2.859446 0.9305645 3.263139
Feb 2018       2.188226 1.0358981 3.340553 0.4258927 3.950558
Mar 2018       2.275417 0.8585829 3.692252 0.1085561 4.442279
    \end{verbatim}

    
    \begin{center}
    \adjustimage{max size={0.9\linewidth}{0.9\paperheight}}{output_100_3.png}
    \end{center}
    { \hspace*{\fill} \\}
    
    \begin{Verbatim}[commandchars=\\\{\}]
{\color{incolor}In [{\color{incolor}39}]:} \PY{k+kp}{summary}\PY{p}{(}data\PYZus{}forecast\PYZus{}arima\PY{p}{)}
\end{Verbatim}


    \begin{Verbatim}[commandchars=\\\{\}]

Forecast method: ARIMA(1,0,1) with non-zero mean

Model Information:

Call:
arima(x = time\_series2, order = c(1, 0, 1))

Coefficients:
         ar1     ma1  intercept
      0.9542  0.1786     4.0934
s.e.  0.0173  0.0550     0.8477

sigma\^{}2 estimated as 0.3541:  log likelihood = -260.53,  aic = 529.07

Error measures:
                      ME      RMSE       MAE       MPE     MAPE      MASE
Training set 0.006323252 0.5950555 0.3656777 -1.272849 7.823498 0.2883696
                   ACF1
Training set 0.01722624

Forecasts:
         Point Forecast     Lo 80    Hi 80     Lo 95    Hi 95
Jan 2018       2.096852 1.3342576 2.859446 0.9305645 3.263139
Feb 2018       2.188226 1.0358981 3.340553 0.4258927 3.950558
Mar 2018       2.275417 0.8585829 3.692252 0.1085561 4.442279

    \end{Verbatim}

    Aşağıda genel olarak bütün doğrulukları göstereceğiz.Hatırlamak için

ME:Ortalama Hata

MAE : Ortalama Mutlak Hata

MSE: Ortalama Hata Kare

MPE: Ortalama Yüzde Hata

MAPE: Ortalama Mutlak Yüzde Hata

    \begin{Verbatim}[commandchars=\\\{\}]
{\color{incolor}In [{\color{incolor}40}]:} accuracy\PY{p}{(}data\PYZus{}forecast\PYZus{}ets\PY{p}{)}
         accuracy\PY{p}{(}data\PYZus{}forecast\PYZus{}simple\PY{p}{)}
         accuracy\PY{p}{(}data\PYZus{}forecast\PYZus{}holt\PY{p}{)}
         accuracy\PY{p}{(}data\PYZus{}forecast\PYZus{}holtWinter\PY{p}{)}
         accuracy\PY{p}{(}data\PYZus{}forecast\PYZus{}arima\PY{p}{)}
\end{Verbatim}


    \begin{tabular}{r|lllllll}
  & ME & RMSE & MAE & MPE & MAPE & MASE & ACF1\\
\hline
	Training set & -0.001786263 & 0.5867771    & 0.3525912    & -0.490289    & 7.254289     & 0.2780497    & 0.1595532   \\
\end{tabular}


    
    \begin{tabular}{r|lllllll}
  & ME & RMSE & MAE & MPE & MAPE & MASE & ACF1\\
\hline
	Training set & -0.002665668 & 0.6110392    & 0.3739591    & -0.6182691   & 7.84757      & 0.2949002    & 0.1759845   \\
\end{tabular}


    
    \begin{tabular}{r|lllllll}
  & ME & RMSE & MAE & MPE & MAPE & MASE & ACF1\\
\hline
	Training set & 0.01639161  & 0.6123117   & 0.3755175   & -0.06358456 & 7.867708    & 0.2961291   & 0.1759017  \\
\end{tabular}


    
    \begin{tabular}{r|lllllll}
  & ME & RMSE & MAE & MPE & MAPE & MASE & ACF1\\
\hline
	Training set & -0.01718173 & 0.6298921   & 0.3989148   & -1.049968   & 8.759911    & 0.31458     & 0.1432632  \\
\end{tabular}


    
    \begin{tabular}{r|lllllll}
  & ME & RMSE & MAE & MPE & MAPE & MASE & ACF1\\
\hline
	Training set & 0.006323252 & 0.5950555   & 0.3656777   & -1.272849   & 7.823498    & 0.2883696   & 0.01722624 \\
\end{tabular}


    
    Yapılan 5 farklı öngörü modelinin hata değerlerini karşılaştırmak için
literatürde en çok kullanılan ve tercih edilen Ortalama Mutlak Hata
(MAE) değerini kullanılırız. Bu karşılaştırmaya göre ets() yöntemi en
uygun yöntemdir.

Not: Hatırlarsak alpha değerleri ets() modülünde diğer modüllere göre daha düşüktür. Yani,ets() modülünde koşullara verdiği
tepki, diğer modellerde verdiği tepkiden daha az, daha tutarlıdır.

    \subsubsection{Natural Gas Futures - Zaman Serilerinde Öngörü Özet}\label{natural-gas-futures---zaman-serilerinde-uxf6nguxf6ruxfc-uxf6zet}

Verimizde öngörü yapabilmek için tahminleme verisi ve test verisi olmak
üzere iki veri seti oluşturduk.Tahminleme verisini Ocak 1994 ile Aralık
2017 arasını , test verisi için son üç ayı yani Ocak 2018 ile Mart 2018
arasında aldık. Sırasıyla ets ,holtwinters ve arima modellerini
uyguladık. Hata terimleri ve alpha değerleri incelendiğinde ets()
modelinin öngörü için daha uygun olduğuna karar verdik.

    İlk verimiz ile ilgili analiz ve öngörü hesaplamalarını yapmış olduk.
Şimdi ikinci verimiz olan petrol verilerini inceleyeceğiz.

    \section{Florida Field Production of Crude
Oil}\label{florida-field-production-of-crude-oil}

Verinin Yayıncısı: U.S. Energy Information Administration Data

ABD ulusal ve devlet verilerinin;elektrik, kömür, doğal gaz ve petrol
gibi tüm ana enerji ürünleri üzerindeki üretim, tüketim ve diğer
göstergelere ilişkisini göstermektedir.

Verinin Adı: Florida Field Production of Crude Oil

Florida'nın ham petrol üretiminin göstergelere ilişkisini
göstermektedir.Veri zaman serisi olarak ilerlemektedir.

Verinin Sıklığı: Aylık

Verinin Zaman Aralığı: 31.01.1981- 28.02.2018

Verinin Alındığı Yer: Quandl

Quandl yatırım profesyonellerine hizmet veren finansal, ekonomik ve
alternatif veriler için bir platformdur.Verileri 500'den fazla
yayıncıdan sağlar. Quandl'ın verilerine bir API aracılığıyla
erişilebilir. Bir çok programlama dili için API erişimi (R, Python,
Matlab, Maple ve Stata) vardır.

Excel eklentisi, stok fiyat bilgisi dahil olmak üzere verilere erişim
sağlar.

    \begin{Verbatim}[commandchars=\\\{\}]
{\color{incolor}In [{\color{incolor}41}]:} data2\PY{o}{\PYZlt{}\PYZhy{}}Quandl\PY{p}{(}\PY{l+s}{\PYZdq{}}\PY{l+s}{EIA/PET\PYZus{}MCRFPFL1\PYZus{}M\PYZdq{}}\PY{p}{,} api\PYZus{}key\PY{o}{=}\PY{l+s}{\PYZdq{}}\PY{l+s}{BtbwkANCR4aHKBscujRb\PYZdq{}}\PY{p}{,} collapse\PY{o}{=}\PY{l+s}{\PYZdq{}}\PY{l+s}{monthly\PYZdq{}}\PY{p}{)}
\end{Verbatim}


    Veriyi quandl veri sitesinden alıyoruz. Veri iki sütundan oluşmaktadır.
İlk sütun tarih ikinci sütun ise üretim miktarıdır.

   Aşağıda verimizi çizgizel grafik
gösterimini görüyoruz.


    \begin{Verbatim}[commandchars=\\\{\}]
{\color{incolor}In [{\color{incolor}42}]:} \PY{k+kp}{rownames}\PY{p}{(}data2\PY{p}{)} \PY{o}{\PYZlt{}\PYZhy{}} data2\PY{p}{[}\PY{p}{,}\PY{l+m}{1}\PY{p}{]}
         plot\PY{p}{(}data2\PY{p}{,}type\PY{o}{=}\PY{l+s}{\PYZdq{}}\PY{l+s}{l\PYZdq{}}\PY{p}{)}
\end{Verbatim}

 

    \begin{center}
    \adjustimage{max size={0.9\linewidth}{0.9\paperheight}}{output_112_0.png}
    \end{center}
    { \hspace*{\fill} \\}
 
    Grafik incelendiğinde petrol üretimi gittikçe
azaldığı fark ediliyor.
 \subsection{Decompose- Geleneksel Zaman Serisi Ayrışım Yöntemleri}  
    Veriyi sezonsal veriden ayırmak için R paketinin decompose fonksiyonunu
kullanıyoruz. Decompose fonksiyonu ile veri sezonsal etkiden ve trend
etkisinden arınmaktadır.Veri aylık olduğundan dolayı sıklığı 12 olarak
alıyoruz.

Ek olarak verinin ayrışma yönteminin, toplamsal ya da çarpımsal olduğunu
anlamak için type= c("additive") ve type= c("multiplicative") fonksiyonu
kullanıyoruz.

Dönemsel mevsimselliği incelemek için ise R 'ın figure fonksiyonu
kullanılmaktadır.

    Aşağıda zaman serisi (ts) verimize toplamsal ayrıştırma tipine decompose
fonsiyonu uyguluyoruz. Böylelikle verimizdeki sezonsal ve trendsel
etkiden çıkartmış oluyoruz. Plot fonksiyonu ile yaptığımız modeli
grafiksel olarak gösteriyoruz. Figure fonksiyonu ile de zaman seriminizn
12 aylık sezonsal etkisini gösteriyoruz.

    \begin{Verbatim}[commandchars=\\\{\}]
{\color{incolor}In [{\color{incolor}43}]:} additive\PYZus{}model\PY{o}{\PYZlt{}\PYZhy{}}decompose\PY{p}{(}ts\PY{p}{(}data2\PY{p}{[}\PY{p}{,}\PY{l+m}{2}\PY{p}{]}\PY{p}{,} start \PY{o}{=} \PY{k+kt}{c}\PY{p}{(}\PY{l+m}{1981}\PY{p}{,} \PY{l+m}{1}\PY{p}{)}\PY{p}{,} end \PY{o}{=} \PY{k+kt}{c}\PY{p}{(}\PY{l+m}{2018}\PY{p}{,} \PY{l+m}{2}\PY{p}{)}\PY{p}{,}freq\PY{o}{=}\PY{l+m}{12}\PY{p}{)}\PY{p}{,}type\PY{o}{=} \PY{k+kt}{c}\PY{p}{(}\PY{l+s}{\PYZdq{}}\PY{l+s}{additive\PYZdq{}}\PY{p}{)}\PY{p}{)}
         \PY{k+kp}{summary}\PY{p}{(}additive\PYZus{}model\PY{p}{)}
         plot\PY{p}{(}additive\PYZus{}model\PY{p}{)}
         plot\PY{p}{(}additive\PYZus{}model\PY{o}{\PYZdl{}}figure\PY{p}{,}type\PY{o}{=}\PY{l+s}{\PYZdq{}}\PY{l+s}{b\PYZdq{}}\PY{p}{)}
\end{Verbatim}


    
    \begin{verbatim}
         Length Class  Mode     
x        446    ts     numeric  
seasonal 446    ts     numeric  
trend    446    ts     numeric  
random   446    ts     numeric  
figure    12    -none- numeric  
type       1    -none- character
    \end{verbatim}

    
    \begin{center}
    \adjustimage{max size={0.9\linewidth}{0.9\paperheight}}{output_115_1.png}
    \end{center}
    { \hspace*{\fill} \\}
    
    \begin{center}
    \adjustimage{max size={0.9\linewidth}{0.9\paperheight}}{output_115_2.png}
    \end{center}
    { \hspace*{\fill} \\}
    
    Toplamsal ayrışma modelini incelediğimizde trend bileşeninin 1990
yılında en düşük etkide bulunduğunu ancek sonrasında sürekli arttığı
gözlenmektedir. Sezonsal etkiyi incelersek veri bütün yıllarda aynı
sezonsal değerleri aldığını görüyoruz.

Figure fonksiyonu ile gelen mevsimsellik etki inclendiğinde veri mart
ayında en yüksek seviyede, ocak ayında ise en düşük seviyede
mevsimsellikten etkilendiğini görebiliyoruz.
\pagebreak  

Aşağıda zaman serisi (ts) verimize çarpımsal ayrıştırma tipine decompose
fonsiyonu uyguluyoruz. Böylelikle verimizdeki sezonsal ve trendsel
etkiden çıkartmış oluyoruz. Plot fonksiyonu ile yaptığımız modeli
grafiksel olarak gösteriyoruz. Figure fonksiyonu ile de zaman seriminizn
12 aylık sezonsal etkisini gösteriyoruz.

    \begin{Verbatim}[commandchars=\\\{\}]
{\color{incolor}In [{\color{incolor}44}]:} multiplicative\PYZus{}model\PY{o}{\PYZlt{}\PYZhy{}}decompose\PY{p}{(}ts\PY{p}{(}data2\PY{p}{[}\PY{p}{,}\PY{l+m}{2}\PY{p}{]}\PY{p}{,} freq\PY{o}{=}\PY{l+m}{12}\PY{p}{)}\PY{p}{,} type\PY{o}{=} \PY{k+kt}{c}\PY{p}{(}\PY{l+s}{\PYZdq{}}\PY{l+s}{multiplicative\PYZdq{}}\PY{p}{)}\PY{p}{)}
         \PY{k+kp}{summary}\PY{p}{(}multiplicative\PYZus{}model\PY{p}{)}
         plot\PY{p}{(}multiplicative\PYZus{}model\PY{p}{)}
         plot\PY{p}{(}multiplicative\PYZus{}model\PY{o}{\PYZdl{}}figure\PY{p}{,} type\PY{o}{=}\PY{l+s}{\PYZdq{}}\PY{l+s}{b\PYZdq{}}\PY{p}{)}
\end{Verbatim}


    
    \begin{verbatim}
         Length Class  Mode     
x        446    ts     numeric  
seasonal 446    ts     numeric  
trend    446    ts     numeric  
random   446    ts     numeric  
figure    12    -none- numeric  
type       1    -none- character
    \end{verbatim}

    
    \begin{center}
    \adjustimage{max size={0.9\linewidth}{0.9\paperheight}}{output_118_1.png}
    \end{center}
    { \hspace*{\fill} \\}
    
    \begin{center}
    \adjustimage{max size={0.9\linewidth}{0.9\paperheight}}{output_118_2.png}
    \end{center}
    { \hspace*{\fill} \\}
    
    Çarpımsal ayrışma modelini incelediğimizde trend bileşeninin 1990
yılında en düşük etkide bulunduğunu ancek sonrasında sürekli arttığı
gözlenmektedir. Sezonsal etkiyi incelersek veri bütün yıllarda aynı
sezonsal değerleri aldığını görüyoruz.

Figure fonksiyonu ile gelen mevsimsellik etki inclendiğinde veri mart
ayında en yüksek seviyede, ocak ayında ise en düşük seviyede
mevsimsellikten etkilendiğini görebiliyoruz.

    Toplamsal ve çarpımsal ayrışma model grafiksel olarak incelendiğinde
random kısımlarında büyük farklılıklar gözlenmektedir. Grafiksel olarak
karar verilirse additive model dahal uygundur. Ancak doğruluğundan emin
olmak için ilerleyen aşamalarda veriye arima modeli uygulayacağız.
\pagebreak
Summary fonksiyonu ile verinin özetini göstermiş olunur.

    Aşağıda asıl verimize, toplamsal ayrıştırma sezonsal ve trendsel etkiden
ayrılmış verimize, çarpımsal ayrıştırma ile sezonsal ve trendsel etkiden
ayrılmış verimizin özetlerini görüyoruz.

    \begin{Verbatim}[commandchars=\\\{\}]
{\color{incolor}In [{\color{incolor}45}]:} \PY{k+kp}{summary}\PY{p}{(}data2\PY{p}{[}\PY{p}{,}\PY{l+m}{2}\PY{p}{]}\PY{p}{)}
         \PY{k+kp}{summary}\PY{p}{(}na.omit\PY{p}{(}additive\PYZus{}model\PY{o}{\PYZdl{}}random\PY{p}{)}\PY{p}{)}
         \PY{k+kp}{summary}\PY{p}{(}na.omit\PY{p}{(}multiplicative\PYZus{}model\PY{o}{\PYZdl{}}random\PY{p}{)}\PY{p}{)}
\end{Verbatim}


    
    \begin{verbatim}
   Min. 1st Qu.  Median    Mean 3rd Qu.    Max. 
   53.0   185.5   390.0   539.2   563.8  3606.0 
    \end{verbatim}

    
    
    \begin{verbatim}
     Min.   1st Qu.    Median      Mean   3rd Qu.      Max. 
-190.9838  -12.0257    0.9057   -1.0926   15.3770  112.8715 
    \end{verbatim}

    
    
    \begin{verbatim}
   Min. 1st Qu.  Median    Mean 3rd Qu.    Max. 
 0.4991  0.9697  1.0046  0.9970  1.0395  1.4679 
    \end{verbatim}

 \subsection{Normallik Testi - Shapiro Wilk Test}   
Verinin normalliğini kontrol etmek için shapiro.test
fonksiyonu kullanılmaktadır.P değeri testen ne kadar emin olduğunu
göstermektedir.P değerinde güven aralığını genelde \%95 olarak alınır,
yani p değeri 0.05'tan küçük ise yaptığımız testen emin olabiliriz. W
değeri 1 değerine ne kadar yakınsa veri normalliğe o derece yakındır.

    Shapiro testini toplumsal ve çarpımsal ayrışma modeline uygulanır.

    \begin{Verbatim}[commandchars=\\\{\}]
{\color{incolor}In [{\color{incolor}46}]:} shapiro.test\PY{p}{(}na.omit\PY{p}{(}additive\PYZus{}model\PY{o}{\PYZdl{}}random\PY{p}{)}\PY{p}{)}
         shapiro.test\PY{p}{(}na.omit\PY{p}{(}multiplicative\PYZus{}model\PY{o}{\PYZdl{}}random\PY{p}{)}\PY{p}{)}
\end{Verbatim}


    
    \begin{verbatim}

	Shapiro-Wilk normality test

data:  na.omit(additive_model$random)
W = 0.90008, p-value = 2.956e-16

    \end{verbatim}

    
    
    \begin{verbatim}

	Shapiro-Wilk normality test

data:  na.omit(multiplicative_model$random)
W = 0.80709, p-value < 2.2e-16

    \end{verbatim}

    
    Toplamsal ve çarpımsal ayrışma modeline uygulanan normallik testi
incellendiğinde p değerlerinin oldukça küçük, neredeyse 0 yakın olduğunu
görüyoruz. Yaptığımız iki testte de oldukça eminiz. W değeri
incelendiğin de ise toplamsal modelin 1 daha yakın olduğu görülmektedir.
Verimiz normaldir, diyebiliriz.
\subsection{Otokorelasyon}
    Zaman serisiyle bu serinin gecikmeli serileri arasındaki ilişkiyi
göstermek için R'ın acf ve pacf fonksiyonunu toplamsal ve çarpımsal
ayrışma modeline uygulanır.

    ACF eğer çok yüksek bir değerden başlayıp çok yavaş küçülüyorsa, bu
serinin durağan olmadığının bir göstergesidir.

    \begin{Verbatim}[commandchars=\\\{\}]
{\color{incolor}In [{\color{incolor}47}]:} acf\PY{p}{(}na.omit\PY{p}{(}additive\PYZus{}model\PY{o}{\PYZdl{}}random\PY{p}{)}\PY{p}{)}
         acf\PY{p}{(}na.omit\PY{p}{(}multiplicative\PYZus{}model\PY{o}{\PYZdl{}}random\PY{p}{)}\PY{p}{)}
\end{Verbatim}


    \begin{center}
    \adjustimage{max size={0.9\linewidth}{0.9\paperheight}}{output_129_0.png}
    \end{center}
    { \hspace*{\fill} \\}
    
    \begin{center}
    \adjustimage{max size={0.9\linewidth}{0.9\paperheight}}{output_129_1.png}
    \end{center}
    { \hspace*{\fill} \\}
    
    Toplamsal ve çarpımsal ayrıştırma modelleri için acf değerlerini
incelediğimizde iki durumunda durağan halde olduğunu söyleyebiliriz.

    \begin{Verbatim}[commandchars=\\\{\}]
{\color{incolor}In [{\color{incolor}48}]:} pacf\PY{p}{(}na.omit\PY{p}{(}additive\PYZus{}model\PY{o}{\PYZdl{}}random\PY{p}{)}\PY{p}{)}
         pacf\PY{p}{(}na.omit\PY{p}{(}multiplicative\PYZus{}model\PY{o}{\PYZdl{}}random\PY{p}{)}\PY{p}{)}
\end{Verbatim}


    \begin{center}
    \adjustimage{max size={0.9\linewidth}{0.9\paperheight}}{output_131_0.png}
    \end{center}
    { \hspace*{\fill} \\}
    
    \begin{center}
    \adjustimage{max size={0.9\linewidth}{0.9\paperheight}}{output_131_1.png}
    \end{center}
    { \hspace*{\fill} \\}
\pagebreak      
    \begin{Verbatim}[commandchars=\\\{\}]
{\color{incolor}In [{\color{incolor}49}]:} hist\PY{p}{(}na.omit\PY{p}{(}additive\PYZus{}model\PY{o}{\PYZdl{}}random\PY{p}{)}\PY{p}{)}
\end{Verbatim}


    \begin{center}
    \adjustimage{max size={0.9\linewidth}{0.9\paperheight}}{output_132_0.png}
    \end{center}
    { \hspace*{\fill} \\}
\pagebreak  
 \subsection{Durağanlık Analizi}   
    Veride durağanlık durumunu belirlemek için sırasıyla ;Augmented
Dickey-Fuller Test- Kwiatkowski-Phillips-Schmidt-Shin (KPSS) Test -
Phillips--Perron Birim Kök Testi uygulanır.

    \paragraph{Augmented Dickey Fuller
Test}\label{augmented-dickey-fuller-test}

İlk olarak Augmented Dickey Fuller testini verinin bileşenlerden
ayrılmamış haline, toplamsal ve çarpımsal ayrışma modellerine sırasıyla
uyguluyarak verinin durağanlığını kontrol ediyoruz. ADF testini R içinde
bulunan adf.test fonksiyonu ile elde edebiliriz.

    \begin{Verbatim}[commandchars=\\\{\}]
{\color{incolor}In [{\color{incolor}50}]:} adf.test\PY{p}{(}data2\PY{p}{[}\PY{p}{,}\PY{l+m}{2}\PY{p}{]}\PY{p}{)}
         adf.test\PY{p}{(}na.omit\PY{p}{(}additive\PYZus{}model\PY{o}{\PYZdl{}}random\PY{p}{)}\PY{p}{)}
         adf.test\PY{p}{(}na.omit\PY{p}{(}multiplicative\PYZus{}model\PY{o}{\PYZdl{}}random\PY{p}{)}\PY{p}{)}
\end{Verbatim}


    \begin{Verbatim}[commandchars=\\\{\}]
Warning message in adf.test(data2[, 2]):
"p-value greater than printed p-value"
    \end{Verbatim}

    
    \begin{verbatim}

	Augmented Dickey-Fuller Test

data:  data2[, 2]
Dickey-Fuller = 6.9986, Lag order = 7, p-value = 0.99
alternative hypothesis: stationary

    \end{verbatim}

    
    \begin{Verbatim}[commandchars=\\\{\}]
Warning message in adf.test(na.omit(additive\_model\$random)):
"p-value smaller than printed p-value"
    \end{Verbatim}

    
    \begin{verbatim}

	Augmented Dickey-Fuller Test

data:  na.omit(additive_model$random)
Dickey-Fuller = -9.7154, Lag order = 7, p-value = 0.01
alternative hypothesis: stationary

    \end{verbatim}

    
    \begin{Verbatim}[commandchars=\\\{\}]
Warning message in adf.test(na.omit(multiplicative\_model\$random)):
"p-value smaller than printed p-value"
    \end{Verbatim}

    
    \begin{verbatim}

	Augmented Dickey-Fuller Test

data:  na.omit(multiplicative_model$random)
Dickey-Fuller = -8.2286, Lag order = 7, p-value = 0.01
alternative hypothesis: stationary

    \end{verbatim}

    
    Yapılan ADF testteki sonuçlar: Additive ve multiplicative modeline göre
veriler güven aralığı (\(p<0.05\)) içerisindedir, yani veri durağandır.

    \subparagraph{Kwiatkowski-Phillips-Schmidt-Shin (KPSS)
Test}\label{kwiatkowski-phillips-schmidt-shin-kpss-test}

İlk olarak Kwiatkowski-Phillips-Schmidt-Shin (KPSS) Test testini verinin
bileşenlerden ayrılmamış haline, toplamsal ve çarpımsal ayrışma
modellerine sırasıyla uyguluyarak verinin durağanlığını kontrol
ediyoruz. KPSS testini R içinde bulunan kpss.test fonksiyonu ile elde
edebiliriz.

    \begin{Verbatim}[commandchars=\\\{\}]
{\color{incolor}In [{\color{incolor}51}]:} kpss.test\PY{p}{(}data2\PY{p}{[}\PY{p}{,}\PY{l+m}{2}\PY{p}{]}\PY{p}{,} null\PY{o}{=}\PY{l+s}{\PYZdq{}}\PY{l+s}{Trend\PYZdq{}}\PY{p}{)}
         kpss.test\PY{p}{(}na.omit\PY{p}{(}additive\PYZus{}model\PY{o}{\PYZdl{}}random\PY{p}{)}\PY{p}{,} null\PY{o}{=}\PY{l+s}{\PYZdq{}}\PY{l+s}{Trend\PYZdq{}}\PY{p}{)}
         kpss.test\PY{p}{(}na.omit\PY{p}{(}multiplicative\PYZus{}model\PY{o}{\PYZdl{}}random\PY{p}{)}\PY{p}{,} null\PY{o}{=}\PY{l+s}{\PYZdq{}}\PY{l+s}{Trend\PYZdq{}}\PY{p}{)}
\end{Verbatim}


    \begin{Verbatim}[commandchars=\\\{\}]
Warning message in kpss.test(data2[, 2], null = "Trend"):
"p-value smaller than printed p-value"
    \end{Verbatim}

    
    \begin{verbatim}

	KPSS Test for Trend Stationarity

data:  data2[, 2]
KPSS Trend = 1.0851, Truncation lag parameter = 4, p-value = 0.01

    \end{verbatim}

    
    \begin{Verbatim}[commandchars=\\\{\}]
Warning message in kpss.test(na.omit(additive\_model\$random), null = "Trend"):
"p-value greater than printed p-value"
    \end{Verbatim}

    
    \begin{verbatim}

	KPSS Test for Trend Stationarity

data:  na.omit(additive_model$random)
KPSS Trend = 0.036044, Truncation lag parameter = 4, p-value = 0.1

    \end{verbatim}

    
    \begin{Verbatim}[commandchars=\\\{\}]
Warning message in kpss.test(na.omit(multiplicative\_model\$random), null = "Trend"):
"p-value greater than printed p-value"
    \end{Verbatim}

    
    \begin{verbatim}

	KPSS Test for Trend Stationarity

data:  na.omit(multiplicative_model$random)
KPSS Trend = 0.016544, Truncation lag parameter = 4, p-value = 0.1

    \end{verbatim}

    
    Yapılan KPSS testteki sonuçlar:

Additive ve multiplicative modeline göre veriler güven aralığı
(\(p>0.05\)) içerisindedir, yani veri durağandır.

    \subparagraph{Phillips--Perron Birim Kök
Testi}\label{phillipsperron-birim-kuxf6k-testi}
\pagebreak
Phillips--Perron Birim Kök Testi testini verinin bileşenlerden
ayrılmamış haline, toplamsal ve çarpımsal ayrışma modellerine sırasıyla
uyguluyarak verinin durağanlığını kontrol ediyoruz. Phillips--Perron
Birim Kök testini R içinde bulunan pp.test fonksiyonu ile elde
edebiliriz.

    \begin{Verbatim}[commandchars=\\\{\}]
{\color{incolor}In [{\color{incolor}52}]:} pp.test\PY{p}{(}data2\PY{p}{[}\PY{p}{,}\PY{l+m}{2}\PY{p}{]}\PY{p}{)}
         pp.test\PY{p}{(}na.omit\PY{p}{(}additive\PYZus{}model\PY{o}{\PYZdl{}}random\PY{p}{)}\PY{p}{)}
         pp.test\PY{p}{(}na.omit\PY{p}{(}multiplicative\PYZus{}model\PY{o}{\PYZdl{}}random\PY{p}{)}\PY{p}{)}
\end{Verbatim}


    \begin{Verbatim}[commandchars=\\\{\}]
Warning message in pp.test(data2[, 2]):
"p-value greater than printed p-value"
    \end{Verbatim}

    
    \begin{verbatim}

	Phillips-Perron Unit Root Test

data:  data2[, 2]
Dickey-Fuller Z(alpha) = 17.551, Truncation lag parameter = 5, p-value
= 0.99
alternative hypothesis: stationary

    \end{verbatim}

    
    \begin{Verbatim}[commandchars=\\\{\}]
Warning message in pp.test(na.omit(additive\_model\$random)):
"p-value smaller than printed p-value"
    \end{Verbatim}

    
    \begin{verbatim}

	Phillips-Perron Unit Root Test

data:  na.omit(additive_model$random)
Dickey-Fuller Z(alpha) = -280.28, Truncation lag parameter = 5, p-value
= 0.01
alternative hypothesis: stationary

    \end{verbatim}

    
    \begin{Verbatim}[commandchars=\\\{\}]
Warning message in pp.test(na.omit(multiplicative\_model\$random)):
"p-value smaller than printed p-value"
    \end{Verbatim}

    
    \begin{verbatim}

	Phillips-Perron Unit Root Test

data:  na.omit(multiplicative_model$random)
Dickey-Fuller Z(alpha) = -196.72, Truncation lag parameter = 5, p-value
= 0.01
alternative hypothesis: stationary

    \end{verbatim}

    
    Yapılan Phillips-Person testteki sonuçlar:

Additive ve multiplicative modeline göre veriler güven aralığı
(\(p<0.05\)) içerisindedir, yani veri durağandır.

    \subsection{Arima}\label{arima}

    R'da olan auto.arima fonksiyonu ile verinin kalan sezonsal, trend ve
diğer etkenleri kontrol eder ve onlardan arındırır. Veriye auto.arima
uygulandığında AIC , BIC ve AICc değerlerinin en düşük olanı analiz için
daha kabul edilebilir bir değer almaktadır. ARIMA için p,d,q
fonksiyonları ile verinin karmaşıklığı fark edilir ve en sade , en kolay
model seçilir.

    \begin{Verbatim}[commandchars=\\\{\}]
{\color{incolor}In [{\color{incolor}53}]:} auto.arima\PY{p}{(}ts\PY{p}{(}data2\PY{p}{[}\PY{p}{,}\PY{l+m}{2}\PY{p}{]}\PY{p}{)}\PY{p}{)}
         auto.arima\PY{p}{(}ts\PY{p}{(}na.omit\PY{p}{(}additive\PYZus{}model\PY{o}{\PYZdl{}}random\PY{p}{)}\PY{p}{,} start\PY{o}{=}\PY{k+kt}{c}\PY{p}{(}\PY{l+m}{1981}\PY{p}{,} \PY{l+m}{1}\PY{p}{)}\PY{p}{,} end\PY{o}{=}\PY{k+kt}{c}\PY{p}{(}\PY{l+m}{2018}\PY{p}{,} \PY{l+m}{1}\PY{p}{)}\PY{p}{,} freq\PY{o}{=}\PY{l+m}{12}\PY{p}{)}\PY{p}{)}
         auto.arima\PY{p}{(}ts\PY{p}{(}na.omit\PY{p}{(}multiplicative\PYZus{}model\PY{o}{\PYZdl{}}random\PY{p}{)}\PY{p}{,} start\PY{o}{=}\PY{k+kt}{c}\PY{p}{(}\PY{l+m}{1981}\PY{p}{,} \PY{l+m}{1}\PY{p}{)}\PY{p}{,} end\PY{o}{=}\PY{k+kt}{c}\PY{p}{(}\PY{l+m}{2018}\PY{p}{,} \PY{l+m}{1}\PY{p}{)}\PY{p}{,} freq\PY{o}{=}\PY{l+m}{12}\PY{p}{)}\PY{p}{)}
\end{Verbatim}


    
    \begin{verbatim}
Series: ts(data2[, 2]) 
ARIMA(0,2,2) 

Coefficients:
          ma1     ma2
      -1.4603  0.5198
s.e.   0.0439  0.0435

sigma^2 estimated as 2159:  log likelihood=-2335.03
AIC=4676.06   AICc=4676.11   BIC=4688.35
    \end{verbatim}

    
    
    \begin{verbatim}
Series: ts(na.omit(additive_model$random), start = c(1981, 1), end = c(2018,      1), freq = 12) 
ARIMA(2,0,1) with non-zero mean 

Coefficients:
         ar1      ar2      ma1     mean
      1.0579  -0.3104  -0.8928  -1.0126
s.e.  0.0519   0.0456   0.0314   0.6058

sigma^2 estimated as 885.3:  log likelihood=-2139.52
AIC=4289.03   AICc=4289.17   BIC=4309.52
    \end{verbatim}

    
    
    \begin{verbatim}
Series: ts(na.omit(multiplicative_model$random), start = c(1981, 1),      end = c(2018, 1), freq = 12) 
ARIMA(2,0,3)(0,0,1)[12] with non-zero mean 

Coefficients:
         ar1      ar2      ma1     ma2     ma3     sma1    mean
      1.5297  -0.8885  -1.0847  0.3008  0.2076  -0.1940  0.9973
s.e.  0.0499   0.0461   0.0736  0.0774  0.0615   0.0509  0.0034

sigma^2 estimated as 0.005781:  log likelihood=518.12
AIC=-1020.24   AICc=-1019.91   BIC=-987.46
    \end{verbatim}

    
    Veriye, son(şubat ayı) verisi eklenmeden önce arima modeli uyguladığımda
sonuç aşağıdaki gibi olmuştur.

Verinin hiçbir bileşenden ayrılmadığı durumda auto.arima uygulandığında
p =0 , d= 2 ve q=2 çıkmaktadır. d=2 olması verinin durağan olmadığını
gösterir.

Verinin toplamsal ayrışma modeline auto.arima uygulandığında p= 2 , d=0
ve q=1 çıkmaktadır. d=0 veri durağandır.

Verilerde çarpımsal modele auto.arima uygulandığında p= 2 , d=0 ve q=3
çıkmaktadır. d=0 veri durağandır.

Hangi ayrışma modelini seçmemiz gerektiğine karar vermek için ilk olarak
p,d,q değerlerini karşılaştırmak gerekmektedir. Verinin durağan modeli
seçilmesi gerektiğinden hiçbir bileşenden ayrılmadığı model elenir.
Toplamsal ve çarpımsal model karşılaştırıldığında toplamsal modelin daha
basit olduğu görülür. İkinci olarak ( Seçtiğimiz modelin doğruluğundan
emin olmak için ) auto.arima nın bize vermiş olduğu AIC,AICc ve BIC
değerlerinin en küçük olanı seçilmesi gerekmektedir. Ayrışma modelleri
için incelendiğinde toplamsal modelin daha uygun olduğu görülmektedir.

Formülü :

\[ \hat Y_{d_t} =1.0579 Y_{t-1} -0.3104 Y_{t-2}+E\]

d=0 olduğunda orijinal seri zaten durağandır ve farkının alınmasına da
gerek yoktur.

    Aşağıda her ay güncellenen verimizde şubat ayı eklendiği şekline arima
modeli uygulayacağız.

    \begin{Verbatim}[commandchars=\\\{\}]
{\color{incolor}In [{\color{incolor}54}]:} auto.arima\PY{p}{(}ts\PY{p}{(}data2\PY{p}{[}\PY{p}{,}\PY{l+m}{2}\PY{p}{]}\PY{p}{)}\PY{p}{)}
         auto.arima\PY{p}{(}ts\PY{p}{(}na.omit\PY{p}{(}additive\PYZus{}model\PY{o}{\PYZdl{}}random\PY{p}{)}\PY{p}{)}\PY{p}{)}
         auto.arima\PY{p}{(}ts\PY{p}{(}na.omit\PY{p}{(}multiplicative\PYZus{}model\PY{o}{\PYZdl{}}random\PY{p}{)}\PY{p}{)}\PY{p}{)}
\end{Verbatim}


    
    \begin{verbatim}
Series: ts(data2[, 2]) 
ARIMA(0,2,2) 

Coefficients:
          ma1     ma2
      -1.4603  0.5198
s.e.   0.0439  0.0435

sigma^2 estimated as 2159:  log likelihood=-2335.03
AIC=4676.06   AICc=4676.11   BIC=4688.35
    \end{verbatim}

    
    
    \begin{verbatim}
Series: ts(na.omit(additive_model$random)) 
ARIMA(2,0,1) with non-zero mean 

Coefficients:
         ar1      ar2      ma1     mean
      1.0680  -0.3148  -0.9066  -0.8773
s.e.  0.0516   0.0466   0.0291   0.5558

sigma^2 estimated as 902.4:  log likelihood=-2090.78
AIC=4191.55   AICc=4191.69   BIC=4211.92
    \end{verbatim}

    
    
    \begin{verbatim}
Series: ts(na.omit(multiplicative_model$random)) 
ARIMA(2,0,1) with non-zero mean 

Coefficients:
         ar1      ar2      ma1    mean
      1.3402  -0.5220  -0.8800  0.9967
s.e.  0.0704   0.0408   0.0793  0.0025

sigma^2 estimated as 0.006167:  log likelihood=490.07
AIC=-970.14   AICc=-970   BIC=-949.78
    \end{verbatim}

    
    Ancak her ay güncellenen veride, yani şubat ayı eklenen veriye arima
modeli uygulandığında sonuç aşağıdaki gibi olmuştur.

Verinin hiçbir bileşenden ayrılmadığı durumda auto.arima uygulandığında
p =0 , d= 2 ve q=2 çıkmaktadır. d=2 olması verinin durağan olmadığını
gösterir.

Verinin toplamsal ayrışma modeline auto.arima uygulandığında p= 2 , d=0
ve q=1 çıkmaktadır. d=0 veri durağandır.

Verilerde çarpımsal modele auto.arima uygulandığında p= 2 , d=0 ve q=1
çıkmaktadır. d=0 veri durağandır.

Hangi ayrışma modelini seçmemiz gerektiğine karar vermek için ilk olarak
p,d,q değerlerini karşılaştırmak gerekmektedir. Verinin durağan modeli
seçilmesi gerektiğinden hiçbir bileşenden ayrılmadığı model elenir.
Toplamsal ve çarpımsal model karşılaştırıldığında toplamsal modelin eşit
olduğu görülür. İkinci olarak ( Seçtiğimiz modelin doğruluğundan emin
olmak için ) auto.arima nın bize vermiş olduğu AIC,AICc ve BIC
değerlerinin en küçük olanı seçilmesi gerekmektedir.

Geri sapmalardan en düşük olanını yani çarpımsal modeli seçersek:

Formülü :

\[ \hat Y_{d_t} =1.3402 Y_{t-1}+-0.5220 Y_{t-2} +  E\]

d=0 olduğunda orijinal seri zaten durağandır ve farkının alınmasına da
gerek yoktur.

    Bu durumda iki arima modelini karşılaştırırsak anlayacağımız, veride bir
değerin (bir aylık verinin) dahi modeli değiştirebildiğini görmüş
bulunmaktayız.

    Veride sezonsal etkinin olup olmadığını kontrol etmek için veriden
hiçbir bileşenin çıkarılmadığı haline auto.arima uygulanır.Bunun için
arimanın argümanlarından sseasonal.test'ten yararlanılır.

    \begin{Verbatim}[commandchars=\\\{\}]
{\color{incolor}In [{\color{incolor}55}]:} auto.arima\PY{p}{(}ts\PY{p}{(}data2\PY{p}{[}\PY{p}{,}\PY{l+m}{2}\PY{p}{]}\PY{p}{)}\PY{p}{)}
         auto.arima\PY{p}{(}ts\PY{p}{(}data2\PY{p}{[}\PY{p}{,}\PY{l+m}{2}\PY{p}{]}\PY{p}{)} \PY{p}{,}seasonal.test \PY{o}{=} \PY{k+kt}{c}\PY{p}{(}\PY{l+s}{\PYZdq{}}\PY{l+s}{ocsb\PYZdq{}}\PY{p}{,} \PY{l+s}{\PYZdq{}}\PY{l+s}{ch\PYZdq{}}\PY{p}{)}\PY{p}{)}
\end{Verbatim}


    
    \begin{verbatim}
Series: ts(data2[, 2]) 
ARIMA(0,2,2) 

Coefficients:
          ma1     ma2
      -1.4603  0.5198
s.e.   0.0439  0.0435

sigma^2 estimated as 2159:  log likelihood=-2335.03
AIC=4676.06   AICc=4676.11   BIC=4688.35
    \end{verbatim}

    
    
    \begin{verbatim}
Series: ts(data2[, 2]) 
ARIMA(0,2,2) 

Coefficients:
          ma1     ma2
      -1.4603  0.5198
s.e.   0.0439  0.0435

sigma^2 estimated as 2159:  log likelihood=-2335.03
AIC=4676.06   AICc=4676.11   BIC=4688.35
    \end{verbatim}

    
    Herhangi bir işlem yapılmayan veriye auto.arima ile sezonsal test
uygulandığında ARIMA modelinde; p,d,q değerlerinde ve AIC , AICc ,BIC
değerlerinde herhangi bir farklılık görünmemektedir. Bir farklı ifade
ile veride sezonsal etki bulunmamaktadır.

    Normallik testini seçtiğimiz toplamsal modele tekrar uygulanır.

    \begin{Verbatim}[commandchars=\\\{\}]
{\color{incolor}In [{\color{incolor}56}]:} white\PYZus{}noise2\PY{o}{\PYZlt{}\PYZhy{}}arima\PY{p}{(}ts\PY{p}{(}na.omit\PY{p}{(}additive\PYZus{}model\PY{o}{\PYZdl{}}random\PY{p}{)}\PY{p}{)}\PY{p}{,} order\PY{o}{=}\PY{k+kt}{c}\PY{p}{(}\PY{l+m}{2}\PY{p}{,}\PY{l+m}{0}\PY{p}{,}\PY{l+m}{1}\PY{p}{)}\PY{p}{)}
         shapiro.test\PY{p}{(}white\PYZus{}noise2\PY{o}{\PYZdl{}}residuals\PY{p}{)}
\end{Verbatim}


    
    \begin{verbatim}

	Shapiro-Wilk normality test

data:  white_noise2$residuals
W = 0.87153, p-value < 2.2e-16

    \end{verbatim}

    
    Normallik testine göre p değeri oldukça düşük ve w değeri yüksektir.
Artık verinin beyaz gürültü olduğunu ve veride herhangi bir bileşen
kalmadığını söylenebilir.
\pagebreak
\subsection{Zaman Serilerinin Mevsimsellikten Arındırılması (de-seasonalize)}
 
Zaman serisinin mevsimsellik deseni hakkında fikir verir ve mevsimsel
etkiler olmadan verilerin modellenmesine yardımcı olur. Zaman serisini
mevsimsellikten arındırmak için aşağıdaki iki adım kullanılabilir. 1.
Adım: stl() fonksiyonu ile zaman serisi ayrılır. 2. Adım: forecast
kütüphanesi içerisindeki seasadj() fonksiyonu kullanılır.

    \begin{Verbatim}[commandchars=\\\{\}]
{\color{incolor}In [{\color{incolor}57}]:} datastl2\PY{o}{\PYZlt{}\PYZhy{}} stl\PY{p}{(}ts\PY{p}{(}data2\PY{p}{[}\PY{p}{,}\PY{l+m}{2}\PY{p}{]}\PY{p}{,} freq\PY{o}{=}\PY{l+m}{12}\PY{p}{)}\PY{p}{,} s.window\PY{o}{=}\PY{l+s}{\PYZdq{}}\PY{l+s}{periodic\PYZdq{}}\PY{p}{)}
         \PY{k+kp}{summary}\PY{p}{(}datastl2\PY{p}{)}
         plot\PY{p}{(}datastl2\PY{p}{)}
\end{Verbatim}


    \begin{Verbatim}[commandchars=\\\{\}]
 Call:
 stl(x = ts(data2[, 2], freq = 12), s.window = "periodic")

 Time.series components:
    seasonal             trend            remainder         
 Min.   :-34.17510   Min.   :  63.888   Min.   :-183.62055  
 1st Qu.: -9.65014   1st Qu.: 182.538   1st Qu.: -12.84803  
 Median : -1.33071   Median : 392.483   Median :   0.94450  
 Mean   : -0.03514   Mean   : 540.407   Mean   :  -1.18602  
 3rd Qu.: 13.54174   3rd Qu.: 545.367   3rd Qu.:  15.28002  
 Max.   : 18.50047   Max.   :3368.702   Max.   : 218.79751  
 IQR:
     STL.seasonal STL.trend STL.remainder data  
      23.19       362.83     28.13        378.25
   \%   6.1         95.9       7.4         100.0 

 Weights: all == 1

 Other components: List of 5
 \$ win  : Named num [1:3] 4461 19 13
 \$ deg  : Named int [1:3] 0 1 1
 \$ jump : Named num [1:3] 447 2 2
 \$ inner: int 2
 \$ outer: int 0

    \end{Verbatim}

    \begin{center}
    \adjustimage{max size={0.9\linewidth}{0.9\paperheight}}{output_158_1.png}
    \end{center}
    { \hspace*{\fill} \\}
    
    \begin{Verbatim}[commandchars=\\\{\}]
{\color{incolor}In [{\color{incolor}58}]:} ts\PYZus{}sa2 \PY{o}{\PYZlt{}\PYZhy{}} seasadj\PY{p}{(}datastl2\PY{p}{)}
\end{Verbatim}


    \begin{Verbatim}[commandchars=\\\{\}]
{\color{incolor}In [{\color{incolor}59}]:} par\PY{p}{(}mfrow\PY{o}{=}\PY{k+kt}{c}\PY{p}{(}\PY{l+m}{3}\PY{p}{,}\PY{l+m}{2}\PY{p}{)}\PY{p}{)}
         plot\PY{p}{(}data2\PY{p}{[}\PY{p}{,}\PY{l+m}{2}\PY{p}{]}\PY{p}{,} type\PY{o}{=}\PY{l+s}{\PYZdq{}}\PY{l+s}{l\PYZdq{}}\PY{p}{,} main\PY{o}{=}\PY{l+s}{\PYZdq{}}\PY{l+s}{Real\PYZdq{}}\PY{p}{)}  
         plot\PY{p}{(}ts\PYZus{}sa2\PY{p}{,} type\PY{o}{=}\PY{l+s}{\PYZdq{}}\PY{l+s}{l\PYZdq{}}\PY{p}{,} main\PY{o}{=}\PY{l+s}{\PYZdq{}}\PY{l+s}{Seasonal Adjusted\PYZdq{}}\PY{p}{)}
\end{Verbatim}


    \begin{center}
    \adjustimage{max size={0.9\linewidth}{0.9\paperheight}}{output_160_0.png}
    \end{center}
    { \hspace*{\fill} \\}
    
    Normallik testini ölçmek için sezonsal etkisi çıkarılmış veriye
shapiro.test uygulanır.

    \begin{Verbatim}[commandchars=\\\{\}]
{\color{incolor}In [{\color{incolor}60}]:} shapiro.test\PY{p}{(}ts\PYZus{}sa2\PY{p}{)}
\end{Verbatim}


    
    \begin{verbatim}

	Shapiro-Wilk normality test

data:  ts_sa2
W = 0.66646, p-value < 2.2e-16

    \end{verbatim}

    
    Durağanlık testine göre w değeri 1'den oldukça düşüktür. Bu yüzden ilk
yol olarak yapılan çarpımsal modele ilk başta decompose ve sonrasında da
auto.arima modelinin uygulandığı analiz yöntemi tercih edilir.

    \begin{Verbatim}[commandchars=\\\{\}]
{\color{incolor}In [{\color{incolor}61}]:} acf\PY{p}{(}ts\PYZus{}sa2\PY{p}{)}
\end{Verbatim}


    \begin{center}
    \adjustimage{max size={0.9\linewidth}{0.9\paperheight}}{output_164_0.png}
    \end{center}
    { \hspace*{\fill} \\}
    
    Otokorelasyon katsayısı incelendiğinde verinin yavaş yavaş azaldığı
gözlenmektedir. Bu durumda ts\_sa mevsimsel etkiden arındırılmış zaman
serimizin durağan olmadığını göstermektedir.
\pagebreak  
    \begin{Verbatim}[commandchars=\\\{\}]
{\color{incolor}In [{\color{incolor}62}]:} pacf\PY{p}{(}ts\PYZus{}sa2\PY{p}{)}
\end{Verbatim}


    \begin{center}
    \adjustimage{max size={0.9\linewidth}{0.9\paperheight}}{output_166_0.png}
    \end{center}
    { \hspace*{\fill} \\}
    
    Sezonsal etkiden arındırılan verinin (ts\_sa2) , hala sezonsal etki
içerip içermediğini anlamak (mevsimsellik farkını bulmak) için nsdiffs
fonksiyonu uygulanır.

    \begin{Verbatim}[commandchars=\\\{\}]
{\color{incolor}In [{\color{incolor}63}]:} nsdiffs\PY{p}{(}ts\PYZus{}sa2\PY{p}{)}
\end{Verbatim}


    0

    
    Mevsimsellik farkı sıfırdır. Mevsimsel etki yoktur.
\pagebreak  
    \subsection{Florida Field Production of Crude Oil- Zaman Serisi Analizi Özet}\label{florida-field-production-of-crude-oil--zaman-serisi-analizi-uxf6zet}

Verinin Adı: Florida Field Production of Crude Oil

Verinin Alındığı Yer: Quandl

Verinin Sıklığı: Aylık

Verinin Zaman Aralığı: 31.01.1981- 28.02.2018

Veride iki sütun bulunmaktadır. Sütunlardan birisi zaman, diğeri ise
kullanılan değerdir. Amaç zaman serisindeki gözlenen değerlerinin trend,
mevsimsel, çevresel ve rastgele etkenlerden ayırmak ve doğru modeli
seçmektir. Bu doğrultuda iki yöntem kullandık.

\begin{enumerate}
\def\labelenumi{\arabic{enumi}.}
\item
  Yöntem : R'da bulunan decompose fonksiyonu ile seriyi mevsimsel ve
  trendsel etkenlerden ayırmaktır. Decompose fonksiyonunu zaman
  serilerin hep toplamına hem de çarpımına ayrı ayrı uyguladık. Daha net
  sonuç veren veri seçilir.Toplamsal ayrışma modelini incelediğimizde
  trendsel etkinin 1990 yıllarında en düşük etkiye sahip olduğunu ancak
  sonrasında sürekli arttığını ve mevsimsel etkinin aynı olduğunu
  görüyoruz. Çarpımsal model de aynı sonuca varıyoruz. İki ayrıştırma
  modeli arasındaki fark random bileşende ortaya çıkmaktadır.Ancak
  kullanılan veride toplanan ve çarpılan veride net bir karar
  veremiyoruz. Verimizin zaman bileşenlerinden ayırldıktan sonraki
  normallik durumunu kontrol etmek için veriye normallik
  testi(shapiro.test )uyguladık, iki modelin de p değerleri güven
  aralığında (\(p<0.05\) ) çıktı. W değerleri karşılaştırıldığında ise
  toplamsal ayrışma modelinin daha iyi sonuç verdiğini görüyoruz.Sonuç
  olarak iki model içinde verimizin bileşenlerden ayrılmış haline normal
  diyebiliriz. Zaman bileşenlerinden ayrılmış olan verimizin
  durağanlığını kontrol etmek için adf,kpss ve pp testlerini uyguladık.
  Üç testin sonucu da verimizin toplamsal ve çarpımsal model için
  durağan olduğunu gösterdi. Sonraki aşamalarda decompose fonksiyonu ile
  veremediğimiz hangi modelin uygun olduğu kararını anlamak için arima
  testini uyguladık. Veriye bir aylık (şubat ayı) eklenmeden önceki
  durumda arima vurduğumuzda; Arima ile gelen p,q,d değerlerinden en
  basit olanını ve AIC, AICc ve BIC değerlerinden en düşük olanını yani
  toplamsal modele daha uygun olduğuna karar verdik. Ancak şubat ayı da
  eklenen veride p,q,d değerlerinin toplamsal ve çarpımsal model için
  aynı olduğunu ,ama AIC , AICc ve BIC değerlerinin çarpımsal modelde
  daha düşük olduğunu gözlemledik. Burdan verinin bir aylık değerinin
  arima modellemesinde büyük bir etki sahibi olduğunu fark ettik. Ek
  olarak herhangi bir bileşenden ayrılmamış veriye arima ile sezonsal
  etkinin olup olmadığını ölçtük. Sonucunda ise herhangi bir farklılık
  olmadığını yani verimizde sezonsal bir etkinin olmadığını
  söyleyebiliriz. Decompose ve sonrasında uygulanan auto.arima ile kalan
  verimize normallik testini tekrar uygularsak güven aralığını
  sağladığını ( \( p=2.2e-16 \textless{}0.05 \)) ve w= 0.87153 yani 1
  oldukça yakın olduğunu görüyoruz. Bu da verimizin normal olduğunu
  söyler. Ek olarak artık verimize beyaz gürültü verisi diyebiliriz.
\item
  Yöntem :Zaman serisini mevsimsel bileşenlerinden ayırmak için R'da
  bulunan stl fonksiyonunu kullandık. Verilerin mevsimsel etkisinden
  ayrıldığı durumda normallik değerlerini ölçmek için normallik
  (shapiro.test) testini uyguladık.Sonucunda :\(p=2.2e-16 \textless{}0.05 \) ve w=0.66646 değerlerini elde ettik. Ancak w
  değerleri 1'değerinden oldukça düşüktür.Veri için normal dağılmamıştır
  diyeebiliriz. Ek olarak mevsimsellik etkisinden arındırılmış veri için
  acf testi uygulandığında verinin yavaş yavaş azaldığı gözlenmektedir.
  Bu durum için verinin normal olmadığı söylenmektedir.
\end{enumerate}

Ek olarak uyguladığımız nsdiffs fonksiyonu ile birlikte verinin
mevsimsel etki içerip içermediğine, ya da kalıp kalmadığına karar
verilmektedir. Fonksiyonu uyguladığımızda görülüyor ki veri mevsimsel
bileşen içermemektedir.

Yukarıdaki üç yöntemi karşılatırdığımızda (normalliklerini) kullanmamız
gereken modelin çarpımsal ve kullanmamız gereken yöntemin sırasıyla
decompose ve arima olmasına karar verdik.

    \subsection{Forecast}\label{forecast}

    \begin{Verbatim}[commandchars=\\\{\}]
{\color{incolor}In [{\color{incolor}64}]:} \PY{k+kn}{require}\PY{p}{(}forecast\PY{p}{)}
\end{Verbatim}


    İlk olarak verimizin zaman serisi olduğunu gösteriyoruz. Zaman serisi
sıklığını aylık olarak tanımlıyoruz. Zaman aralığımızı 01.1981 - 11.2017
kadar olan süreyi alıyoruz. 12.2017-02.2018 süresini de test verisi
olarak alıyoruz.

    \begin{Verbatim}[commandchars=\\\{\}]
{\color{incolor}In [{\color{incolor}65}]:} time\PYZus{}series3 \PY{o}{\PYZlt{}\PYZhy{}}ts\PY{p}{(}data2\PY{p}{[}\PY{p}{,}\PY{l+m}{2}\PY{p}{]}\PY{p}{,} start\PY{o}{=}\PY{k+kt}{c}\PY{p}{(}\PY{l+m}{1981}\PY{p}{,} \PY{l+m}{1}\PY{p}{)}\PY{p}{,} end\PY{o}{=}\PY{k+kt}{c}\PY{p}{(}\PY{l+m}{2017}\PY{p}{,} \PY{l+m}{11}\PY{p}{)}\PY{p}{,} freq\PY{o}{=}\PY{l+m}{12}\PY{p}{)}
         plot\PY{p}{(}time\PYZus{}series3\PY{p}{)}
\end{Verbatim}


    \begin{center}
    \adjustimage{max size={0.9\linewidth}{0.9\paperheight}}{output_174_0.png}
    \end{center}
    { \hspace*{\fill} \\}
    \subsubsection{Üstel Düzleştirme Yöntemi- Exponential Smooting}
    Aşağıda üstel düzleştirme yönteminin ets() fonksiyonu kullanılmış
halini uygulayacağız. Öngörü doğruluğunu hesaplamak için R'ın accuracy
fonksiyonunu kullanılır. Accuracy fonksiyonu ile gelen verilerde:

    Öngörü modellerinden ilk olarak ets modelini deniyoruz.

    \begin{Verbatim}[commandchars=\\\{\}]
{\color{incolor}In [{\color{incolor}66}]:} model11 \PY{o}{\PYZlt{}\PYZhy{}}ets\PY{p}{(}time\PYZus{}series3\PY{p}{)}
         data\PYZus{}forecast\PYZus{}ets \PY{o}{\PYZlt{}\PYZhy{}} forecast\PY{p}{(}model11\PY{p}{,} h\PY{o}{=}\PY{l+m}{3}\PY{p}{)}
         accuracy\PY{p}{(}data\PYZus{}forecast\PYZus{}ets\PY{p}{)}
         plot\PY{p}{(}data\PYZus{}forecast\PYZus{}ets\PY{p}{,} \PY{p}{,}xlim\PY{o}{=}\PY{k+kt}{c}\PY{p}{(}\PY{l+m}{2017.12}\PY{p}{,}\PY{l+m}{2018.2}\PY{p}{)}\PY{p}{)}
\end{Verbatim}


    \begin{tabular}{r|lllllll}
  & ME & RMSE & MAE & MPE & MAPE & MASE & ACF1\\
\hline
	Training set & 2.774238   & 39.27737   & 26.06343   & -0.3375955 & 7.387411   & 0.2734303  & 0.04872081\\
\end{tabular}


    
    \begin{center}
    \adjustimage{max size={0.9\linewidth}{0.9\paperheight}}{output_177_1.png}
    \end{center}
    { \hspace*{\fill} \\}
    
    Verimize ets() fonksiyonu ile öngörü hesapladığımızda bize ETS(A,A,A)
argümanlarını vermiştir. Bu sonuçta öngörüde toplamsal hata, toplamsal
trend ve toplamsal sezonsal etkinin uygulandığı yöntemdir denilebilir.

    \begin{Verbatim}[commandchars=\\\{\}]
{\color{incolor}In [{\color{incolor}67}]:} \PY{k+kp}{summary}\PY{p}{(}data\PYZus{}forecast\PYZus{}ets\PY{p}{)}
\end{Verbatim}


    \begin{Verbatim}[commandchars=\\\{\}]

Forecast method: ETS(A,A,A)

Model Information:
ETS(A,A,A) 

Call:
 ets(y = time\_series3) 

  Smoothing parameters:
    alpha = 0.6024 
    beta  = 0.0461 
    gamma = 0.1034 

  Initial states:
    l = 166.9415 
    b = 2.3989 
    s=4.2413 -11.3018 5.7337 -9.4752 -6.5625 -2.0671
           -5.4372 19.2079 3.535 19.4916 9.5222 -26.8879

  sigma:  39.2774

     AIC     AICc      BIC 
5985.646 5987.086 6055.237 

Error measures:
                   ME     RMSE      MAE        MPE     MAPE      MASE
Training set 2.774238 39.27737 26.06343 -0.3375955 7.387411 0.2734303
                   ACF1
Training set 0.04872081

Forecasts:
         Point Forecast    Lo 80    Hi 80    Lo 95    Hi 95
Dec 2017       3142.650 3092.314 3192.986 3065.667 3219.632
Jan 2018       3101.060 3041.067 3161.054 3009.309 3192.812
Feb 2018       3260.957 3191.518 3330.395 3154.760 3367.153

    \end{Verbatim}

    ETS yöntemiyle gelen alpha değeri 0.6024 çıkmıştır. Bu sonuçta değişen
bileşenlerden ortalama bir derecede etkinleneceğini göstermektedir.
\subsubsection{Basit Üstel Düzleştirme Yöntemi (Durağan Seriler için)}
    Aşağıda üstel düzleştirme yönteminin HoltWinters() fonksiyonu
kullanılmış halini uygulayacağız.Bu model trend, sezonsal bileşenleri de
içermemediği durumdur.

    \begin{Verbatim}[commandchars=\\\{\}]
{\color{incolor}In [{\color{incolor}68}]:} model12 \PY{o}{\PYZlt{}\PYZhy{}}HoltWinters\PY{p}{(}time\PYZus{}series3\PY{p}{,} beta\PY{o}{=}\PY{k+kc}{FALSE}\PY{p}{,} gamma\PY{o}{=}\PY{k+kc}{FALSE}\PY{p}{)}
         model12
         data\PYZus{}forecast\PYZus{}simple \PY{o}{\PYZlt{}\PYZhy{}} forecast\PY{p}{(}model12\PY{p}{,} h\PY{o}{=}\PY{l+m}{3}\PY{p}{)}
         accuracy\PY{p}{(}data\PYZus{}forecast\PYZus{}simple\PY{p}{)}
         plot\PY{p}{(}data\PYZus{}forecast\PYZus{}simple\PY{p}{,} \PY{p}{,}xlim\PY{o}{=}\PY{k+kt}{c}\PY{p}{(}\PY{l+m}{2017.11}\PY{p}{,}\PY{l+m}{2018.2}\PY{p}{)}\PY{p}{)}
\end{Verbatim}


    
    \begin{verbatim}
Holt-Winters exponential smoothing without trend and without seasonal component.

Call:
HoltWinters(x = time_series3, beta = FALSE, gamma = FALSE)

Smoothing parameters:
 alpha: 0.7810215
 beta : FALSE
 gamma: FALSE

Coefficients:
      [,1]
a 3051.299
    \end{verbatim}

    
    \begin{tabular}{r|lllllll}
  & ME & RMSE & MAE & MPE & MAPE & MASE & ACF1\\
\hline
	Training set & 8.410204    & 47.56481    & 30.51237    & 0.03430261  & 7.450028    & 0.320104    & -0.06453352\\
\end{tabular}


    
    \begin{center}
    \adjustimage{max size={0.9\linewidth}{0.9\paperheight}}{output_182_2.png}
    \end{center}
    { \hspace*{\fill} \\}
    
    \begin{Verbatim}[commandchars=\\\{\}]
{\color{incolor}In [{\color{incolor}69}]:} \PY{k+kp}{summary}\PY{p}{(}data\PYZus{}forecast\PYZus{}simple\PY{p}{)}
\end{Verbatim}


    \begin{Verbatim}[commandchars=\\\{\}]

Forecast method: HoltWinters

Model Information:
Holt-Winters exponential smoothing without trend and without seasonal component.

Call:
HoltWinters(x = time\_series3, beta = FALSE, gamma = FALSE)

Smoothing parameters:
 alpha: 0.7810215
 beta : FALSE
 gamma: FALSE

Coefficients:
      [,1]
a 3051.299

Error measures:
                   ME     RMSE      MAE        MPE     MAPE     MASE
Training set 8.410204 47.56481 30.51237 0.03430261 7.450028 0.320104
                    ACF1
Training set -0.06453352

Forecasts:
         Point Forecast    Lo 80    Hi 80    Lo 95    Hi 95
Dec 2017       3051.299 2991.235 3111.363 2959.439 3143.159
Jan 2018       3051.299 2975.086 3127.512 2934.741 3167.857
Feb 2018       3051.299 2961.805 3140.793 2914.430 3188.168

    \end{Verbatim}

    Düzleştirme katsayı değeri (alpha) arttıkça daha büyük tahmin
farklılıkları oluşmaktadır. Basit üstel düzleştirme yöntemi
uyguladığımızda gelen alpha değeri 0.7810215'dir.

    \subsubsection{Holt Üstel Düzleştirme Yöntemi D: (Doğrusal Trendi Olan Seriler için)}\label{holt-uxfcstel-duxfczleux15ftirme-yuxf6ntemi-doux11frusal-trendi-olan-seriler-iuxe7in}

Trende sahip mevsimsel dalgalanması olmayan serilerin tahmin işleminde
Holt üstel düzleştirme yöntemi kullanılmaktadır.

    \begin{Verbatim}[commandchars=\\\{\}]
{\color{incolor}In [{\color{incolor}70}]:} model13 \PY{o}{\PYZlt{}\PYZhy{}}HoltWinters\PY{p}{(}time\PYZus{}series3\PY{p}{,} gamma\PY{o}{=}\PY{k+kc}{FALSE}\PY{p}{)}
         model13
         data\PYZus{}forecast\PYZus{}holt \PY{o}{\PYZlt{}\PYZhy{}} forecast\PY{p}{(}model13\PY{p}{,} h\PY{o}{=}\PY{l+m}{3}\PY{p}{)}
         accuracy\PY{p}{(}data\PYZus{}forecast\PYZus{}holt\PY{p}{)}
         plot\PY{p}{(}data\PYZus{}forecast\PYZus{}holt\PY{p}{,} \PY{p}{,}xlim\PY{o}{=}\PY{k+kt}{c}\PY{p}{(}\PY{l+m}{2017.11}\PY{p}{,}\PY{l+m}{2018.2}\PY{p}{)}\PY{p}{)}
         data\PYZus{}forecast\PYZus{}holt
\end{Verbatim}


    
    \begin{verbatim}
Holt-Winters exponential smoothing with trend and without seasonal component.

Call:
HoltWinters(x = time_series3, gamma = FALSE)

Smoothing parameters:
 alpha: 0.5032057
 beta : 0.09194811
 gamma: FALSE

Coefficients:
        [,1]
a 3050.39138
b   62.40611
    \end{verbatim}

    
    \begin{tabular}{r|lllllll}
  & ME & RMSE & MAE & MPE & MAPE & MASE & ACF1\\
\hline
	Training set & 2.323311   & 43.32455   & 29.40548   & -0.857256  & 7.895946   & 0.3084917  & 0.02210322\\
\end{tabular}


    
    
    \begin{verbatim}
         Point Forecast    Lo 80    Hi 80    Lo 95    Hi 95
Dec 2017       3112.797 3057.292 3168.303 3027.909 3197.686
Jan 2018       3175.204 3111.871 3238.537 3078.344 3272.063
Feb 2018       3237.610 3166.164 3309.056 3128.343 3346.877
    \end{verbatim}

    
    \begin{center}
    \adjustimage{max size={0.9\linewidth}{0.9\paperheight}}{output_186_3.png}
    \end{center}
    { \hspace*{\fill} \\}
    
    \begin{Verbatim}[commandchars=\\\{\}]
{\color{incolor}In [{\color{incolor}71}]:} \PY{k+kp}{summary}\PY{p}{(}data\PYZus{}forecast\PYZus{}holt\PY{p}{)}
\end{Verbatim}


    \begin{Verbatim}[commandchars=\\\{\}]

Forecast method: HoltWinters

Model Information:
Holt-Winters exponential smoothing with trend and without seasonal component.

Call:
HoltWinters(x = time\_series3, gamma = FALSE)

Smoothing parameters:
 alpha: 0.5032057
 beta : 0.09194811
 gamma: FALSE

Coefficients:
        [,1]
a 3050.39138
b   62.40611

Error measures:
                   ME     RMSE      MAE       MPE     MAPE      MASE       ACF1
Training set 2.323311 43.32455 29.40548 -0.857256 7.895946 0.3084917 0.02210322

Forecasts:
         Point Forecast    Lo 80    Hi 80    Lo 95    Hi 95
Dec 2017       3112.797 3057.292 3168.303 3027.909 3197.686
Jan 2018       3175.204 3111.871 3238.537 3078.344 3272.063
Feb 2018       3237.610 3166.164 3309.056 3128.343 3346.877

    \end{Verbatim}

    Düzleştirme katsayı değeri arttıkça daha büyük tahmin farklılıkları
oluşmaktadır. Basit üstel düzleştirme yöntemi uyguladığımızda gelen
alpha değeri 0.5032057'dir. Bu sonuçta değerin düşük olduğunu gösterir.
Hata oranı ortalama değerdedir.Diğer modellerde incelenip en uygun
öngörü model seçilmelidir.

    \subsubsection{Holt-Winters Yöntemi: (Doğrusal Trendi ve Mevsimlik Etkileri Olan Seriler için)}\label{holt-winters-yuxf6ntemi-doux11frusal-trendi-ve-mevsimlik-etkileri-olan-seriler-iuxe7in}

Holt-Winters Yöntemi üç temel düzeltme esitliğine sahiptir. Temel, trend
ve mevsimsellik için olan bu üç düzeltme sabiti ile yapılan düzeltmeler,
Holt'un Dogrusal Yöntemi'ne benzer olarak tek toplam esitlikte
toplanarak mevsimsellik esitligi elde etmektedir.

    \begin{Verbatim}[commandchars=\\\{\}]
{\color{incolor}In [{\color{incolor}72}]:} model14 \PY{o}{\PYZlt{}\PYZhy{}}HoltWinters\PY{p}{(}time\PYZus{}series3\PY{p}{)}
         model14
         data\PYZus{}forecast\PYZus{}holtWinter \PY{o}{\PYZlt{}\PYZhy{}} forecast\PY{p}{(}model14\PY{p}{,} h\PY{o}{=}\PY{l+m}{3}\PY{p}{)}
         accuracy\PY{p}{(}data\PYZus{}forecast\PYZus{}holtWinter\PY{p}{)}
         plot\PY{p}{(}data\PYZus{}forecast\PYZus{}holtWinter\PY{p}{,} \PY{p}{,}xlim\PY{o}{=}\PY{k+kt}{c}\PY{p}{(}\PY{l+m}{2017.1}\PY{p}{,}\PY{l+m}{2018.2}\PY{p}{)}\PY{p}{)}
         data\PYZus{}forecast\PYZus{}holtWinter
\end{Verbatim}


    
    \begin{verbatim}
Holt-Winters exponential smoothing with trend and additive seasonal component.

Call:
HoltWinters(x = time_series3)

Smoothing parameters:
 alpha: 0.6076044
 beta : 0.07480512
 gamma: 0.2946021

Coefficients:
           [,1]
a   3051.309931
b     58.897088
s1    38.646959
s2   -67.966539
s3    39.639302
s4    49.920330
s5    11.787548
s6    51.898138
s7   -17.438469
s8    -7.923449
s9    15.317955
s10  -10.934216
s11   40.844742
s12    6.342074
    \end{verbatim}

    
    \begin{tabular}{r|lllllll}
  & ME & RMSE & MAE & MPE & MAPE & MASE & ACF1\\
\hline
	Training set & 3.013304   & 39.89378   & 26.40596   & -0.2875972 & 7.228475   & 0.2770238  & 0.05248844\\
\end{tabular}


    
    
    \begin{verbatim}
         Point Forecast    Lo 80    Hi 80    Lo 95    Hi 95
Dec 2017       3148.854 3097.815 3199.893 3070.796 3226.912
Jan 2018       3101.138 3040.179 3162.096 3007.909 3194.366
Feb 2018       3267.640 3197.022 3338.259 3159.639 3375.642
    \end{verbatim}

    
    \begin{center}
    \adjustimage{max size={0.9\linewidth}{0.9\paperheight}}{output_190_3.png}
    \end{center}
    { \hspace*{\fill} \\}
    
    \begin{Verbatim}[commandchars=\\\{\}]
{\color{incolor}In [{\color{incolor}73}]:} \PY{k+kp}{summary}\PY{p}{(}data\PYZus{}forecast\PYZus{}holtWinter\PY{p}{)}
\end{Verbatim}


    \begin{Verbatim}[commandchars=\\\{\}]

Forecast method: HoltWinters

Model Information:
Holt-Winters exponential smoothing with trend and additive seasonal component.

Call:
HoltWinters(x = time\_series3)

Smoothing parameters:
 alpha: 0.6076044
 beta : 0.07480512
 gamma: 0.2946021

Coefficients:
           [,1]
a   3051.309931
b     58.897088
s1    38.646959
s2   -67.966539
s3    39.639302
s4    49.920330
s5    11.787548
s6    51.898138
s7   -17.438469
s8    -7.923449
s9    15.317955
s10  -10.934216
s11   40.844742
s12    6.342074

Error measures:
                   ME     RMSE      MAE        MPE     MAPE      MASE
Training set 3.013304 39.89378 26.40596 -0.2875972 7.228475 0.2770238
                   ACF1
Training set 0.05248844

Forecasts:
         Point Forecast    Lo 80    Hi 80    Lo 95    Hi 95
Dec 2017       3148.854 3097.815 3199.893 3070.796 3226.912
Jan 2018       3101.138 3040.179 3162.096 3007.909 3194.366
Feb 2018       3267.640 3197.022 3338.259 3159.639 3375.642

    \end{Verbatim}

    Düzleştirme katsayı değeri arttıkça daha büyük tahmin farklılıkları
oluşmaktadır. Basit üstel düzleştirme yöntemi uyguladığımızda gelen
alpha değeri 0.6076044'dir. Bu sonuçta değerin büyük olduğunu gösterir.
Hata oranı fazladır. O yüzden öngörü yaparken bu model seçilmemelidir.
\pagebreak 

Aşağıda genel olarak bütün doğrulukları göstereceğiz.Hatırlamak için

ME:Ortalama Hata

MAE : Ortalama Mutlak Hata

MSE: Ortalama Hata Kare

MPE: Ortalama Yüzde Hata

MAPE: Ortalama Mutlak Yüzde Hata

    \begin{Verbatim}[commandchars=\\\{\}]
{\color{incolor}In [{\color{incolor}74}]:} accuracy\PY{p}{(}data\PYZus{}forecast\PYZus{}ets\PY{p}{)}
         accuracy\PY{p}{(}data\PYZus{}forecast\PYZus{}simple\PY{p}{)}
         accuracy\PY{p}{(}data\PYZus{}forecast\PYZus{}holt\PY{p}{)}
         accuracy\PY{p}{(}data\PYZus{}forecast\PYZus{}holtWinter\PY{p}{)}
\end{Verbatim}


    \begin{tabular}{r|lllllll}
  & ME & RMSE & MAE & MPE & MAPE & MASE & ACF1\\
\hline
	Training set & 2.774238   & 39.27737   & 26.06343   & -0.3375955 & 7.387411   & 0.2734303  & 0.04872081\\
\end{tabular}


    
    \begin{tabular}{r|lllllll}
  & ME & RMSE & MAE & MPE & MAPE & MASE & ACF1\\
\hline
	Training set & 8.410204    & 47.56481    & 30.51237    & 0.03430261  & 7.450028    & 0.320104    & -0.06453352\\
\end{tabular}


    
    \begin{tabular}{r|lllllll}
  & ME & RMSE & MAE & MPE & MAPE & MASE & ACF1\\
\hline
	Training set & 2.323311   & 43.32455   & 29.40548   & -0.857256  & 7.895946   & 0.3084917  & 0.02210322\\
\end{tabular}


    
    \begin{tabular}{r|lllllll}
  & ME & RMSE & MAE & MPE & MAPE & MASE & ACF1\\
\hline
	Training set & 3.013304   & 39.89378   & 26.40596   & -0.2875972 & 7.228475   & 0.2770238  & 0.05248844\\
\end{tabular}


    
    Yapılan 5 farklı öngörü modelinin hata değerlerini karşılaştırmak için
literatürde en çok kullanılan ve tercih edilen Ortalama Mutlak Hata
(MAE) değerini kullanılırız. Bu karşılaştırmaya göre ets yöntemi en
uygun yöntemdir.

    \subsubsection{Florida Field Production of Crude Oil - Zaman Serilerinde Öngörü Özet}\label{florida-field-production-of-crude-oil---zaman-serilerinde-uxf6nguxf6ruxfc-uxf6zet}

Verimizde öngörü yapabilmek için tahminleme verisi ve test verisi olmak
üzere iki veri seti oluşturduk.Tahminleme verisini Ocak 1981 ile Kasım
2017 arasını , test verisi için son üç ayı yani Aralık 2017 ile Şubat
2018 arasında aldık. Sırasıyla ets,holtwinters modellerini uyguladık. Hata terimleri incelendiğinde arima modelinin öngörü için daha uygun olduğuna karar verdik.

    Petrol verileri ile ilgili analiz ve öngörü hesaplamalarını yapmış olduk.
\pagebreak
    \section{Kaynakça}\label{kaynakuxe7a}

\begin{itemize}
\tightlist
\item
  Brockwell, Richard A. Davis (2002). Introduction to Time Series and
  Forecasting
\item
  Box, G. E. P., Jenkins, G. M., and Reinsel, G. C. (1994). Time Series
  Analysis, Forecasting and Control, 3rd ed. Prentice Hall, Englewood
  Clifs, NJ.
\item
  Cryer, J. D. and Chan, K. S. (2008). Time series analysis: with
  applications in R.Springer, 106-107.
\item
  Dickey, D. A. and W. A. Fuller (1981) Likelihood ratio statistics for
  autoregressive time series with a unit root, Econometrica, 49,
  1057--1071.
\item
  Frances PH (1991). Seasonality, non-stationarity, and the forecasting
  of monthly time series.International Journal of Forecasting, 7:
  199--208.
\item
  Introduction to Forecasting with ARIMA in R,
  https://www.datascience.com/blog/introduction-to-forecasting-with-arima-in-r-learn-data-science-tutorials
\item
  Kayım, H., 1985. İstatistiksel ön tahmin yöntemleri: Hacettepe
  Üniversitesi İktisadi ve İdari Bilimler Fakültesi.
\item
  Montgomery, D. C., Johnson, L. A. and Gardiner, J. S., 1990.
  Forecasting and time series analysis: McGraw-Hill Companies.
\item
  The R Project for Statistical Computing , https://www.r-project.org/
\end{itemize}


    % Add a bibliography block to the postdoc
    
    
    
    \end{document}
